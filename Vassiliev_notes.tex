\documentclass{ctexart}

\usepackage{amsmath,amssymb,mathrsfs,tikz-cd}
\usepackage{changepage}

% \setlength{\parindent}{0pt}
\pagestyle{plain}

\newcommand{\sk}[2]{\mathrm{sk}_{#1}#2}
\newcommand{\Rn}{\mathbb{R}^n}

\begin{document}
\begin{enumerate}
\item A \textit{basis} of a topology $\tau = \{ V_\alpha \}$ is a subset
$\{ W_\lambda \} \subset \tau$ such that each open set can be represented
as the union of a (probably infinite) family of sets from $W_\lambda$.

\item A map $f : X \to Y$ of a topological space $X$ to a topological space
$Y$ is called \textit{continuous} with respect to the given topologies
if the preimage of any open set in $Y$ is open in $X$.

\item \textbf{Induced topology.} Let $Y$ be a topological space, and
  let $X \subset Y$ be a subset. Then one can introduce a topology on
  $X$ by considering the intersections of $X$ with open subsets in Y
  as open subsets in $X$. This topology is said to be the induced
  topology.

In a more general setting, the induced topology is defined as
follows. Let $Y$ be a topological space, and let $f : X \to Y$ be a
map. The \textit{induced topology} in $X$ consists of all preimages of
open sets in $Y$. (The map $f$ is then continuous with respect to the
induced topology.). For the inclusion map $f$ of a subset $X \subset
Y$ we recover the previous definition.

\item \textbf{Separability.} A topological space X is said to be a
  \textit{Hausdorff space} if any two distinct points $x,y \in X$
  possess nonintersecting neighborhoods. (A \textit{neighborhood} of a
  point $x$ is an arbitrary open set containing $x$.)

\item \textbf{Quotient topology.} Suppose an equivalence relation $\sim$ is given on a topological space $X$, i.e., a subset $A \subset X \times X$ is chosen so that
    \begin{adjustwidth}{2em}{}
    \begin{enumerate}
      \item $(x,x) \in A$ for all $x \in X$;
      \item $(x,y) \in A \Rightarrow (y,x) \in A$;
      \item $(x,y), (x,z) \in A \Rightarrow (y,z) \in A$.
    \end{enumerate}
    \end{adjustwidth}
    Then a topology, called the \textit{quotient topology}, on the set of equivalence classes $X/\!\!\sim$ is defined in the following way. A set $U$ is open if its preimage under the canonical projection $X \to X/\!\!\sim$ is open. The space $X/\!\!\sim$ supplied with the quotient topology is called the \textit{quotient space} of $X$ by $\sim$.

\item \textbf{Definition.} A topological space $X$ is \textit{path-connected} if for any pair of its points $x,y$ there is a continuous map $f : [0,1] \to X$ such that $f(0) = x$ and $f(1) = y$. Maximal path-connected topological subspaces of a topological space will be called its \textit{path-connected components} or briefly \textit{path-components}.

\item \textbf{Suspension.} Let $X$ be a topological space. The suspension $\Sigma X$ is defined as the quotient space $X \times I/\!\!\sim$, where $I$ is the segment $[-1,1]$ and the equivalence relation is the following:
    \begin{adjustwidth}{2em}{}
    for $\lambda \neq -1,1\ (x,\lambda) \sim (y,\mu) \Longleftrightarrow x = y, \lambda = \mu$;

    $(x,1) \sim (y,1)$ for all $x,y \in X$;

    $(x,-1) \sim (y,-1)$ for all $x,y \in X$.
    \end{adjustwidth}

In other words, we take the \textit{cylinder} $X \times I$ and contract its upper base $X \times \{1\}$ to a point and its lower base $X \times \{-1\}$ to a point.

\textbf{Example.} The \textit{n-dimensional sphere} $S^n$ is the topological space homeomorphic to the subset in $\mathbb{R}^{n+1}$ specified by the equation $x_1^2 + \cdots + x_{n+1}^2 = 1$. It can be proved that $\Sigma S^n \approx S^{n+1}$ (this statement holds for $S^0 = \{-1,1\}$ as well).

\item \textbf{Attaching a topological space by a map.} Let $X$ and $Y$ be disjoint topological spaces, let $A \subset X$ and let $\varphi : A \to Y$ be a continuous map. Then the operation of \textit{attaching $X$ to $Y$ by the map $\varphi$} produces the space
    \[
    X \cup_\varphi Y = (X \cup Y) / \{ a \sim \varphi(a) \};
    \]
this means that a point $a \in A$ is equivalent to the point $\varphi(a)$, while any point not in $A$ and not in the image of $\varphi$ is equivalent only to itself.

The space $X \cup Y$ (here the union is understood in the usual sense) can be treated as $X$ attached to $Y$ by the identical inclusion $X \cap Y \hookrightarrow Y$.

\item \textbf{Join.} The space $X * Y$ can be visualized as follows: we connect each point of $X$ with each point of $Y$ by a segment, and then take the union of all these segments. (The segments are not allowed to have common internal points.) For example, $S^0 * S^0 \approx S^1$.

    The formal definition is
    \[
    X * Y = X \times [-1,1] \times Y / \sim,
    \]
where the equivalence relation is given by the following rule:
\begin{adjustwidth}{2em}{}
  for $\lambda \neq -1,1$ each point $(x,\lambda,y)$ is equivalent only to itself;

  $(x,-1,y) \sim (x,-1,y')$ for all $x \in X$ and $y,y' \in Y$;

  $(x,1,y) \sim (x',1,y)$ for all $x,x' \in X$ and $y \in Y$.
\end{adjustwidth}

\textbf{定义.} 拓扑空间 $X$ 称为\textit{列紧的}, 若 $X$ 中任意无限点列 $x_1,x_2,\cdots$ 有一个收敛于 $X$ 中某点的无限子列.

\item \textbf{Higher homotopy groups}

It is convenient to solve the recognition problem for topological spaces $X,X'$ by choosing a model set $A$ and considering the set of homotopy equivalence classes of continuous maps $[A,X]$. Depending on the specific problem, choosing an appropriate subset $B \subset A$ and fixing the map $B \to X$ can also be useful.

\qquad Suppose points $a_0 \in A$ and $x_0 \in X$ are fixed. Homotopy provides an equivalence relation on the set of all maps $A \to X$ taking $a_0$ to $x_0$. The set of all equivalence classes is denoted by $\Pi(A,X)$. The set $\Pi(S^n,X)$ is also denoted by $\pi_n(X)$.

\qquad Let us introduce a group structure on the set $\pi_n(X)$. (The group $\pi_n(X)$ is called the \textit{n-dimensional homotopy group}, or the \textit{nth homotopy group}.)

\textbf{Exercise.} Let $B^n$ be the $n$-dimensional disk (the ball) in $\mathbb{R}^n$ with boundary $S^{n-1}$. Then $B^n/S^{n-1}$ is homeomorphic to $S^n$.

\textbf{Corollary.} For any topological space $X$, the set of maps $(S^n,a_0) \to (X, x_0)$ can be identified with the set of maps $B^n \to X$ taking the whole boundary sphere $S^{n-1} = \partial B^n$ to the point $x_0$. This bijection determines a one-to-one correspondence between the homotopy classes of such maps.

\qquad The groups $\pi_n(X)$ form a set of topological invariants: homeomorphic spaces have isomorphic homotopy groups. The sets $\Pi(A,X)$ for arbitrary spaces $A$ form a more general class of topological invariants. However, if the set $\Pi(A,X)$ is not endowed with a algebraic structure, then the corresponding invariant is a very weak one.

\qquad There are examples of nonhomeomorphic topological spaces that the invariants of the form $\Pi(X,A)$ and $\Pi(A,X)$ cannot distinguish. A weaker equivalence relation, homotopy equivalence, is sometimes more useful than homeomorphism.

\item Topological spaces $X$ and $X'$ are called \textit{homotopy
    equivalent} if there exist continuous maps $f : X \to X'$ and $g :
  X' \to X$ such that the composite maps $fg$ and $gf$ are homotopic
  to the identity maps $id_{X'}$ and $id_X$, respectively. Such maps $f$
  and $g$ are called \textit{homotopy equivalences}.

\item If two topological spaces $X$ and $X'$ are homotopy equivalent,
  then all their homotopy groups are isomorphic, and for each $A$,
  there is a natural one-to-one correspondence between the sets
  $\Pi(A,X)$ and $\Pi(A,X')$, as well as between the sets $\Pi(X,A)$
  and $\Pi(X',A)$.

\textbf{Definition.} Let $A$ and $B$ be topological spaces with the basepoints $a_0 \in A$ and $b_0 \in B$ respectively. Consider the disjoint union of $A$ and $B$ modulo the equivalence relation $a_0 \sim b_0$ gluing the points $a_0$ and $b_0$. The topological space thus obtained is called the \textit{wedge product} of the spaces $A$ and $B$ and is denoted by $(A,a_0) \vee (B,b_0)$.

\item A map $p : X \to Y$, where the space $Y$ is path-connected, is
  called a \textit{covering} if for any point $y \in Y$ there is a
  neighborhood $U$ of $y$ such that the preimage $p^{-1}(U)$ is
  homeomorphic to a number of copies of $U$ (i.e., $p^{-1}(U) \approx
  U \times \Delta$, where $\Delta$ is a discrete topological space),
  and this homeomorphism is compatible with the map $p$, i.e., the
  natural projection $U \times \Delta \to U$ coincides with $p$. If $\Delta$
  consists of $k$ points, then the covering is called a
  \textit{k-fold} or \textit{k-sheeted} covering.

\item \textbf{覆叠同伦定理.} 考虑一个覆叠 $p : E \to X$. 设我们已给一映射 $f : Y \to E$ 以及映射 $p \circ f : Y \to X$ 的同伦, 即一映射 $F : Y \times [0,1] \to X$, 它在 $Y\times \{0\}$ 上与 $p\circ f$ 一致. 若空间 $Y$ 不太差(即若 $Y$ 为道路连通), 则同伦 $F$ 唯一地提升成一个 $f$ 的同伦, 就是说, 存在一个映射 $\Phi : Y\times [0,1]\to E$ 使
\begin{adjustwidth}{2em}{}
    \begin{enumerate}
      \item $\Phi$ 在 $Y\times \{0\}$ 上与 $f$ 一致;
      \item 方程 $p\Phi = F$ 成立.
    \end{enumerate}
\end{adjustwidth}
\begin{center}
\begin{tikzcd}
&& E \arrow[d, "p"] \\
Y \arrow[rru, "f"] \arrow[r, "\mathrm{id} \times 0"'] & Y \times [0,1]
\arrow[ru, dashrightarrow, "\Phi"'] \arrow[r, "F"'] & X
\end{tikzcd}
\end{center}

\textbf{定理.} 令 $X$ 为一局部 $1$-连通(特别地,局部道路连通)拓扑空间, 则下面结论成立:
\begin{enumerate}
  \item 对任一覆叠映射 $p: E \to X$, 映射 $p_\ast: \pi_1(E) \to \pi_1(X)$ 为一单同态;
  \item 在集 $p^{-1}(x_0)$ 与 $\pi_1(X)/p_\ast(\pi_1(E))$ 的陪集之间存在一一对应.
  \item 对 $\pi_1(X)$ 中的任一子群 $G$, 存在一个覆叠 $p : E \to X$ 使得 $p_\ast(\pi_1(E)) = G$ 成立.
  \item 
  \item 
\end{enumerate}

\item Usually, the objects of topology are "sufficiently good"
  topological spaces, mostly, cell spaces. Other kinds of spaces are
  used mainly to construct various counterexamples.

\item Let $X$ be a Hausdorff topological space. A \textit{cell space
    structure} on $X$ is a decomposition of $X$ into a disjoint union
  of subsets $B_\alpha^k$ homeomorphic to open disks (possibly, of
  different dimension). For each $B_\alpha^k \subset X$ a
  \textit{characteristic homeomorphism} $\chi_\alpha^k : D^k \to
  B_\alpha^k$, where $D^k$ is an open $k$-disk, must be fixed. We
  assume that this homeomorphism admits an extension to a continuous
  map (the characteristic map) of the closed disk $\overline{D^k} \to X$
  satisfying the following conditions (\textit{cell space axioms}):

\textbf{(C)} The image of the boundary of the disk $\overline{D^k}$ is
contained in a finite set of cells $B_\beta^j$ of smaller dimensions
$j < k$.

\textbf{(W)} A subset $A \subset X$ is closed if its intersection with
the closure of any cell is closed.

\item The most import properties of cell spaces are concentrated in
  the Borsuk lemma, the cellular approximation theorem, and the local
  contractibility property of cell spaces.

\item \textbf{The Borsuk property.} Let $X$ be a topological space and
  let $A \subset X$ be a subspace. Denote by $f_A : A \to Y$ the
  restriction of a map $f : X \to Y$ to $A$. Suppose a homotopy $F_A :
  A \times I \to Y$ of the map $f_A$ is given, i.e., $F_A(a,0) =
  f_A(a)$ for $a \in A$. If for any topological space $Y$ and any map
  $f : X \to Y$ each homotopy $F_A$ extends to a homotopy $F : X
  \times I \to Y$ of $f$, then the pair of spaces $(X,A)$ is called a
  \textit{Borsuk pair}.

\item \textbf{Borsuk lemma.} If $X$ is a cell space and $A \subset X$
  is a cell subspace of $X$ (i.e., a cell space formed by a union of
  some cells of the space $X$), then $(X,A)$ is a Borsuk pair.

\item A cell space $X$ admits a natural filtration
\[
X^0 \subset X^1 \subset X^2 \subset \cdots \subset X,
\]
where $X^0$ is the union of all zero-dimensional cells, $X^1$ is the
union of all one-dimensional and zero-dimensional cells, $X^k$ is the
union of all cells of dimension not greater than $k$. Obviously, the
set $X^k \subset X$ also is a cell space; it is called the
\textit{k-skeleton} of the cell space $X$ and sometimes also denoted
by $\sk{k}{X}$.

\item Let $X$ and $Y$ be cell spaces. A continuous map $f : X \to Y$
  is called \textit{cellular} if $f(\sk{n}{X}) \subset \sk{n}{Y}$ for $n
  = 0,1,2,...$.

\item \textbf{Cellular approximation theorem.} Any map of a cell space
  to a cell space is homotopic to a cellular map.

\item \textbf{Corollary.} For any cell space $X$ the group $\pi_i(X)$
  is isomorphic to the group $\pi_i(\sk{i+1}{X})$.

\item \textbf{Definition.} A topological space $X$ is called
  \textit{k-connected} if it is path-connected and the groups
  $\pi_1(X), \pi_2(X),\dots, \pi_k(X)$ are trivial. $1$-connected
  spaces are also called \textit{simply connected}.

\item \textbf{Definition.} A \textit{cellular pair} $(X,Y)$ is a pair
  consisting of a cellular space $X$ and its cellular subspace $Y
  \subset X$.

\item \textbf{Theorem.} Suppose $(X,Y)$ is a cellular pair and the
  space $Y$ is contractible. Then the space $X/Y$ is homotopy
  equivalent to $X$.

\item \textbf{Theorem.} If a cell space $X$ is k-connected (in
  particular, path connected), then $X$ is homotopy equivalent to a
  cell space $X'$ with precisely one zero-dimensional cell and no
  cells of dimensions $1,2,\dots,k$.

\item \textbf{Definition.} Two coverings $p : X \to Y$ and $p' : X'
  \to Y'$ are said \textit{equivalent} if there are homeomorphisms $f
  : X \to X'$ and $g : Y \to Y'$ such that $gp = p'f$. Coverings $p :
  X \to Y$ and $p' : X' \to Y$ over the same base $Y$ are said to be
  \textit{equivalent} if there is a homeomorphism $f : X \to X'$ such
  that $p = p'f$.

\item \textbf{Covering homotopy theorem.} Consider a covering $p : E
  \to X$. Suppose we are given a map $f : Y \to E$ and a homotopy of
  the map $p \circ f : Y \to X$, i.e., a map $F : Y \times I \to X$,
  coinciding with $p \circ f$ on $Y \times \{0\}$. If the space $Y$ is
  not too bad (namely, if $Y$ is locally path-connected), then the
  homotopy $F$ lifts uniquely to a homotopy of f, i.e., there is a map
  $\Phi : Y \times I \to E$ such that

(1) $\Phi$ coincides with $f$ on $Y \times \{0\}$;

(2) the equation $p\Phi = F$ holds.

\begin{tikzcd}
Y \arrow[r,"f"] \arrow[d,"Y\times \{0\}"'] & E \arrow [d,"p"]\\
Y \times I \arrow[ru,dashrightarrow,"\Phi"] \arrow[r,"F"'] & X
\end{tikzcd}

\item \textbf{Theorem.} Let $X$ be a locally $1$-connected (in
  particular, locally path connected) topological space. Then the
  following hold:

(1) For any covering $p : E \to X$ the map $p_* : \pi_1(E) \to
\pi_1(X)$ is a monomorphism.

(2) There is a one-to-one correspondence between the set
$\pi^{-1}(x_0)$ and the set of cosets $\pi_1(X)/p_*\pi_1(E)$.

(3) For any subgroup $G$ in $\pi_1(X)$ there is a covering $p : E \to
X$ such that $p_*\pi_1(E) = G$.

\item The notion of a covering can be generalized to the case in which
  the preimage of a point is not discrete.

\textbf{Definition.} A \textit{locally trivial fiber bundle} is a
quadruple $(E,B,F,p)$, where $E,B$ and $F$ are topological spaces, and
$p : E \to B$ is a map possessing the following properties:

1) any point $x \in B$ admits a neighborhood $U$ with the preimage
$p^{-1}(U)$ homeomorphic to $U \times F$;

2) the homeomorphism $U \times F \to p^{-1}(U)$ is consistent with the
map $p$, i.e., the diagram

\begin{tikzcd}
U \times F \arrow[rr] \arrow[dr] & & p^{-1}(U) \arrow[ld,"p"] \\
& U
\end{tikzcd}

is commutative.

Condition 2 implies, in particular, that the preimage of each point is
homeomorphic to $F$.

The map $p$ is called the \textit{projection}, the space $E$ is called
the \textit{total space} of the bundle, the space $B$ is the \textit{base} of
the bundle, the space $F$ is the \textit{fiber}, or the
\textit{typical fiber}.

By abuse of language, we shall often write simply "fiber bundle"
instead of "locally trivial fiber bundle".

\item \textbf{Definition.} Two fiber bundles $p_1 : E_1 \to B$ and
  $p_2 : E_2 \to B$ are \textit{equivalent} if there is a
  homeomorphism $\varphi : E_1 \to E_2$ such that $p_1 = p_2\varphi$.

\item \textbf{Definition.} A \textit{trivialization} of a fiber bundle
  $p : E \to B$ is a homeomorphism $E \to B \times F$ of the form
\[
e \mapsto (p(e),p_1(e)),
\]
where $p_1(e) \in F$ (whenever such a homeomorphism exists).

\item A significant part of present research in topology is devoted to
  the sutdy of topological spaces that locally look like the vector
  space $\Rn$ of appropriate dimension. In this chapter we
  come step-by-step to the definition of such spaces, which are called
  smooth manifolds.

Recall some notions from calculus. A function $f : \Rn \to
\mathbb{R}$ belongs to the class $C^r$ if for $i \leq r$ all partial
derivatives $\partial^if/\partial x_{l_1}\dots\partial x_{l_i}$ exist
and are continuous. In this case the derivative is independent of the
order of differentiation.

A map $F : \Rn \to \Rn$ can be represented as an
$n$-tuple of functions $y_1 = f_1(x_1,\dots,x_n),\dots,y_n =
f_n(x_1,\dots,x_n)$. Suppose all these functions are $r$ times
continuously differentiable, where $r \geq 1$. The \textit{Jacobi
  matrix} of the map $F$ at a point $x = (x_1,\dots,x_n) \in
\Rn$ is the $n \times n$-matrix with entries $a_{ij}
= \partial f_i/\partial x_j$. The \textit{Jacobian} $J$ of $F$ is the
determinant of the Jacobi matrix.

\textbf{Inverse function theorem.} Let $F : \Rn \to \Rn$ be a map of
class $C^r, r \geq 1$, such that the Jacobian of $F$ differs from zero
for some point $x \in \Rn$. Then there are a neighborhood $O(x)$ of
the point $x$, a neighborhood $O(F(x))$ of the point $F(x)$, and a map
$\Phi : O(F(x)) \to O(x)$, which is inverse to $F$. The map $\Phi$
belongs to the class $C^r$.

In other words, the restriction of $F$ to $O(x)$ is a
$C^r$-diffeomorphism of the neighborhood $O(x)$ onto the neighborhood
$O(F(x))$; a $C^r$-\textit{diffeomorphism} is a map of class $C^r$
having an inverse map also belonging to the class $C^r$.
\end{enumerate}
\end{document} 