\documentclass{ctexart}

\usepackage{amsmath,amssymb,mathrsfs}

\setlength{\parindent}{0pt}

\begin{document}
Let $\Sigma$ be a partially ordered set. Given a subset $S\subset \Sigma$, an \textit{upper bound} of $S$ is an element $u\in\Sigma$ such that $s<u$ for all $s\in S$. A \textit{maximal element} of $\Sigma$ is an element $m\in \Sigma$ such that $m<s$ does not hold for any $s\in\Sigma$. A subset $S\subset\Sigma$ is \textit{totally ordered} if for every pair $s_1,s_2\in S$ of elements of $S$, either $s_1\leq s_2$ or $s_2\leq s_1$.

\textbf{Axiom (Zorn's lemma)} Suppose that $\Sigma$ is a nonempty set with a partial order $<$, and that any totally ordered subset $S\subset\Sigma$ has an upper bound in $\Sigma$. Then $\Sigma$ has a maximal element.
\end{document}