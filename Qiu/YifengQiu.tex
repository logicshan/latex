\documentclass[12pt]{ctexart}
\usepackage{amssymb,amsmath}
\usepackage[backend = biber, style = caspervector, citestyle = caspervector, utf8]{biblatex}
\addbibresource{YifengQiu.bib}

\usepackage[T1]{fontenc}
\usepackage[numbered,framed]{matlab-prettifier}

\lstset{
  style              = Matlab-editor,
  basicstyle         = \mlttfamily\small,
  escapechar         = ",
  mlshowsectionrules = true,
}

\textheight=8.0in \topmargin=0in \textwidth=6.05in
\oddsidemargin=0.28in \baselineskip=30pt %\linespread{1.5}
\date{}
\title{《The method of conjugate gradients used in inverse iteration》读书报告}
\author{邱一峰}
\pagestyle{plain}
\begin{document}
\maketitle
\begin{center}
\textbf{前言}
\end{center}

QL方法是求中小型实对称三角矩阵全部特征值的有效算法。通常,为了提高算法的收敛速度,它都要带特定的位移。常用的位移有(1)Rayleigh商位移;(2)Wilkinson位移;但是,Rayleigh商位移无法保证对任意的对称三对角阵都收敛。而Wilkinson位移,虽然Wilkinson猜测其具有三次的收敛率,但至今尚无人证明这一结果。所以本文就结合了Rayleigh商位移和Wilkinson位移的一些特点,提出了一种称之为RW的位移,它是一种收敛且有三阶敛速的位移。
\begin{center}
\textbf{预备知识}
\end{center}
\textbf{定义1:}记\[
T^{(k)} =
\renewcommand\arraystretch{1.5}
\begin{bmatrix}
\alpha_1^{(k)} & \beta_1^{(k)} & & & & & \\
\beta_1^{(k)}  & \alpha_2^{(k)} & \beta_2^{(k)} & & & & \\
              & \mathord{\cdot} & \mathord{\cdot} & \mathord{\cdot} & && & \\
              &                 & \mathord{\cdot} & \mathord{\cdot} & &\mathord{\cdot} & & \\
              &                 &                 & \mathord{\cdot} & &\mathord{\cdot} & \mathord{\cdot} & \\
              &                 &                 &                 & &\mathord{\cdot} & \mathord{\cdot} &
              \beta_{n-1}^{(k)} \\
              &                 &                 &                 & &               & \beta_{n-1}^{(k)} & \alpha_n^{(k)}
\end{bmatrix}, \hspace{1em} k = 0,1,2,...,
\]

令$T\equiv T^{(0)},\{\sigma_k\}$是一列称为位移的实数,且$\{\sigma_{k}\}\cap\sigma(T)=\varnothing$,这里$\sigma(T)$表示$T$ 的谱,作
\[
\begin{cases}
T^{(k)}-\sigma_k = Q_k  L_k \\
T^{(k+1)}=L_k Q_k+\sigma_k,k =0,1,2,...,
\end{cases}
\]
这里$Q_{k}$是正交阵,$L_{k}$是对角元大于零的下三角阵,则以上过程为QL算法。


\textbf{定义2:}若在带位移的QL 算法中有
$\beta_1^{(k)}\rightarrow0(k\rightarrow\infty)$,
则称算法是收敛的。此时若对充分大的k,成立

\begin{center}
$\beta_1^{(k+1)}=O([\beta_1^{(k)}]^{p}),(p\geq1)$
\end{center}
则称算法具有p阶渐近收敛速度,简称p阶收敛。

显然,若$\beta_1^{(k)}\rightarrow0(k\rightarrow\infty)$,则$\alpha_1^{(k)}$将趋于$T^{(k)}$的特征值,然而,QL迭代的实质是通过正交变换来生成新矩阵,所以迭代不改变特征值,从而,$\alpha_1^{(k)}$将趋于$T^{(0)}$的特征值.

为了简单起见,我们往往省略指标k,用$\beta_1$,$\hat{\beta_1}$,$T$,$\hat{T}$
分别表示 $\beta_1^{(k)}$,$\beta_1^{(k+1)}$,$T^{(k)},T^{(k+1)}$.
\begin{center}
\textbf{主要结果}
\end{center}

本文正是要构造出一种新的位移,使得该位移有以下性质:

(1) $\beta_1^{(k)}\rightarrow0(k\rightarrow\infty)$
(或者是,如果$|\beta_1^{(k)}|<\varepsilon$  ,其中$\varepsilon$ 是给定的很小的常数,则$\alpha_1^{(k)}$ 是$T^{(0)}$ 的特征值的一个好的近似。)

(2) $\beta_1$收敛于0的速度至少是三次,即至少为$\hat{\beta_1}=O(\beta_1^3)$

那么,我们如何来构造这个位移呢?

首先,我们考虑方程:
\begin{equation}
(\alpha_1-\lambda)(\alpha_2-\lambda)=k\beta_1^{2}
\end{equation}
其中,$k\in[0,1]$.

如果$\sigma$是(1)式的一个根,而且更接近于$\alpha_1$,则由论文中的引理4,知:
\[
l_{1,1}^2 \leq \frac{(1+k)\beta_2^2 + (1-k)^2\beta_1^2}{(1+k)\beta_1^2
  + \beta_2^2} \cdot \beta_1^2
\]


分析这个不等式,易知:

若$k=0$,则 $l_{1,1}^2\leq\beta_1^2$.

若$k=1$,则\[ l_{1,1}^2\leq\frac{2\beta_2^2\beta_1^2 }{2\beta_1^2
  + \beta_2^2}\].

而由(1)式可知:

若$k=0$时,$\alpha_1$为方程的根,此时,$\sigma=\alpha_1$为Rayleigh商位移;

若$k=1$时,方程变为$(\alpha_1-\lambda)(\alpha_2-\lambda)=\beta_1^{2}$,易得
\[\sigma=\alpha_1-sign(\delta)\cdot\frac{\beta_1^2 }{|\delta|+\sqrt{\delta^2+\beta_1^2}}
\].
这里$\delta=(\alpha_2-\alpha_1)/2$.
即为Wilkinson位移。

又若(1)\[\beta_1^{2}\leq\frac{2\beta_2^2\beta_1^2 }{2\beta_1^2 + \beta_2^2}\] 即可得$2\beta_1^{2}\leq\beta_2^{2}$.由以上可知此时$k=0$.

若(2)\[\beta_1^{2}>\frac{2\beta_2^2\beta_1^2 }{2\beta_1^2 + \beta_2^2}\] 即可得$2\beta_1^{2}>\beta_2^{2}$.由以上可知此时$k=1$.

综上所述,

\[
\sigma =
\begin{cases}
\alpha_1 & , \beta_2^2 \geq 2\beta_1^2 \\
\alpha_1 - \mathrm{sign}(\delta) \cdot
\frac{\beta_1^2}{\left\lvert\delta\right\rvert + \sqrt{\delta^2 +
    \beta_1^2}} & , \beta_2^2 < 2\beta_1^2
\end{cases}
\]



由以上又可以进一步推出:对于以上定义的$\sigma$,有:$l_{1,1}^2\leq\min(\beta_1^2,\frac{1}{\sqrt{2}}|\beta_1\beta_2|)$,
也即是论文中的引理5.

再由论文中的引理1可以得到:$|\hat{\beta_1}|\leq|l_{1,1}^2|$,由引理6可得:$|\hat{\beta_1\beta_2}|\leq|l_{1,1}^2\beta_1|$.

我们再结合引理5:
\[
|\hat{\beta_1}^2\hat{\beta_2}|=|\hat{\beta_1}|\cdot|\hat{\beta_1}\hat{\beta_2}|\leq|l_{1,1}|\cdot|l_{1,1}\beta_1|
=l_{1,1}^2|\beta_1|\leq\frac{1}{\sqrt{2}}|\beta_1\beta_2|=\frac{1}{\sqrt{2}}|\beta_1^2\beta_2|
\]

所以,$\beta_1^2\beta_2\rightarrow0$.再由引理1和引理5最终可以推出:
\begin{center}
$|\hat{\beta_1}^3|\leq|l_{1,1}^3|\leq|\beta_1|\cdot\frac{1}{\sqrt{2}}|\beta_1\beta_2|=\frac{1}{\sqrt{2}}|\beta_1^2\beta_2|$
\end{center}

因此,$\beta_1^3\rightarrow0$与$\beta_1\rightarrow0$同时成立.

再又论文中的定理2,可得到:$\beta_1$收敛于0的速度至少是三次,即至少为$\hat{\beta_1}=O(\beta_1^3)$

所以,我们想要选取的新位移即为:
\[
\sigma =
\begin{cases}
\alpha_1 & , \beta_2^2 \geq 2\beta_1^2 \\
\alpha_1 - \mathrm{sign}(\delta) \cdot
\frac{\beta_1^2}{\left\lvert\delta\right\rvert + \sqrt{\delta^2 +
    \beta_1^2}} & , \beta_2^2 < 2\beta_1^2
\end{cases}
\]

\begin{center}
\textbf{数值实验与程序}
\end{center}

下面给出一个数值例子.
给定一个11阶的对称三对角阵$T$:
\begin{center}
$\alpha_j=2j-1,j=1,2,...,11,$
\end{center}
\begin{center}
$\beta_1=\beta_3=\beta_4=...=\beta_{10}=1$
\end{center}
对一些不同的$\beta_2$用三种不同的位移做QL迭代,从而找得到$T$的9个不同的特征值,直到
\begin{center}
$|\beta_i^{(k)}|<10^{-10},i=1,2,...,9$
\end{center}

另外,我们通过Matlab来执行这个数值实验,但Matlab中只有现成的QR分解的函数,而没有QL 分解的函数,所以在此需要做一下处理:

设$T$为给定的对称三对角阵,对$(T')^{-1}$进行QR分解:
\[(T')^{-1}=QR
\Rightarrow T'=R^{(-1)}Q^{(-1)}
\Rightarrow T=(Q^{(-1)})'(R^{(-1)})'
\]
因为$R$为上三角矩阵,则$(R^{(-1)})$为下三角矩阵;而$Q$为正交阵,$(Q^{(-1)})'$仍然为正交阵。令$Q_1=(Q^{(-1)})',L=(R^{(-1)})'$,即:$T=Q_1L$

$T$的QL分解完成。


程序如下:


(1)按要求生成矩阵$T$。

\lstinputlisting{create.m}


(2)新位移——RW位移的QL迭代
\lstinputlisting{RW_S.m}

(3)Rayleigh商位移的QL迭代
\lstinputlisting{Rayleigh_S.m}

(4)Wilkinson位移的QL迭代
\lstinputlisting{wilkinson_S.m}

\begin{center}
\textbf{结果推广和一些想法}
\end{center}

其实,我们还可以对论文的结果做一些推广,同样去构造一个新的位移:

\[
\sigma =
\begin{cases}
\alpha_1 & , \beta_2^2 \geq \varepsilon\beta_1^2 \\
\alpha_1 - \mathrm{sign}(\delta) \cdot
\frac{\beta_1^2}{\left\lvert\delta\right\rvert + \sqrt{\delta^2 +
    \beta_1^2}} & , \beta_2^2 < \varepsilon\beta_1^2
\end{cases}
\]

其中,$\delta=(\alpha_2-\alpha_1)/2$,$\varepsilon\in(1,2]$.
同样可以证明,该算法也是收敛的,并且至少是三次收敛的。证明见[4]。

实际上,QL算法跟QR算法有很多类似的地方,我觉得有些方法是可以共用的。所以,可以做一下考虑:

\parencite{nla}仿照QR算法,引入隐式的QL算法。

\parencite{ych1989}在QR算法中,也通过将Rayleigh商位移和Wilkinson位移联合起来,从而构造新的位移。

\clearpage

\nocite{ych1991,xjs1992}
\printbibliography

\end{document}