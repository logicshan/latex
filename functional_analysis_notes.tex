\documentclass{ctexrep}

\usepackage{amsmath,amssymb,amsthm}
\usepackage{mathrsfs}
\usepackage{color}

\newtheorem{defn}{定义}[section]
\newtheorem{prop}[defn]{命题}
\newtheorem{thm}[defn]{定理}

\newcommand{\X}{\mathscr{X}}
\newcommand{\Xp}{(\mathscr{X},\rho)}
\newcommand{\Y}{\mathscr{Y}}
\newcommand{\Yt}{(\mathscr{Y},\tau)}
\newcommand{\SCRL}{\mathscr{L}}
\newcommand{\D}{\mathscr{D}}
\newcommand{\K}{\mathbb{K}}
\newcommand{\C}{\mathbb{C}}
\newcommand{\R}{\mathbb{R}}
\newcommand{\LXY}{\mathscr{L}(\mathscr{X},\mathscr{Y})}
\newcommand{\LXX}{\mathscr{L}(\mathscr{X},\mathscr{X})}
\newcommand{\LXK}{\mathscr{L}(\mathscr{X},\mathbb{K})}
\newcommand{\norm}[1]{\left\lVert#1\right\rVert}
\newcommand{\abs}[1]{\left\lvert#1\right\rvert}
\newcommand{\DO}{\mathscr{D}(\Omega)}
\newcommand{\CRM}{\mathrm{C}}

\DeclareMathOperator{\Ima}{Im}
\DeclareMathOperator{\im}{im}
\DeclareMathOperator{\supp}{supp}

\begin{document}
\chapter{度量空间}

\section{压缩映像原理}
\begin{defn}
设$\X$是一个非空集,若存在$\X$上一个双变量的实值函数$\rho(x,y)$,满足
下列三个条件:
\renewcommand{\labelenumi}{(\theenumi)}
\begin{enumerate}
\item 正定性: $\rho(x,y)\geq 0$,而且$\rho(x,y) = 0$当且仅当$x = y$;
\item 对称性: $\rho(x,y) = \rho(y,x)$;
\item 三角不等式: $\rho(x,z)\leq \rho(x,y) + \rho(y,z)(\forall x,y,z \in
  \X)$,
\end{enumerate}
则称$\rho$为$\X$上一个\textbf{距离},$\X$称为\textbf{距离空间}.一个以$\rho$为距离的距离空间$\X$记作$\textcolor{red}{\Xp}$.
\end{defn}

\begin{defn}
距离空间$\Xp$中的点列$\{x_n\}$称为\textbf{收敛列},是指存在$\X$中
的点$x$,当$n \rightarrow \infty$时,$\rho(x_n,x) \rightarrow 0$.此时称$x$是点列
$\{x_n\}$的\textbf{极限},记作$x_n \rightarrow x (n \rightarrow \infty)$,也记作
$\lim\limits_{n\rightarrow \infty}{x_n} = x$.
\end{defn}

\begin{defn}
距离空间$\Xp$上的点列$\{x_n\}$叫做\textbf{基本列},是指当$n,m \rightarrow
\infty$时,$\rho(x_n,x_m)\rightarrow 0$.
\end{defn}

\begin{defn}
距离空间$\Xp$叫做\textbf{完备的},是指每个基本列是收敛列.
\end{defn}

\begin{defn}
设$T : \Xp \to \Yt$是一个映射,给定$x_0 \in \X$,称$T$在点$x_0$处连续,是
指$\forall \{x_0\} \subset \X$,
\[
\lim_{n\rightarrow \infty}{\rho(x_n,x_0) = 0} \Rightarrow
\lim_{n\rightarrow \infty}{\tau(Tx_n,Tx_0) = 0}
\].
\end{defn}

\begin{prop}
给定$x_0 \in \X$,映射$T : \Xp \to \Yt$在点$x_0$处连续的充分必要条件
是:$\forall \epsilon > 0, \exists \delta = \delta(x_0,\epsilon)
> 0$,使得对于$x \in \X$,
\[
\rho(x,x_0) < \delta \Rightarrow \tau(Tx,Tx_0) < \epsilon
\].
\end{prop}

\begin{defn}
称$T : \Xp \to \Xp$是一个\textbf{压缩映射},如果存在$0 < \alpha < 1$,使
得$\rho(Tx,Ty)\leq\alpha\rho(x,y)(\forall x,y \in \X)$.
\end{defn}

\begin{thm}[\textbf{Banach不动点定理--压缩映像原理}]
设$\X$是一个完备的距离空间,$T$是$\X$到其自身的一个压缩映射,那么在$\X$中存在唯一的$T$的不动点.
\end{thm}

\section{完备化}
\begin{thm}
每个度量空间$\Xp$有一个完备化空间,且其完备化空间在等距同构的意义下唯一.
\end{thm}

\section{列紧集}
\begin{defn}
给定距离空间$\Xp$,$A$是$\X$的子集,如果$A$中的任意点列在$A$中有一个收
敛子列,则称$A$是\textbf{列紧的}.如果这个收敛子列还收敛到$A$中的点,则
称$A$是\textbf{自列紧的}.如果空间$\X$是列紧的,那么称$\X$是\textbf{列紧空间}.
\end{defn}

\begin{prop}
$\mathbb{R}^n$中有界集是列紧集,任意有界闭集是自列紧集。
\end{prop}

\begin{prop}
列紧空间内任意子集是列紧集;任意闭子空间是自列紧集.
\end{prop}

\begin{prop}
列紧空间必是完备空间.
\end{prop}

\begin{defn}
给定距离空间$\Xp$,$M \subset \X$.
\renewcommand{\labelenumi}{(\theenumi)}
\begin{enumerate}
\item 设存在$N \subset M$,$\epsilon > 0$,若对任意$x \in M$,总存在$y
  \in N$,使得$x \in B(y,\epsilon)$,那么称$N$是$M$的一个$\epsilon$\textbf{网}.如果$N$还是一个有限集,那么称$N$是$M$的一个\textbf{有限$\epsilon$网}.
\item 如果对任意$\epsilon>0$,都存在$M$的一个有限$\epsilon$网,则称集合
  $M$是\textbf{完全有界的}.
\end{enumerate}
\end{defn}

\begin{thm}[\textbf{Hausdorff定理}]
设$\Xp$是距离空间,$M \subset \X$.
\end{thm}

\renewcommand{\labelenumi}{(\theenumi)}
\begin{enumerate}
\item 若$M$在$\X$中列紧,则$M$完全有界;
\item 若$\X$是完备空间,$M$完全有界,则$M$列紧.
\end{enumerate}

\begin{defn}
一个距离空间若有可数稠密子集,就称为是\textbf{可分的}.
\end{defn}

\begin{thm}
完全有界的距离空间是可分的.
\end{thm}

\begin{defn}
紧集
\end{defn}

\begin{thm}
设$\Xp$是距离空间,为了$M \subset \X$是紧致集,必须且仅须它是自列紧集.
\end{thm}

\begin{defn}
一致有界
\end{defn}

\begin{defn}
等度连续
\end{defn}

\begin{thm}[\textbf{Arzela-Ascoli定理}]
结合$F \subset C(M)$是一个列紧集的充分必要条件是:
\renewcommand{\labelenumi}{(\theenumi)}
\begin{enumerate}
\item $F$一致有界,即存在常数$M_1$,对任何$f \in F$,都有$|f(x)|
  \leq M_1$;
\item $F$是等度连续的,即任给$\epsilon > 0$,存在$\delta =
  \delta(\epsilon) > 0$,使得对任意$f \in F$以及$x_1,x_2 \in M$,只要
  $\rho(x_1,x_2) < \delta$,就有$|f(x_1) - f(x_2)| < \epsilon$.
\end{enumerate}
\end{thm}

\begin{prop}
$M$是一个紧距离空间,则$(C(M), d)$是完备距离空间.
\end{prop}

\section{线性赋范空间}
$B^*$ - 复的线性赋范空间

完备的线性赋范空间叫做\textbf{Banach空间},简称为$B$空间.

\begin{thm}
有穷维线性空间上任意两个范数等价.
\end{thm}


\chapter{线性算子与线性泛函}

\section{线性算子与线性泛函的定义}
\begin{defn}
设$\X,\Y$是两个线性空间,$D$是$\X$的一个线性子空间.$T : D \to \Y$是一
种映射,$D$称为$T$的\textbf{定义域},有时记作$D(T)$.$R(T) = \{Tx \mid
  \forall x \in D\}$称为$T$的\textbf{值域}.如果
\[
T(\alpha x + \beta y) = \alpha Tx + \beta Ty (\forall x,y \in D,
\forall \alpha,\beta \in \K),
\]
那么称$T$是一个\textbf{线性算子}.
\end{defn}

\begin{defn}
当$\Y$是数域$\K$时,$T$称为$\K$域上的线性泛函.特别地,
取值于实数(或复数)的线性算子称为\textbf{实}(或\textbf{复})\textbf{线性
  泛函}.
\end{defn}

\begin{defn}
设$\X,\Y$是$B^*$空间,线性算子$T : D(T) \to \Y, D(T) \subset \X$. 称
$T$在$x_0 \in D(T)$\textbf{连续}是指
\[
\left.
\begin{array}{l}
x_n \in D(T)\\
x_n \rightarrow x_0
\end{array}
\right\}
\Rightarrow Tx_n \to Tx_0.
\]
\end{defn}

\begin{prop}
$T$在定义域上处处连续的充分必要条件是$T$在$x=\theta$处连续.
\end{prop}

\begin{defn}
设$\X$和$\Y$是线性赋范空间,$T$是$\X$到$\Y$的线性算子,如果有常数$M >
0$,使得
\[
\norm{Tx}_\Y \leq M \norm{x}_\X (\forall x \in \X),
\]
则称$T$是\textbf{有界线性算子}.
\end{defn}

\begin{prop}
设$\X$和$\Y$都是线性赋范空间,为了线性算子$T$连续,必须且仅须$T$是有界
的,即$T$连续$\Leftrightarrow T$有界.
\end{prop}

\begin{defn}
用$\LXY$表示一切由$\X$到$\Y$的有界线性算子的全体,并规定
\[
\norm{T} = \sup_{x\neq \theta}\frac{\norm{Tx}}{\norm{x}} =
\sup_{\norm{x} = 1}\norm{Tx}, \text{其中}T \in \LXY.
\]
\end{defn}

特别用$\SCRL(\X)$表示$\LXX$以及用$\X^*$表示$\LXK$,即$\X^*$表示$\X$上的有
界线性泛函的全体.

若在$\LXY$上规定线性运算
\[
(a_1T_1 + a_2T_2)x = a_1T_1x + a_2T_2x, \forall x \in \X,
\]
其中$a_1,a_2 \in \K, T_1,T_2 \in \LXY$,则$\LXY$是一个线性空间.

\begin{thm}
当$\X$是$B^*$空间,$\Y$是$B$空间时,$\LXY$按
$\left\lVert\cdot\right\rVert$构成一个Banach空间.
\end{thm}

\section{Riesz定理及其应用}

\begin{thm}[\textbf{Riesz表示定理}]
设$f$是Hilbert空间$\X$上的连续线性泛函,则存在唯一的元$y_f \in \X$,使
得
\[
f(x) = (x,y_f) (\forall x \in \X),
\]
而且$\norm{f} = \norm{y_f}$.于是由$f$到$y_f$给出了$\X^*$到$\X$的一个同
构.
\end{thm}

\section{纲与开映像定理}

\section{Hahn-Banach定理}

\section{共轭空间 $\cdot$ 弱收敛 $\cdot$ 自反空间}

\begin{defn}
设$\X$是$B^*$空间,$\X$上所有连续线性泛函全体,按范数
\[
\norm{f} = \sup_{\norm{x}=1}\abs{f(x)}
\]
构成一个$B$空间,称为$\X$的\textbf{共轭空间},记为$\X^*$.
\end{defn}

\begin{defn}
设$\X$是一个$B^*$空间,$\{x_n\} \subset \X, x \in \X$, 称$\{x_n\}$弱收
敛到$x$,记作$x_n \rightharpoonup x$,是指:对$\forall f \in \X^*$都有
$\lim\limits_{n\to \infty}f(x_n) = f(x)$,$x$称做点列$\{x_n\}$的弱极限.
\end{defn}

\section{线性算子的谱}
\begin{defn}
设 $A$ 是闭线性算子, $D(A)\subset \X, A : \X \to \X, \lambda \in \C$
称为 $A$ 的\textbf{本征值}是指: $\exists x_0 \in D(A) \setminus \{\theta\}$, 适合
$Ax_0 = \lambda x_0$, 并称相应的 $x_0$ 为对应于 $A$ 的\textbf{本征元}.
\end{defn}

\begin{defn}
设 $\X$ 是复 $B$ 空间, $A : D(A) \to \X$ 是闭线性算子, 称
\[
\rho(A) \overset{\text{def}}{=} \{\lambda \in \C \mid (\lambda I -
A)^{-1} \in \SCRL(\X)\}
\]
为 $A$ 的\textbf{预解集}. $\lambda \in \rho(A)$ 称为 $A$ 的\textbf{正则
  值}.
\end{defn}

\begin{defn}
称集合 $\sigma(A) \overset{\text{def}}{=} \C \setminus \rho(A)$ 为 $A$
的\textbf{谱集}. $\lambda \in \sigma(A)$ 称为 $A$ 的\textbf{谱点}.
\end{defn}

注: 闭线性算子-定义2.3.9.

\chapter{广义函数与Sobolev空间}

\section{广义函数的概念}

广义函数是定义在一类"性质很好"的函数空间上的连续线性泛函.

设 $\Omega \subset \R^n$ 是一个开集, $u \in \CRM(\overline{\Omega})$,
称集合
\[
F = \{x \in \Omega \mid u(x) \neq 0\}
\]
的闭包(关于$\Omega$)为 $u$ 的关于 $\Omega$ 的\textbf{支集}, 记作
$\supp{u}$. 换句话说, 连续函数 $u$ 的支集是在此集外 $u$ 恒为 $0$ 的相对于
$\Omega$ 的最小闭集.

对于整数 $k \le 0$ (可以是 $\infty$), $\CRM^k_0(\Omega)$ 表记支集
在 $\Omega$ 内紧的全体 $\CRM^k(\overline{\Omega})$ 函数所组成的集合,
于是
\[
\CRM^\infty_0(\Omega) \subset \cdots \subset \CRM^{k+1}_0(\Omega)
\subset \CRM^k_0(\Omega) \subset \cdots \subset \CRM^0_0(\Omega).
\]

\begin{defn}
在集合 $\CRM^\infty_0(\Omega)$ 上定义收敛性如下: 我们说序列
$\{\varphi_j\}$ 收敛于 $\varphi_0$, 如果
\renewcommand{\labelenumi}{(\theenumi)}
\begin{enumerate}
\item 存在一个相对于 $\Omega$ 的紧集 $\mathrm{K} \subset \Omega$, 使得
  $\supp{(\varphi_j)} \subset \mathrm{K} (j = 0, 1, 3, \cdots);$
\item 对于任意指标 $\alpha = (\alpha_1, \cdots, \alpha_n)$ 都有
\[
\max_{x\in \mathrm{K}}{\abs{\partial^\alpha\varphi_j(x)
    - \partial^\alpha\varphi_0(x)}} \rightarrow 0 (j \rightarrow \infty).
\]
\end{enumerate}
带上述收敛性的线性空间 $\CRM^\infty_0(\Omega)$, 称为\textbf{基本空间} $\D(\Omega)$.
\end{defn}

\begin{defn}
$\DO$ 上的一切线性连续泛函都称为\textbf{广义函数}, 即广义函数是这样的
泛函 $f : \DO \to \R^1$, 满足
\renewcommand{\labelenumi}{(\theenumi)}
\begin{enumerate}
\item 线性:
\[
\langle f, \lambda_1\varphi_1 + \lambda_2\varphi_2\rangle =
\lambda_1\langle f,\varphi_1\rangle + \lambda_2\langle
f,\varphi_2\rangle
\]
\[
(\forall \varphi_1, \varphi_2 \in \DO, \forall \lambda_1, \lambda_2
\in \R^1);
\]
\item 对于任意的 $\{\varphi_j\} \subset \DO$, 只要 $\varphi_j \to
  \varphi_0 (\DO)$, 都有
\[
\langle f,\varphi_j\rangle \to \langle f,\varphi_0\rangle (j \to \infty).
\]
\end{enumerate}
一切广义函数所组成的集合记作 $\D'(\Omega)$.
\end{defn}

\section{$B_0$空间}

\section{广义函数的运算}

$\D(\Omega)$ 中的运算(指若干 $\D(\Omega)$ 到自身的连续线性算子的运算).

线性算子 $A : \D(\Omega) \to \D(\Omega)$ 称为连续的, 是指:
\[
\varphi_j \rightarrow \varphi (\DO) \Rightarrow A\varphi_j \rightarrow
A\varphi (\DO).
\]
$A$ 是连续线性算子 $\Leftrightarrow \varphi_j \rightarrow 0 (\DO)
\Rightarrow A\varphi_j \rightarrow 0 (\DO)$, 记作 $A \in \SCRL(\DO).$

$\D'(\Omega)$ 中的运算(指若干 $\D'(\Omega)$ 到自身的连续线性算子的运算).

在 $\D'(\Omega)$ 上定义算子 $A^*$ 如下:

设 $A : \DO \to \DO$ 为线性算子, $\forall f \in \D'(\Omega), A^*f \in
\D'(\Omega)$ 由下式确定:
\[
\langle A^*f, \varphi \rangle = \langle f, A\varphi\rangle (\forall \varphi \in \DO),
\]
则 $A^* : \D'(\Omega) \to \D'(\Omega)$ 并且是连续的.

\end{document}