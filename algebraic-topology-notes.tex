\documentclass[UTF8]{ctexart}

\usepackage{amsmath,amssymb,amsthm}
\usepackage{mathrsfs}
\usepackage{color}
\usepackage{tikz-cd}
\usepackage{relsize}

\usepackage{filecontents}

\usepackage[hyperref=true,backend=biber,sorting=none,backref=true]{biblatex}
\addbibresource{algebraic-topology-notes.bib}
\usepackage{hyperref}

\newtheorem*{prop}{命题}
\newtheorem*{thm}{定理}
\newtheorem*{defn}{定义}

\DeclareMathOperator{\rel}{rel}
\DeclareMathOperator{\pullback}{\mathsmaller{\prod}}
\DeclareMathOperator{\pushout}{\mathsmaller{\coprod}}
\DeclareMathOperator{\im}{im}
\DeclareMathOperator{\homology}{H_\ast}

\newcommand{\Top}{\mathbf{Top}}
\newcommand{\Ab}{\mathbf{Ab}}

\begin{document}
\begin{defn}
设 $X,Y$ 是两个拓扑空间. 记
\[
C(X,Y) = \{f : X \to Y \mid f \text{是连续映射}\}.
\]
任取 $f,g \in C(X,Y)$, 若有连续映射 $H : X \times [0,1] \to Y$ 满足
\[
H(x,0) \equiv f(x),\hspace{1em} H(x,1) \equiv g(x),
\]
则称 $f$ 和 $g$ \textbf{同伦}(homotopic), 记为 $f \overset{H}{\simeq}
g$; 并称 $H$ 为从 $f$ 到 $g$ 的 \textbf{伦移}(homotopy), 称每个映射
$h_t : X \to Y, x \mapsto H(x,t)$ 为 $H$ 在 $t$ 时刻的 \textbf{切
  片}(section), 称每条道路 $\eta_x : [0,1] \to Y, t \mapsto H(x,t)$ 为
$H$ 在 $x$ 点处的 \textbf{踪}(trace).
\end{defn}
\begin{defn}
按同伦关系将 $C(X,Y)$ 中的映射划分等价类, 每一个等价类称为一个
\textbf{映射类}(mapping class). $f$ 所在的映射类记为 $\langle f
\rangle$. 所有从 $X$ 到 $Y$ 的映射类构成的集合记为 $[X,Y]$.
\end{defn}

$[X,Y]$ 总是非空的, 因为它至少要含有一个常值映射所在的映射类. 如果 $f$
同伦于常值映射, 则称 $f$ \textbf{零伦}(null homotopic).

\begin{defn}
设 $A \subset X, f,g \in C(X,Y)$. 如果存在伦移 $H : f \simeq g$ 满足任
取 $a \in A, H(a,t) \equiv f(a) = g(a)$, 则称 $f$ 和 $g$ \textbf{相对
  于} $A$ \textbf{同伦}(homotopic relative to A), 记为 $H : f \simeq g
\rel A$, 并称 $H$ 为一个 \textbf{相对于} $A$ \textbf{的伦移}(homotopy
relative to A).
\end{defn}

\begin{defn}
设 $X$ 与 $Y$ 为拓扑空间, 如果存在连续映射 $f : X \to Y$ 和 $g : Y \to
X$, 使得
\[
g \circ f \simeq id_X, \hspace{1em} f \circ g \simeq id_Y,
\]
则称 $X$ 和 $Y$ \textbf{同伦等价}(homotopy equivalent), 记为 $X \simeq
Y$. 我们也称映射 $f$ 和 $g$ 为 \textbf{同伦等价}(homotopy
equivalence), 并称它们互为 \textbf{同伦逆}(homotopy inverse).
\end{defn}
\textcolor{red}{注意, 一个同伦等价的同伦逆并不唯一.}

\begin{defn}
设 $A \subset X, i : A \hookrightarrow X$ 是包含映射. 如果映射 $r : X
\to A$ 满足 $r \circ i = r|_A = id_A$, 则称之为 \textbf{收缩映
  射}(retraction), 称 $A$ 为 $X$ 的 \textbf{收缩核}(retract). 如果一个
收缩映射 $r$ 还满足 $i \circ r \simeq id_X$, 则称之为 \textbf{形变收
  缩}(deformation retraction), 称 $A$ 为 $X$ 的 \textbf{形变收缩
  核}(deformation retract). 如果形变收缩 $r$ 还满足 $i \circ r \simeq
id_X \rel A$, 则称之为 \textbf{强形变收缩}(strong deformation
retraction), 称 $A$ 为 $X$ 的\textbf{强形变收缩核}(strong deformation retract).
\end{defn}

\begin{defn}
如果一个空间可以形变收缩到一点, 则称它 \textbf{可缩}(contractible).
\end{defn}

显然可缩空间一定是道路连通的, 因为这个空间中的每个点都可以沿一条道路移
动到作为收缩核的那个点上去. \textcolor{red}{注意, 在收缩过程中那个点完全可以先暂时跑到
别的地方再跑回来. 换言之, 这个概念不要求它是强形变收缩核.}

\begin{defn}
设 $a,b : [0,1] \to X$ 都是从 $x$ 到 $y$ 的道路, $H : [0,1] \times
[0,1] \to X$ 是从 $a$ 到 $b$ 的伦移, 并且每个时刻 $t$ 的切片 $h_t :
[0,1] \to X, s \mapsto H(s,t)$ 也都是从 $x$ 到 $y$ 的道路, 即
\[
h_0 = a, \hspace{1ex}h_1 = b,\hspace{1ex} h_t(0) \equiv x,\hspace{1ex} h_t(1) \equiv y,
\]
则称 $a$ 和 $b$ \textbf{定端同伦}(path homotopic, 也称 \textbf{道路同
  伦}), 记为 $a \simeq b$, 称 $H$ 为 \textbf{定端伦移}(path homotopy,
也称 \textbf{道路伦移}), 并称所有与 $a$ 定端同伦的道路构成的集合为 $a$
的 \textbf{道路类}(class of paths), 记为 $\langle a \rangle$.
\end{defn}

\begin{defn}
称从 $x_0$ 到 $x_0$ 的道路为以 $x_0$ 为 \textbf{基点}(base point) 的
\textbf{闭道路}(loop). 闭道路所在的道路类称为 \textbf{闭道路类}(class
of loops). $X$ 上所有以 $x_0$ 为基点的闭道路类构成的集合记为 $\pi_1(X,x_0)$.
\end{defn}

\begin{defn}
设 $a$ 是从 $x$ 到 $y$ 的道路, $b$ 是从 $y$ 到 $z$ 的道路, 则称以两倍
速度依次走完 $a$ 和 $b$ 的道路
\[
c(t) =
\begin{cases}
a(2t),     & \text{若} 0 \leq t \leq \frac{1}{2};\\
b(2t - 1), & \text{若} \frac{1}{2} \leq t \leq 1
\end{cases}
\]
为 $a$ 和 $b$ 的 \textbf{乘积道路}(product path), 记为 $ab$.
\end{defn}

\begin{defn}
基本群平凡的道路连通空间称为 \textbf{单连通}(simply connected)空间.
\end{defn}

\begin{thm}
连续映射 $f : X \to Y$ 诱导基本群的同态
\[
f_\pi : \pi_1(X,x_0) \to \pi_1(Y,f(x_0)), \hspace{1em}\langle a
\rangle \mapsto \langle f \circ a \rangle,
\]
称为 $f$ 的 \textbf{诱导同态}(induced homomorphism).
\end{thm}

\begin{defn}
设 $p : E \to B$ 连续. 如果连续映射 $f : X \to B$ 和 $f^\uparrow : X
\to E$ 满足如下交换图表, 即 $f = p \circ f^\uparrow$, 则称
$f^\uparrow$ 为 $f$ 关于 $p$ 的 \textbf{提升}(lift 或 lifting).
\begin{center}
\begin{tikzcd}
& E \arrow[d, "p"] \\
X \arrow[r, "f"] \arrow[ur, dashrightarrow, "f^\uparrow"] & B
\end{tikzcd}
\end{center}
\end{defn}

\begin{defn}
设 $p : E \to B$ 连续. 称一个空间 $X$ 关于 $p$ 满足 \textbf{同伦提升性
  质}(homotopy lifting property), 如果它满足下述条件: 任取连续映射 $f
: X \to B$, 只要 $f$ 有提升 $f^\uparrow$, 则从 $f$ 开始的任意伦移 $F$
一定存在从 $f^\uparrow$ 开始的伦移作为其提升.
\end{defn}
\begin{center}
\begin{tikzcd}
&& E \arrow[d, "p"] \\
X \arrow[rru, "f^\uparrow"] \arrow[r, "\mathrm{id} \times 0"'] & X \times [0,1]
\arrow[ru, dashrightarrow, "F^\uparrow"'] \arrow[r, "F"'] & B
\end{tikzcd}
\end{center}

\begin{defn}
设 $p : E \to B$ 连续. 如果任何空间 $X$ 关于 $p$ 均满足同伦提升性质,
则称 $p$ 为一个 \textbf{纤维化}(fibration). 称 $E$ 为 \textbf{全空
  间}(total space), 称 $B$ 为 \textbf{底空间}(base space), 称每个
$p^{-1}(b)$ 为 $b$ 上的一根 \textbf{纤维}(fiber).
\end{defn}

\begin{defn}
设 $p : E \to B$ 是一个连续满射, 并且对于每个 $b \in B$ 存在其开邻域
$U_b$, 使得 $p^{-1}(U_b)$ 是一族互不相交的开集
$\{V_{b,\lambda}\}_{\lambda \in \Lambda}$ 的并, 并且每个
$p|_{V_{b,\lambda}} : V_{b,\lambda} \to U_b$ 都是同胚, 则称 $(E,p)$ 为
$B$ 上的 \textbf{复迭空间}(covering space), 并称 $p$ 为一个 \textbf{复
  迭映射}(covering map). 称 $E$ 为其 \textbf{全空间}(total space), 称
$B$ 为其 \textbf{底空间}(base space), 称这些 $U_b$ 为 \textbf{均匀复迭
  邻域}(evenly-covered neighborhood), 称每个 $p^{-1}(b)$ 为 $b$ 上的
\textbf{纤维}(fiber).
\end{defn}

以上内容引用自\parencite{包志强2013点集拓扑与代数拓扑引论}

%\noindent\hrulefill

\noindent\makebox[\linewidth]{\rule{\paperwidth}{0.4pt}}

\begin{enumerate}
\item The different types of homology theory arise from functors mapping from various categories of mathematical objects to the category of chain complexes. In each case the composition of the functor from objects to chain complexes and the functor from chain complexes to homology groups defines the overall homology functor for the theory.\parencite{types_of_homology}

比如, 单纯同调(simplicial homology)定义在单纯复形(simplicial complex)上. 奇异同调(singular homology)定义在所有拓扑空间上. 胞腔同调(cellular homology)定义在 CW复形(CW-complex)上.

For $X$ a CW-complex, its cellular homology $H^{CW}_\bullet(X)$ agrees with its singular homology $H_\bullet(X)$:
\[
H^{CW}_\bullet(X)\simeq H_\bullet(X)
\]
可以把一个同调理论(homology theory)看成一族函子.

上面定理也可以叙述为: The cellular and singular homology of a CW-complex are naturally isomorphic.

The homology $H_\ast(K;R)$ of an abstract simplicial complex $K$ is isomorphic to $H_\ast(\left|K\right|;R)$, the singular homology of its geometric realization. This can be seen by noting that $\left|K\right|$ is naturally a CW-complex. The cellular chain complex of $\left|K\right|$ is isomorphic to the simplicial chain complex of $K$.\parencite[p.8]{kirklecture}

\item Pullback and Pushout in $\Top$\parencite{algebraic_topology_notes_of_the_lecture}

Consider the diagram
\begin{center}
\begin{tikzcd}[row sep=large]
  & Y \arrow[d, "g"] \\
X \arrow[r, "f"'] & Z
\end{tikzcd}
\end{center}
in $\Top$. Then the pullback of $f$ and $g$ is $X \pullback_Z Y \in \Top$ given by
\[
X \pullback_{Z} Y := \{(x,y) \in X \times Y \mid f(x) = g(y)\} \subset X \times Y
\]
(with subspace topology).

Consider the diagram
\begin{center}
\begin{tikzcd}[row sep=large]
Z \arrow[d,"f"'] \arrow[r,"g"] & Y \\
X
\end{tikzcd}
\end{center}
in $\Top$. Then the pushout $X \pushout_Z Y\in \Top$ of $f$ and $g$ is given by $X\pushout Y/\sim$ where $i_X f(z) \sim i_Y g(z)$ for all $z \in Z$.

\item Attaching Maps\parencite{attaching_maps}

Given $A\subset X$ with $f : A\to Y$, we define "the space obtained from $Y$ by attaching $X$ by means of $f$" (wrriten $X \cup_f Y$) as
\[
X\cup_f Y = (X\pushout Y)/\sim
\]
where $a\sim f(a)\forall a\in A$.
\begin{center}
\begin{tikzcd}[row sep=huge]
A \arrow[d, hook'] \arrow[r, "f"] & Y \arrow[d, hook', "i_Y"]\\
X \arrow[r, "j_X"] & X\cup_f Y
\end{tikzcd}
\end{center}
is a pushout in the category of topological spaces.

$i_Y$ is always an injection.

$j_X$ is an injection iff $f$ is.

\item CW complex

$S^n$ : the unit sphere in $\mathbb{R}^{n+1}$, all points of distance $1$ from the origin.

$D^n$ : the unit disk or ball in $\mathbb{R}^n$, all points of distance $\leq 1$ from the origin.

$\partial D^n = S^{n-1}$ : the boundary of the $n$-disk.

$e^n$ : an $n$-cell, homeomorphic to the open $n$-disk $D^n - {\partial D^n}$.

In particular, $D^0$ and $e^0$ consist of a single point since $\mathbb{R}^0 = \{0\}$. But $S^0$ consists of two points since it is $\partial D^1$.

\textcolor{red}{C} - closure finite

\textcolor{blue}{W} - weak topology

\[
X_0 = \text{a set of discrete pts}
\]

\begin{center}
\begin{tikzcd}[row sep=huge]
\amalg{\partial D^{n+1}} \arrow[d,hook'] \arrow[r] & X_n \arrow[d,hook']\\
\amalg{D^{n+1}} \arrow[r] & X_{n+1}
\end{tikzcd}
\end{center}

\item \textbf{standard n-simplex}

The standard n-simplex (or unit n-simplex) is the subset of $\mathbb{R}^{n+1}$ given by
\[
\Delta^n = \left\{(t_0,\ldots,t_n)\in \mathbb{R}^{n+1}\mid \sum_{i=0}^{n} t_i = 1 \text{ and } t_i \geq 0 \text{ for all } i\right\}
\]

\item Singular homology\cite[p.1]{kirklecture}\cite{singular_homology}

Construction of the singular chain complex.

The \textit{(geometric) q-simplex} $\Delta^q$ is defined by
\[
\Delta^q = \left\{(t_0,t_1,\ldots,t_q)\in \mathbb{R}^{q+1}\mid \mathsmaller{\sum} t_i = 1, t_i \geq 0\text{ for all } i\right\}
\]
The \textit{face maps} are the functions
\[
f_m^q : \Delta^{q-1} \to \Delta^q
\]
defined by
\[
(t_0,t_1,\ldots,t_{q-1}) \mapsto (t_0,\ldots,t_{m-1},0,t_m,\ldots,t_{q-1})
\]
A \textit{singular $q$-simplex in a space} $X$ is a continuous map $\sigma : \Delta^{q} \to X$.

Let $R$ be a \textit{commutative ring} with unit. Denote by $S_q(X;R)$ the free $R$-module with basis the singular $q$-simplices $\left\{\sigma : \Delta^q \to X\right\}$ and define the differential $\partial : S_q(X;R)\to S_{q-1}(X;R)$ to be the $R$-linear map defined on a singular simplex $\sigma$ by
\[
\partial(\sigma) = \sum_{m=0}^{q} (-1)^m\sigma\circ f_m^q.
\]
\[
\Delta^{q-1} \overset{f_m^q}{\to} \Delta^q \overset{\sigma}{\to} X
\]
Thus, on a chain $\sum_{i=1}^{\ell}r_i\sigma_i$, $\partial$ has the formula
\[
\partial\left(\sum_{i=1}^{\ell}r_i\sigma_i\right) = \sum_{i=1}^{\ell}r_i\left(\sum_{m=0}^{q}\left(-1\right)^m \sigma_i \circ f_m^q \right).
\]
One calculates that $\partial^2 = 0$.

例子: 单点的 singular homology. $X = \{\ast\}, R=\mathbb{Z}$.

$q\geq0$ 时, $\Delta^q \to \{\ast\}$ 都只有一个映射 $\sigma_q$, 此时 $S_q(X) = \mathbb{Z}\sigma_q \cong \mathbb{Z}$.
$\partial\sigma_q = \sum\limits_{m=0}^{q}(-1)^m\sigma_q\circ f_m^q$, 由于$\sigma_q\circ f_m^q$ 只可能是 $\sigma_{q-1}$, 所以
\[
\partial\sigma_q =
\begin{cases}
0, & q\text{为奇数}\\
\sigma_{q-1}, & q\text{为偶数}
\end{cases}
\]
\[
\cdots\to S_{q+1}(X)\to S_q(X) \to\cdots \to S_1(X)\to S_0(X)\to 0
\]
当 $q=0$ 时, $Z_0 = \ker\partial_0 \cong \mathbb{Z}$, $B_0 = \im\partial_1 = 0$, 所以$H_0(X) = Z_0/B_0 \cong \mathbb{Z}$.

当 $q > 0$ 且为奇数时, $Z_q = \ker\partial_q \cong \mathbb{Z}$, $B_q = \im\partial_{q+1} \cong \mathbb{Z}$, 所以 $H_q(X) = Z_q/B_q \cong 0$.

当 $q > 0$ 且为偶数时, $Z_q = \ker\partial_q \cong 0$, $B_q = \im\partial_{q+1} \cong 0$, 所以 $H_q(X) = Z_q/B_q \cong 0$.

\[
H_n(\{\ast\}) =
\begin{cases}
\mathbb{Z}, & n = 0\\
0, & n \neq 0
\end{cases}
\]

\item $\Top \overset{\homology}{\longrightarrow} \Ab^{\mathbb{N}}$

$\mathbb{N}$ = discrete category of natural numbers

$\Ab^{\mathbb{N}} \triangleq \mathbb{N}\!\!-\!\!\text{graded abelian group } = \{\mathbb{N} \to \Ab \text{ functors}\}$

\item relative homology in singular homology\cite{relative_homology_wiki}\cite{relative_homology_planet}

If $X$ is a topological space, and $A$ a subspace, then the inclusion map $A\hookrightarrow X$ makes $C_n(A)$ into a subgroup of $C_n(X)$(because $C_{\ast}(-)$ is a covariant functor). Since the bounded map on $C_{\ast}(X)$ restricts to the boundary map on $C_{\ast}(A)$(make sure that the bounded maps as follows are well defined), we can take the quotient complex $C_{\ast}(X,A)$,
\[
\overset{\partial}{\leftarrow} C_n(X)/C_n(A) \overset{\partial}{\leftarrow} C_{n+1}(X)/C_{n+1}(A) \overset{\partial}{\leftarrow}
\]
The homology groups of this complex $H_n⁢(X,A) \triangleq H_n(C_{\bullet}(X)/C_{\bullet}(A))$, are called the \textit{relative homology groups} of the pair $(X,A)$.

Given a subspace $A\subset X$, one may form the short exact sequence
\[
0\to C_{\bullet}(A)\to C_{\bullet}(X)\to C_{\bullet}(X)/C_{\bullet}(A) \to 0
\]

From a short exact sequence of singular complexes, we can get a long exact sequence of homology groups:
\[
\cdots \to H_n(A) \to H_n(X) \to H_n(X,A) \overset{\delta}{\to} H_{n-1}(A) \to \cdots
\]
It follows that $H_n(X, x_0)$, where $x_0$ is a point in $X$, is the $n$-th reduced homology group of $X$.(We can take this property as the definition of reduced homology.) In other words, $H_i(X, x_0) = H_i(X)$ for all $i > 0$. When $i = 0, H_0(X, x_0)$ is the free module of one rank less than $H_0(X)$. The connected component containing $x_0$ becomes trivial in relative homology.

The excision theorem says that removing a sufficiently nice subset $Z\subset A$ leaves the relative homology groups $H_n(X, A)$ unchanged. Using the long exact sequence of pairs and the excision theorem, one can show that $H_n(X, A)$ is the same as the $n$-th \textcolor{red}{reduced} homology groups of the quotient space $X/A$.\cite[p.50]{vick2012homology}

Functoriality

The map $C_{\bullet}$ can be considered to be a functor
\[
C_{\bullet} : \Top^2 \to \mathcal{CC}
\]
where $\Top^2$ is the category of pairs of topological spaces and $\mathcal{CC}$ is the category chain complexes of abelian groups.

\iffalse
Examples

One important use of relative homology is the computation of the homology groups of quotient spaces $X/A$. In the case that $A$ is a subspace of $X$ fulfilling the mild regularity condition that there exists a neighborhood of $A$ that has $A$ as a deformation retract, then the group $\tilde{H}_n(X/A)$(reduced homology) is isomorphic to $H_n(X,A)$. We can immediately use this fact to compute the homology of a sphere. We can realize $S^n$ as the quotient of an $n$-disk by its boundary, i.e. $S^n = D^n/S^{n-1}$. Applying the exact sequence of relative homology gives the following:
\[
0 \to C_{\bullet}(S^{n-1}) \to C_{\bullet}(D^n) \to C_{\bullet}(D^n)/C_{\bullet}(S^{n-1}) \to 0
\]
\[
\cdots\to H_i(D^n)\to H_i(D^n,S^{n-1})\to H_{i-1}(S^{n-1})\to H_{i-1}(D^n)\to\cdots
\]
Because the disk is contractible, we know its reduced homology groups vanish in all dimensions (Homology groups of all contractible topological spaces coincide with those of the point.), so the above sequence collapses to the short exact sequence:
\[
0\to H_i(D^n,S^{n-1})\to H_{i-1}(S^{n-1})\to 0
\]
Therefore, we get isomorphisms $H_i(D^n,S^{n-1}) \cong H_{i-1}(S^{n-1})$.

So we have
\[
\tilde{H}_i(S^n) = \tilde{H}_i(D^n/S^{n-1})\cong H_i(D^n,S^{n-1})\cong H_{i-1}(S^{n-1})
\]
\fi

\item excision theorem

$X \supset A (\supset A^{\circ} \supset \overline{B}) \supset B$

Then $H_n(X,A) \cong H_n(X-B,A-B)$.

vick 的书上\cite[p.47]{vick2012homology}, 证明用到了定理1.14\cite[p.20]{vick2012homology}.

\item Mayer-vietoris sequence

$U\subset X, V \subset X, X = U^{\circ} \cup V^{\circ}$, we have exact sequence
\[
\cdots\to H_n(U\cap V)\to H_n(U)\oplus H_n(V)\to H_n(X)\to H_{n-1}(U\cap V)\to\cdots
\]
\end{enumerate}

\begin{filecontents*}{algebraic-topology-notes.bib}
@book{包志强2013点集拓扑与代数拓扑引论,
  title={点集拓扑与代数拓扑引论},
  author={包志强},
  year={2013},
  publisher={北京大学出版社}
}

@book{kirklecture,
  title={Lecture Notes in Algebraic Topology},
  author={Kirk, J.F.D.P.},
  isbn={9780821872208},
  url={https://books.google.co.jp/books?id=TaE4F-bsRYQC},
  publisher={American Mathematical Soc.}
}

@online{singular_homology,
author = "Wikipedia",
title = "Singular homology",
url = "https://en.wikipedia.org/wiki/Singular_homology"
}

@book{vick2012homology,
  title={Homology Theory: An Introduction to Algebraic Topology},
  author={Vick, J.W.},
  isbn={9781461208815},
  lccn={93005255},
  series={Graduate Texts in Mathematics},
  url={https://books.google.co.jp/books?id=U2rgBwAAQBAJ},
  year={2012},
  publisher={Springer New York}
}

@online{relative_homology_wiki,
author = "Wikipedia",
title = "Relative homology",
url = "https://en.wikipedia.org/wiki/Relative_homology"
}

@online{relative_homology_planet,
author = "Wikipedia",
title = "relative homology groups",
url = "http://planetmath.org/relativehomologygroups"
}

@online{types_of_homology,
author = "Wikipedia",
title = "Types of homology",
url = "https://en.wikipedia.org/wiki/Homology_(mathematics)#Types_of_homology"
}

@online{algebraic_topology_notes_of_the_lecture,
author = "G. Mislin. Thomas Rast. Luca Gugelmann",
title = "Algebraic Topology Notes of the Lecture",
url = "https://www.mitschriften.ethz.ch/main.php?page=3&scrid=1&pid=46&oid=69&eid=1"
}

@online{attaching_maps,
author = "MAT1301S",
title = "Attaching Maps",
url = "http://www.math.toronto.edu/mat1300/CW-complexes.pdf"
}
\end{filecontents*}

\newpage
\printbibliography
\end{document} 