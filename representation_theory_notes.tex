\documentclass{ctexart}

\usepackage{amsmath,amssymb,amsthm}
\usepackage{mathrsfs}
\usepackage{color}
\usepackage{tikz-cd}
\usepackage{url}

\usepackage{biblatex}
\addbibresource{representation_theory_notes.bib}

\newtheorem*{prop}{命题}
\newtheorem*{thm}{定理}

\newcommand{\abs}[1]{\left\lvert#1\right\rvert}

\DeclareMathOperator{\Ima}{Im}
\DeclareMathOperator{\im}{im}

\begin{document}
\begin{enumerate}
\item Artin-Wedderburn 定理

Artin-Wedderburn 定理是关于半单环和半单代数的分类定理. 它是说, 如果 $A$ 是一个半单代数, 那么 $A$ 同构于除环上矩阵代数的直和. 这是说, $A$ 作为一个左正则模是半单的, 令 $A$ 作为左正则模有如下分解式:
\[
_AA \cong S_1^{n_1} \oplus \cdots \oplus S_r^{n_r}
\]
那么 $A$ 作为代数, 可分解为以下矩阵代数的直和:
\[
A \cong M_{n_1}(D_1) \oplus \cdots \oplus M_{n_r}(D_r)
\]
其中 $D_i = End_A(S_i)^{op}$.

\item Proposition 4.3.1

$RG$ 作为右 $RH$-模, 可分解为如下右 $RH$-模的直和:
\[
RG_{RH} = \bigoplus^{\abs{G:H}}_{i=1} g_iRH
\]
其中 $g_iRH$ 为 右 $RH$-模.

$V\uparrow^G_H = RG \otimes_{RH}V = (\bigoplus^{\abs{G:H}}_{i=1} g_iRH) \otimes_{RH} V = \bigoplus^{\abs{G:H}}_{i=1}(g_iRH \otimes_{RH} V) = \bigoplus^{\abs{G:H}}_{i=1} g_i\otimes_{RH}V$

注意, $g_i\otimes_{RH}$ 不是 $RG$-模, 仅是 $R$-模, 因为 $\forall x \in G, x\cdot (g_i\otimes v) = (xg_i)\otimes v$, 而 $xg_i$ 可能不是 $g_i$, 从而 $(xg_i)\otimes v$ 可能不属于 $g_i \otimes_{RH} V$. 而 $g_i\otimes_{RH}V$ 是 $R$-模. $\forall r \in R, r \cdot (g_i\otimes v) = re\cdot(g_i\otimes v) = (r(e\cdot g_i))\otimes v = (rg_i)\otimes v = (g_ir)\otimes v = g_i\otimes (rv) \in g_i \otimes_{RH} V$. 其中 $e$ 是 $G$ 的单位元, 并且 $R$ 是交换的.

\item 高阶 socle 和 radical

$Rad^n(U) = Rad(Rad^{n-1}(U))$

$Soc^n(U)/Soc^{n-1}(U) = Soc(U/Soc^{n-1}(U))$.

以下是对高阶 socle 的解释:

首先 $Soc^{n-1}(U)$ 是 $U$ 的一个子模, 可以得到一个商模 $U/Soc^{n-1}(U)$. 有如下自然同态:
\[
\pi : U \to U/Soc^{n-1}(U)
\]
可以取这个商模的 socle, 得到 $Soc(U/Soc^{n-1}(U))$, 它是这个商模的一个子模. 由定理1.4.7\cite{dlab2012finite}, $U/Soc^{n-1}(U)$ 中的每一个子模, 在 $U$ 中都有一个包含 $Soc^{n-1}(U)$ 的子模与之对应, 这个子模即为 $Soc^n(U)$. \textcolor{red}{(注意记号方面的问题, 由于商模的一个子模还是一个商模, 所以 $Soc(U/Soc^{n-1}(U))$ 是一个商模, 可以有形式 $Soc^n(U)/Soc^{n-1}(U)$.)}

\item 表示的共轭(conjugation of representation)

令 $H$ 为 $G$ 的一个子群, $g \in G$, $V$ 是 $H$ 的一个表示. 我们定义 $^gH$ 的一个表示 $^gV$, 令 $^gV = V$, 并且 $^gh \cdot v = hv, \forall ^gh \in {}^gH$.

\item Double cosets

Let $G$ be a group with subgroups $H$ and $K$.

1. $G/H$ denotes the set of left cosets $\{gH : g \in G\}$ of $H$ in $G$.

2. $H\backslash G$ denotes the set of right cosets $\{Hg : g \in G\}$ of $H$ in $G$.

3. $K\backslash G/H$ denotes the set of double cosets $\{KgH : g \in G\}$ of $H$ and $K$ in $G$.

注意与以下符号的区别

If $\Omega$ is a left $G$-set we use the notation $G\backslash \Omega$ for the set of orbits of $G$ on $\Omega$, and denote a set of representatives for the orbits by $[G\backslash \Omega]$. Similarly if $\Omega$ is a right $G$-set we write $\Omega /G$ and $[\Omega/G]$.

\item 7.2 Projectives by means of idempotents

One way to obtain projective $A$-modules is from idempotents of the ring $A$. If $e^2 = e \in A$ then ${}_AA = Ae \oplus A(1 - e)$ as $A$-modules, and so the submodules $Ae$ and $A(1 - e)$ are projective. We formalize this with the next result, which should be \textcolor{red}{compared with Proposition 3.6.1} in which we were dealing with ring summands of $A$ and central idempotents.

In Proposition 3.6.1 it was proved that in a decomposition of $A$ as a direct sum of indecomposable rings, the rings are uniquely determined as subsets of $A$ and the corresponding primitive central idempotents are also unique. We point out that the corresponding uniqueness property need not hold with module decompositions of ${}_AA$ that are not ring decompositions. 就是说, $A$ 本身作为环, 若分解为\textcolor{red}{不可分解环的直和}, 则是唯一的, 若作为正则模, 分解为\textcolor{red}{不可分解模的直和}, 则不一定唯一.

We will see later that if $A$ is a finite dimensional algebra over a field then in any two decompositions of ${}_AA$ as a direct sum of indecomposable submodules, the submodules are isomorphic in pairs.

We will also see that when $A$ is a finite dimensional algebra over a field, every indecomposable projective $A$-module may be realized as $Ae$ for some primitive idempotent $e$. For other rings this need not be true.

\item Changing the Ground Ring
Given $f : R \to S$ a ring homomorphism, $f$ induces a functor

\begin{align*}
\{\text{(left) R-mods.}\} &\to \{\text{(left) S-mods.}\}\\
V &\mapsto V_S := S \otimes_R V.
\end{align*}

The map $f$ makes $S$ a two-sided $R$-module (and in particular, a right module), so $S \otimes_R V$ makes sense. $S \otimes_R V$ is an $S$-module via the action
\[
s'(s \otimes v) = (s's)\otimes v.
\]
If
\[
0 \to U \to V \to W \to 0
\]
is a short exact sequence of left $R$-modules and $M$ is a right $R$-module then
\[
M\otimes_R U \to M\otimes_R V \to M\otimes_R W \to 0
\]
is exact, although the first map may not be injective. However, if $M$ is a free $R$-module, $M \cong R^n$ then $M\otimes_R N \cong N^n$, and so
\[
M\otimes_R U \hookrightarrow M\otimes_R V
\]
in this case.
\qquad In particular, if $f : R \hookrightarrow S$ makes $S$ into a free $R$-module then when
\[
0 \to U \to V \to W \to 0
\]
is exact, so is
\[
0\to U_S\to V_S\to W_S\to  0,
\]
ie. $(V/U)_S \cong V_S/U_S$.

\qquad In particular, if $K \subset M$ is a field extension then $M$ is a free $K$-module.

\qquad $f : K \to M$ induces $K[G]\to M[G]$, and thus
\begin{align*}
K[G]\!\!-\!\!\mathrm{mods} &\to M[G]\!\!-\!\!\mathrm{mods}\\
V &\mapsto V_M
\end{align*}

\item 2.4. Behavior under Field Extensions\cite{weintraub2003representation}

ring $R$ is an $F$-algebra, for some field $F$. Let $E$ be an extension field of $F$, and set $R' = E\otimes_F R$.


\item 华丽的分界线, 以下关于模表示

\item $E$ 为 $F$ 的扩域, $R$ 为 $F$-代数, $R':=E\otimes_{F} R$, $M'$ 为 $R'$-模, 若存在 $R$-模 $M$, 使得 $M' \cong E\otimes_{F} M$, 则称 $M'$ 定义在 $F$ 上(defined over $F$).

\item $p$-modular system

Let $p$ be a prime number dividing the order $\left|G\right|$ of the finite group $G$. Let $R$ be a complete discrete rank one valuation ring with maximal ideal $J(R)=\pi R$ such that its residue class field $F = R/\pi R$ has characteristic $p$ and its quotient field $S = quot(R)$ has characteristic zero. Then the triple $(F,R,S)$ is called a $p$-modular system for $G$.

If $F$ is algebraically closed and $S$ is a splitting field for $G$ and its subgroups, then $(F,R,S)$ is called to be a splitting $p$-modular system for $G$.\cite[p.112]{michler2006theory}

Let $R$ be a commutative ring. Then $R$ is called a \textit{discrete valuation ring} if it is a principal ideal domain that has a unique prime ideal $P$. Such $P$ is necessarily a maximal ideal and the field $R/P$ is called the \textit{residue class field} of $R$.

Let $R$ be discrete valuation ring. Then $R$ is obviously noetherian and we say that $R$ is \textit{complete} if it is complete as a noetherian local ring.

Let $p$ be a prime number. By a \textit{p-modular system} we understand a triple $(F,R,K)$ where $R$ is a complete discrete valuation ring of characteristic 0, $K$ is the quotient field of $R$ and $F$ is the residue class field of $R$ of characteristic $p$. A $p$-modular system $(F,R,K)$ is said to be \textit{sufficiently large} with respect to $G$ if $F$ and $K$ are splitting fields for $FH$ and $KH$, respectively where $H$ is any subgroup of $G$. It is a standard fact that sufficientlly large $p$-modular systems exist.

From now on $(F,R,K)$ denotes a $p$-modular system and all modules below are assumed to be finitely generated. Define an \textit{RG-lattice} as an $RG$-module which is R-free on a finite basis. A \textit{full RG-lattice} $W$ in a $KG$-module $V$ is an $RG$-lattice $W$ contained in $V$ such that $KW = V$ (or, equivalently, such that $dim_K V = rink_R W$).\cite[p.307]{karpilovsky1989clifford}\cite[p.140]{karpilovsky2016group}\cite[p.663]{karpilovsky1992part}
\end{enumerate}
\printbibliography
\end{document}