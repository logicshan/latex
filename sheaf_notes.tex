\documentclass{ctexart}

\usepackage{amsmath,amssymb,mathrsfs}

\setlength{\parindent}{0pt}

\newcommand{\F}{\mathscr{F}}
\newcommand{\NBR}[1]{\mathfrak{O}(#1)}
\newcommand{\nbr}[1]{\mathfrak{U}(#1)}

\begin{document}
$\F \colon \NBR{X}^{op} \to \mathbf{A\!b}$

$V \subset U, \rho_{VU} : \F(U) \to \F(V)$

对于拓扑空间 $X$ 上的预层 $\F$ 以及 $X$ 的一个开子集 $U$, $\F(U)$ 中的
元素 $s \in \F(U)$ 称为预层 $\F$ 在开子集 $U$ 上的\textbf{瓣}(\textit{section}). $\F(U)$
就是由 $U$ 上的瓣所构成的群. 全空间 $X$ 上的瓣被称为\textbf{整体
  瓣}(\textit{global section}). 我们往往把 $\F(U)$ 记为 $\Gamma(U,
\F)$. $\rho_{VU}$ 也被称为\textbf{限制映射}(\textit{restriction}). 如果 $s \in
\F(U)$, 常常把 $\rho_{VU}(s)$ 记为 $s|_{V}$.
\end{document}