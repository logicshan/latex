%-*- coding: UTF-8 -*-
% gougu.tex
% 勾股定理
\documentclass[UTF8]{ctexart}

\usepackage{float}
\usepackage{amsmath}
\usepackage{graphicx}
\usepackage{geometry}
\geometry{a6paper,centering,scale=0.8}
\usepackage[format=hang,font=small,textfont=it]{caption}
\usepackage[nottoc]{tocbibind}

\title{\heiti 杂谈勾股定理}
\author{\kaishu 张三}
\date{\today}

\bibliographystyle{plain}

\newtheorem{thm}{定理}
\newcommand\degree{^\circ}
\newenvironment{myquote}
  {\begin{quote}\kaishu\zihao{-5}}
  {\end{quote}}
\begin{document}


\maketitle
\begin{abstract}
  这是一篇关于勾股定理的小短文。
\end{abstract}
\tableofcontents
\section{勾股定理在古代}\label{sec:ancient}
西方称勾股定理为毕达哥拉斯定理, 将勾股定理的发现归功于公元前 6 世纪的毕达哥拉斯学派\cite{Kline}。 该学派得到了一个法则, 可以求出可排成直角三角形三边的三元数组。 毕达哥拉斯学派没有书面著作, 该定理的严格表述和证明则见于欧几里得\footnote{欧几里得, 约公元前 330--275 年。}《几何原本》的命题 47:“直角三角形斜边上的正方形等于两直角边上的两个正方形之和。” 证明是用面积做的。

我国《周髀算经》载商高(约公元前 12 世纪)答周公问:
\begin{myquote}
勾广三, 股修四, 径隅五。
\end{myquote}
又载陈子(约公元前 7--6 世纪)答荣方问:
\begin{myquote}
若求邪至日者, 以日下为勾, 日高为股, 勾股各自乘, 并而开方除之, 得邪至日。
\end{myquote}
都较古希腊更早。 后者已经明确道出勾股定理一般形式。 图 \ref{fig:xiantu} 是我国古代对勾股定理的一种证明\cite{quanjing}。

\begin{figure}[ht]
  \centering
  \includegraphics[scale=0.6]{xiantu.pdf}
  \caption{宋赵爽在《周髀算经》注中作的弦图(仿制), 该图给出了勾股定理的一个极具对称美的证明。}\label{fig:xiantu}
\end{figure}

\section{勾股定理的近代形式}
勾股定理可以用现代语言表述如下:
\begin{thm}[勾股定理]
直角三角形斜边的平方等于两腰的平方和。

可以用符号语言表述为: 设直角三角形 ABC, 其中 $\angle C = 90\degree$, 则有
\begin{equation}\label{eq:gougu}
AB^2 = BC^2 + AC^2
\end{equation}
\end{thm}

满足式 \eqref{eq:gougu} 的整数称为\emph{勾股数}。第 \ref{sec:ancient} 节所说毕达哥拉斯学派得到的三元数组就是勾股数。 下表列出一些较小的勾股数:
\begin{table}[H]
\begin{tabular}{|rrr|}
  \hline
直角边 $a$ & 直角边 $b$ & 斜边 $c$\\
\hline
  3 & 4 & 5 \\
  5 & 12 & 13 \\
\hline
\end{tabular}%
\qquad
($a^2 + b^2 = c^2$)
\end{table}
\nocite{Shiye}
\bibliography{math}

\end{document}