\documentclass{ctexart}

\usepackage{amsmath,amssymb,amsthm}
\usepackage{mathrsfs}
\usepackage{color}
\usepackage{tikz-cd}
\usepackage{url}

\usepackage{biblatex}
\addbibresource{algebra_notes.bib}

\newtheorem*{prop}{命题}
\newtheorem*{thm}{定理}

\DeclareMathOperator{\Ima}{Im}
\DeclareMathOperator{\im}{im}

\begin{document}

除非特别说明,以下考虑的环都为带乘法幺元的结合环。

环 $R$ 看做自身上的左模, 这时 $R$ 的左理想就是一个左 $R$-模.

考虑张量积的时候, 注意得到的是一个 Abel 群和一个 $R$-balanced 映射(或 $R$-双加映射, 或 $R$-bilinear 映射). 其他结构再在这个 Abel 群上考虑.

\begin{enumerate}
\item \cite[p.33]{barot2014introduction}交换环 $R$ 上的代数(称为 $R$-代数),就是一个环 $A$ 和一个包含映射 $\iota : R \to Z(A)$,其中 $Z(A)$ 为 $A$ 的中心,$\iota$ 为一个单的环同态。我们研究较多的是当 $R$ 为一个域 $k$ 时的 $k$-代数。$\iota$ 诱导出以下映射 $r \cdot a = \iota(r)a$ :
\begin{align*}
\cdot : R \times A &\to A\\
(r,a) &\mapsto \iota(r)a
\end{align*}
当 $R$ 为 $k$ 时,$A$ 就成为一个 $k$-向量空间,此时就可以考虑 $A$ 的维数。

\item 模的张量积

考虑一个环 $R$,做张量积的时候,要求一个是右$R$-模,一个是左$R$-模,得到的是一个$Abel$群,和一个 $R$-balanced 映射。当 $R$ 是交换环时,可对得到的$Abel$群赋予一个模结构。以下用更严格的语言描述。

首先定义什么是 $R$-balanced 映射。
对任意的环 $R$,$M$ 为一个右$R$-模,$N$ 为一个左 $R$-模,$G$ 为一个$Abel$ 群,一个映射 $\varphi : M \times N \to G$被称为 $R$-balanced,如果对任意的 $m,m'\in M, n,n'\in N, r\in R$,有以下等式成立:
\begin{align*}
&\varphi(m, n+n') = \varphi(m,n) + \varphi(m,n')&&\text{$(Dl_\varphi)$}\\
&\varphi(m+m', n) = \varphi(m,n) + \varphi(m',n)&&\text{$(Dr_\varphi)$}\\
&\varphi(m\cdot r, n) = \varphi(m, r\cdot n)&&\text{$(A_\varphi)$}
\end{align*}
注意:$\varphi$ 并不要求是群同态。若是群同态,则要满足
\begin{align*}
\varphi((m,n)+(m',n')) &= \varphi(m+m',n+n')\\
                       &= \varphi(m,n) + \varphi(m',n')
\end{align*}
从$M \times N$到$G$的所有$R$-balanced映射的集合记为$L_R(M,N;G)$。\\
如果$\varphi,\psi$为$R$-balanced映射,那么逐点(pointwise)定义的$\varphi + \psi$和$-\varphi$也是$R$-balanced映射。这样集合$L_R(M,N;G)$就成为一个$Abel$群。\\

下面定义模的张量积。
对任意的环 $R$,右 $R$-模 $M$,左 $R$-模 $N$,$M$ 和 $N$ 在 $R$ 上的张量积 $M \otimes_R N$ 是一个$Abel$群和一个 $R$-balanced 映射
\[
\otimes : M \times N \to M \otimes_R N
\]
并满足如下意义上的万有性质:\\
对任意的$Abel$群 $G$ 和任意的 $R$-balanced 映射
\[
f : M \times N \to G
\]
都存在唯一的一个\textcolor{red}{群同态}
\[
\tilde{f} : M \otimes_R N \to G
\]
使得
\[
\tilde{f} \circ \otimes = f
\]

\begin{center}
\begin{tikzcd}[row sep=huge]
M \!\!\times\!\! N \arrow[r,"\otimes"] \arrow[rd,"f"] & M \!\!\otimes_R\!\! N \arrow[d,dashed,"\tilde{f}" pos=0.37]\\
& G
\end{tikzcd}
\end{center}

\[
Hom_\mathbb{Z}(M \otimes_R N,G) \simeq L_R(M,N;G),g \mapsto g \circ \otimes
\]

以下证明$L_R(M,N;G) \simeq Hom_R(M,Hom_\mathbb{Z}(N,G))$
\begin{proof}
$Hom_\mathbb{Z}(N,G)$可被赋予一个右$R$-模结构:$(g \cdot r)(y) = g(ry)$.\\
\begin{align*}
\alpha : L_R(M,N;G) &\to Hom_R(M,Hom_\mathbb{Z}(N,G))\\
f &\mapsto f' = (m \mapsto (n \mapsto f(m,n)))
\end{align*}
下证$f'$为$R$-模同态
\begin{align*}
f'(xr) &= f'(x) \cdot r\\
f'(xr) &= (y \mapsto f'(x)(ry)\\
(y \mapsto f(xr,y) &= (y \mapsto f(x,ry))
\end{align*}
又$f$为$R$-balanced映射,$f(xr,y) = f(x,ry)$。所以,$f'(xr) = f'(x) \cdot r$。

\begin{align*}
\beta : Hom_R(M,Hom_\mathbb{Z}(N,G)) \to L_R(M,N;G)&\\
f \mapsto f' = (m \times n \mapsto f(m)(n))&
\end{align*}
下证$f'$为$R$-balanced映射。
\begin{align*}
f'(xr,y) &= f'(x,ry)\\
f(xr)(y) &= f(x)(ry)\\
(f(x) \cdot r)(y) &= f(x)(ry)\\
f(x)(ry) &= f(x)(ry)
\end{align*}

有$\alpha \circ \beta = 1_{Hom_R(M,Hom_\mathbb{Z}(N,G))},
   \beta \circ \alpha = 1_{L_R(M,N;G)}$
\end{proof}

$\mathbb{Z}$-模就是$Abel$群。

就像所有的万有性质一样,以上定义的张量积在同构的意义下是唯一的。

我们把映射 $\otimes$ 叫做标准(canonical)映射。

注意,这个定义并没有保证张量积$M \otimes_R N$一定存在,但后面会看到,我们可以把张量积明确构造出来。

$\forall x \in M, y \in N$,我们把$(x,y)$在标准映射$\otimes : M \times N \to M \otimes_R N$下的像记为$x \otimes y$。通常称为纯张量(pure tensor)。严格来说,正确的记法应为$x \otimes_R y$,不过按惯例我们通常省略$R$。由定义,我们就可以得到如下关系:
\begin{align*}
&x \otimes (y+y') = x \otimes y + x \otimes y'&&\text{$(Dl_\otimes)$}\\
&(x+x') \otimes y = x \otimes y + x' \otimes y&&\text{$(Dr_\otimes)$}\\
&(x \cdot r) \otimes y = x \otimes (r \cdot y)&&\text{$(A_\otimes)$}
\end{align*}

张量积的万有性质有如下重要的结论:

\begin{prop}
$M \otimes_R N$中的每个元素都可以写成
\[
\sum_i x_i \otimes y_i, x_i \in M, y_i \in N
\]
的形式(并不要求唯一)。也就是说,$\otimes$的像生成了$M \otimes_R N$。 更进一步,如果$f$对每一个元素$x \otimes y$都给出了$Abel$群$G$中的一个值,那么$f$可以扩展为定义在整个$M \otimes N$上的群同态当且仅当$f(x \otimes y)$对$x$和$y$都是$\mathbb{Z}$-双线性的。
\end{prop}
\begin{proof}
对第一条陈述,令$L$为形式为$x \otimes y$的元素所生成的$M \otimes_R N$的子群,$Q = (M \otimes_R N)/L,  q : M \otimes_R N \to (M \otimes_R N) / L$。\\
我们有:$0 = q \circ \otimes$和$0 = 0 \circ \otimes$。
\begin{center}
\begin{tikzcd}[row sep=huge]
M \!\!\times\!\! N \arrow[r,"\otimes"] \arrow[rd,"0"] & M \!\!\otimes_R\!\! N \arrow[d,dashed,"q" pos=0.37]\\
& (M \!\!\otimes_R\!\! N) / L
\end{tikzcd}
\end{center}
由万有性质,$q = 0$。又$q$为满射,所以$Q = (M \otimes_R N)/L = 0$,所以$M \otimes_R N = L$。
\end{proof}
这个命题说的是,我们可以使用张量积的明确形式,而不是使用万有性质。这在实际中是很方便的。

$f : M \to M'$是右$R$-模线性映射,$g : N \to N'$是左$R$-模线性映射,则有一个唯一的群同态
\[
f \otimes g : M \otimes_R N \to M' \otimes_R N'
\]
满足,$(f \otimes g)(x \otimes y) = f(x) \otimes g(y)$
\begin{center}
\begin{tikzcd}[row sep=huge]
M \!\!\times\!\! N \arrow[r,"\otimes"] \arrow[rd,"\otimes"] & M \!\!\otimes_R\!\! N \arrow[d,dashed,"f \otimes g" pos=0.37]\\
& M' \!\!\otimes_R\!\! N'
\end{tikzcd}
\end{center}

张量可成为一个函子
\[
M \otimes_R - : R\!\!-\!\!\mathbf{Mod} \to \mathbf{Ab}
\]
从左$R$-模范畴到$Abel$群范畴
\begin{align*}
N &\mapsto M \otimes N \\
f &\mapsto 1 \otimes f
\end{align*}

若有一个环同态$f : R \to S$,并且$M$是一个右$S$-模,$N$是一个左$S$-模,则有一个标准的满同态
\[
M \otimes_R N \to M \otimes_S N
\]
由 $M \times N \overset{\otimes_S}{\to} M \otimes_S N$ 诱导。
\begin{center}
\begin{tikzcd}[row sep=huge]
M \!\!\times\!\! N \arrow[r,"\otimes_R"] \arrow[rd,"\otimes_S"] & M \!\!\otimes_R\!\! N \arrow[d,dashed]\\
& M \!\!\otimes_S\!\! N
\end{tikzcd}
\end{center}

\item 模

令$M$为一个$Abel$群,$R$为一个环,$\mu : R \times M \to M$为一个给定的映射。则三元组$(M, R, \mu)$被叫做一个左$R$-模,如果对$\forall r,s \in R, x,y \in M$满足下面的公理:
\begin{align*}
&(1) \mu(r, x + y) = \mu(r, x) + \mu(r, y)\\
&(2) \mu(r + s, x) = \mu(r, x) + \mu(s, x)\\
&(3) \mu(rs, x) = \mu(r, \mu(s, x))
\end{align*}

\item $R$-代数既有模结构又有环结构

\item 自由模

令$S$为一个集合,$R$为一个幺环。则有序对$(M, i)$是一个$S$上的自由
左$R$-模,如果$M$是一个左$R$-模,$i : S \to M$是一个集合映射,并且满足
以下万有性质:对$\forall$左$R$-模$N$和集合映射$f : S \to N$,存在唯一
的$R$模映射$g : M \to N$满足下面的交换图
\begin{center}
\begin{tikzcd}
S \arrow[r,"i"] \arrow[rd,"f"] & M \arrow[d, "g"]\\
& N
\end{tikzcd}
\end{center}

我们说一个$R$-模$F$是自由的,如果存在一个$S$上的自由模$(M,i)$使得$F \cong M$。

\begin{thm}
令$(F,i)$为一个$S$上的自由左$R$-模,则 $i$ 是单射且$\Ima i$生成 $F$。
\end{thm}

\begin{thm}
令$(F,i)$和$(K,j)$是$S$上的两个自由模,则$F \cong K$。
\end{thm}

\begin{thm}
任意集合上的自由模一定存在。
\end{thm}

我们现在定义一个模的基的概念,并证明一个模存在基等价于这个模是自由的。
我们把左$R$-模$M$的一个有限集合$\{m_1,\cdots ,m_n\}$叫做线性无关,如果由
$\sum_ir_im_i = 0$可推出$r_1 = \cdots = r_n = 0$。我们把一个子集
$\{m_\alpha\}_{\alpha \in \mathcal{A}}$叫做线性无关,如果此集合的任意有
限子集都线性无关。$M$的一个基就是$M$的一个子集$\mathscr{B}$,这个子集即生成$M$又线
性无关。

\begin{thm}
$M$为左$R$-模,则$\mathscr{B}$为$M$的一个基当且仅当$M$中的每一个元素都
可以唯一写成$\mathscr{B}$中元素的$R$线性组合。
\end{thm}

\begin{thm}
若$(F, i)$是$S$上的一个自由模,则$\Ima i$是$F$的一个基。
\end{thm}

\begin{thm}
令$R$是一个幺环,$F$是一个左$R$-模,则$F$是自由的当且仅当$F$
有一个基。
\end{thm}

\item 代数的张量积

下面我们讨论怎样做代数的张量积并得到一个代数。值得注意的一点是,做代数
的张量积并不真正不同于做模的张量积,但是张量积在
$R\text{-}\mathbf{Alg}$中的“地位”和它在$R\text{-}\mathbf{Mod}$中的
“地位”是完全不同的。这可以很清楚地从对$R\text{-}\mathbf{Alg}$中张量
积的万有性质的刻画中看出。

假设我们有一个交换幺环$R$,$A,B$为$R\text{-}$代数,则$A$可看作右
$R\text{-}$模,$B$可看作左$R\text{-}$模,我们可以做他们通常的张量积$A
\otimes_R B$,而且由于$R$是交换的,得到的张量积有一个自然的$R\text{-}$模结构,
$r(a\otimes b) = (ra)\otimes b$。问题是,我们能否在$A \otimes_R B$上定义
一个环结构,和$R\text{-}$模结构兼容,以得到一个$R\text{-}$代数。

\begin{thm}
令$A, B$为$R\text{-}$代数,则$A \otimes_R B$具有通常的$R\text{-}$模结构,
并且可定义乘法$(a\otimes b)(a' \otimes b') = (aa) \otimes (bb')$,使得
它成为一个$R\text{-}$代数。
\end{thm}

\begin{thm}
\begin{align*}
A \to A \otimes_R B\\
a \mapsto a \otimes 1_B\\
B \to A \otimes_R B\\
b \mapsto 1_A \otimes b
\end{align*}
这两个映射使得张量积在交换$R\text{-}$代数范畴中成为$coproduct$。但是张
量积在所有$R\text{-}$代数范畴中并不是$coproduct$。
\end{thm}

\item 有限生成代数\cite{finitely_generated_algebra}
在数学中,一个有限生成代数(又叫做有限型代数),是一个域$K$上的结合代
数$A$,存在$A$中元素的一个有限集$\{a_1,\cdots,a_n\}$,使得$A$中每一个
元素都可以表示为$a_1,\cdots,a_n$的系数在$K$中的多项式,即$A$中每一个元
素都可以表示成:
\[
\sum\nolimits_ik_ia_1^{\alpha i1}\cdots a_n^{\alpha in}
\]

\item 半单模(Semisimple module)\cite{semisimple_module}

一个环(不一定可交换,但有幺元)上的模被叫做半单(或者完全可约), 如果它是单(不可约)子模的直和.

对一个模 $M$, 以下陈述等价:
\begin{enumerate}
  \item $M$ 是半单的; 也就是说, 是一些不可约模的直和.
  \item $M$ 是它的不可约子模的直和.
  \item $M$ 的每一个子模都是直和因子: 对每一个 $M$ 的子模 $N$, 都有一个补直和因子 $P$, 使得 $M = N \oplus P$.
\end{enumerate}

半单模最基本的例子就是域上的模; 也就是向量空间. 另一方面, 整数环 $\mathbb{Z}$ 不是它自身上的半单模(因为,例如,它不是一个 Artin 环.)(\textcolor{red}{半单环一定是 Artin 环}. 半单环定义见下面.)

半单要强于完全可分解(completely decomposable), 完全可分解模是不可分解子模的直和.

如果 $M$ 是半单模, $N$ 是它的一个子模, 那么 $N$ 和 $M/N$ 也是半单的.

如果每一个 $M_i$ 是半单的, 那么 $\underset{i}{\bigoplus}M_i$ 也是半单的.

模 $M$ 是有限生成的并且是半单的, 当且仅当它是 Artin 的,并且根基(radical)为 $0$.

一个环是(左)半单的,如果它自身作为左模是半单的. 出人意料的是,一个左半单环同时也是右半单的,反过来也是一样(一个右半单环也是左半单的). 左和右的区别是不必要的,我们可以毫不含糊的说一个环是半单的.

半单环也可以用同调代数的语言来刻画:也就是说,一个环 $R$ 是半单的,当且仅当任意的左(或右)$R$-模短正合列是分裂的. 特别有, 半单环上的任意一个模, 既是入射的(injective),又是投射的(projective).

如果环 $R$ 是半单的,则所有的 $R$-模是半单的. 更进一步, 每一个单 $R$-模同构于 $R$ 的一个极小左理想(minimal left ideal).

半单环既是 Artin 的,又是 Noether 的. 从以上的性质可以得到, 一个环是半单的, 当且仅当它是 Artin 的,并且它的 Jacobson 根基为 $0$.

如果一个 Artin 半单环包含一个域作为它的中心子环(central subring)(也就是这个域包含于环的中心), 那么这个环就被叫做一个半单代数(semisimple algebra).

没有非平凡双边理想的环叫做单环. \textcolor{red}{不是所有的单环都是半单的}. The problem is that the ring may be "too big", that is, not (left/right) Artinian. In fact, if R is a simple ring with a minimal left/right(只要求左或右,不要求双边) ideal, then R is semisimple.

\textbf{Jacobson半单} 或 semiprimitive. 一个环叫做 semiprimitive 或 Jacobson半单, 如果它的 Jacobson根基为 $0$.

每一个半单环的 Jacobson根基都为 $0$, 但并不是每一个 Jacobson根基为 $0$ 的环都是半单的. 一个 Jacobson半单的环是半单的, 当且仅当它是 Artin 环. 所以, 为了防止混淆, 半单环经常被叫做 Artin 半单环.

\item Jacobson radical\cite{jacobson_radical}

环 $R$ 的 Jacobson 根基是一个理想, 这个理想由消灭(annihilate)所有单右 $R$-模的元素组成. 凑巧的是, 把定义中的"右"换成"左", 得到相同的理想. 所以,这个概念是左右对称的.

$J(R)$ equals the intersection of all maximal right ideals of the ring. It is also true that $J(R)$ equals the intersection of all maximal left ideals within the ring. These characterizations are internal to the ring, since one only needs to find the maximal right ideals of the ring. For example, if a ring is local, and has a unique maximal right ideal, then this unique maximal right ideal is an ideal because it is exactly $J(R)$. The left-right symmetry of these two definitions is remarkable and has various interesting consequences. This symmetry stands in contrast to the lack of symmetry in the socles of $R$, for it may happen that $soc(R_R)$ is not equal to $soc(_RR)$. If $R$ is a non-commutative ring, $J(R)$ is not necessarily equal to the intersection of all maximal two-sided ideals of $R$.

$J(R)$ equals the sum of all superfluous right ideals (or symmetrically, the sum of all superfluous left ideals) of $R$. Comparing this with the previous definition, the sum of superfluous right ideals equals the intersection of maximal right ideals. This phenomenon is reflected dually for the right socle of $R$: $soc(R_R)$ is both the sum of minimal right ideals and the intersection of essential right ideals. In fact, these two relationships hold for the radicals and socles of modules in general.

As defined in the introduction, $J(R)$ equals the intersection of all annihilators of simple right $R$-modules, however it is also true that it is the intersection of annihilators of simple left modules. An ideal that is the annihilator of a simple module is known as a primitive ideal, and so a reformulation of this states that the Jacobson radical is the intersection of all primitive ideals. This characterization is useful when studying modules over rings. For instance, if $U$ is right $R$-module, and $V$ is a maximal submodule of $U$, $U\cdot J(R)$ is contained in $V$, where $U\cdot J(R)$ denotes all products of elements of $J(R)$ (the "scalars") with elements in $U$, on the right. This follows from the fact that the quotient module, $U/V$ is simple and hence annihilated by $J(R)$.

$J(R)$ is the unique right ideal of $R$ maximal with the property that every element is right quasiregular. Alternatively, one could replace "right" with "left" in the previous sentence. This characterization of the Jacobson radical is useful both computationally and in aiding intuition. Furthermore, this characterization is useful in studying modules over a ring. Nakayama's lemma is perhaps the most well-known instance of this. Although every element of the $J(R)$ is necessarily quasiregular, not every quasiregular element is necessarily a member of $J(R)$.

While not every quasiregular element is in $J(R)$, it can be shown that $y$ is in $J(R)$ if and only if $xy$ is left quasiregular for all $x$ in $R$.

$J(R)$ is the set of all such elements $x \in R$ that every element of $1 + RxR$ is a unit: $J(R) = \{ x \in R \mid 1 + RxR \subset R^\times \}$.

\item 模的根基(Radical of a module)\cite{radical_of_a_module}

令 $R$ 为一个环, $M$ 是一个 $R$-模. $M$ 的一个子模 $N$ 被称为极大的或 cosimple, 如果商模 $M/N$ 是一个单模. 模 $M$ 的根基就是 $M$ 所有极大子模的交
\[
rad(M) = \bigcap\{N\mid N \mbox{ is a maximal submodule of } M\}
\]
等价地,
\[
rad(M) = \sum\{ S \mid S \mbox{ is a superfluous submodule of } M\}
\]
$soc(M)$ 可以类似以上定义对偶的定义.

\item Socle of a module\cite{socle_of_a_module}

\[
soc(M) = \sum\{ N \mid N \mbox{ is a simple submodule of } M \}
\]

\[
soc(M) = \bigcap\{ E \mid E \mbox{ is an essential submodule of } M \}
\]

The \textbf{socle of a ring} $R$ can refer to one of two sets in the ring. Considering $R$ as a right $R$ module, $soc(R_R)$ is defined, and considering $R$ as a left $R$ module, $soc(_RR)$ is defined. Both of these socles are \textcolor{red}{ring ideals}, and it is known they are not necessarily equal.

如果 $M$ 是一个 Artin 模, 那么 $soc(M)$ 本身是 $M$ 的 essential submodule.

一个模是半单的, 当且仅当 $soc(M) = M$. 如果对一个环上的所有模 $M$, 都有 $soc(M) = M$, 则这个环就是半单的.

$soc(soc(M)) = soc(M)$

Since the sum of semisimple modules is semisimple, the socle of a module could also be defined as the unique maximal semi-simple submodule.

From the definition of $rad(R)$, it is easy to see that $rad(R)$ annihilates $soc(R)$. If $R$ is a finite-dimensional unital algebra and $M$ a finitely generated $R$-module then the socle consists precisely of the elements annihilated by the Jacobson radical of $R$.

\item Quasiregular element\cite{quasiregular_element}

Let $R$ be a ring (with unity) and let $r$ be an element of $R$. Then $r$ is said to be quasiregular, if $1 - r$ is a unit in $R$; that is, invertible under multiplication. The notions of right or left quasiregularity correspond to the situations where $1 - r$ has a right or left inverse, respectively.

\item Essential extension\cite{essential_extension}

In mathematics, specifically module theory, given a ring $R$ and $R$-modules $M$ with a submodule $N$, the module $M$ is said to be an essential extension of $N$ (or $N$ is said to be an essential submodule or large submodule of $M$) if for every submodule $H$ of $M$,
\[
    H \cap N = \{ 0 \} \mbox{ implies that } H = \{ 0 \}
\]
As a special case, an essential left ideal of $R$ is a left ideal which is essential as a submodule of the left module $_RR$. The left ideal has non-zero intersection with any non-zero left ideal of $R$. Analogously, and essential right ideal is exactly an essential submodule of the right R module $R_R$

The dual notion of an essential submodule is that of superfluous submodule (or small submodule). A submodule $N$ is superfluous if for any other submodule $H$,
\[
N + H = M \mbox{ implies that } H = M
\]

\item 偏序(partially ordered), 格(lattice) \cite{hazewinkel2004algebras}

一个集合 $S$ 被称为偏序集, 如果它配备有一个关系 $\leq$, 这个关系满足下面的条件:
\begin{enumerate}
  \item $a \leq a, \forall a \in S$ (反身性);
  \item $a \leq b, b \leq c \Rightarrow a \leq c, \forall a,b,c \in S$ (传递性);
  \item $a \leq b, b \leq a \Rightarrow a = b, \forall a,b \in S$ (反对称性).
\end{enumerate}
这个关系 $\leq$ 叫做\textbf{偏序}.

令 $S$ 是一个偏序集, $T$ 是 $S$ 的一个子集. $T$ 的一个上界(下界)是 $S$ 中的一个元素 $a \in S$, 满足 $t \leq a (a \leq t), \forall t \in T$. 一般地, 一个集合可能有多个上界, 也可能一个都没有.

$T$ 中的一个元素 $a \in T$ 被称为 $T$ 的最大元(最小元), 如果 $t \leq a (a \leq t), \forall t \in T$. 不是偏序集 $S$ 的所有子集 $T$ 都有一个最大元(或最小元). 但是如果 $T$ 有这么一个元素, 那它就是唯一的.
所以, 如果一个极大元(极小元)存在, 那么它是唯一的, 并且是 $T$ 的一个上界(下界). 如果 $T$ 的所有上界组成的集合有一个最小元, 那么就把它叫做 $T$ 的最小上界(或上确界), 记作 $sup(T)$. 如果 $T$ 的所有下界组成的集合有一个最大元, 那么就把它叫做 $T$ 的最大下界(或下确界), 记作 $inf(T)$.

一个偏序集 $S$, 如果 $S$ 中的每一对元素在 $S$ 中都有一个上确界和一个下确界, 那么 $S$ 就被称为\textbf{格(lattice)}.

令 $S$ 为一个格. 那么 $S$ 中的任意两个元素 $a,b \in S$ 都有上确界和下确界. 我们记
\[
a \vee b = \sup\{a,b\} \mbox{  和  } a \wedge b = \inf\{a,b\}
\]
那么从 $S \times S$ 到 $S$ 的映射 $\vee$ 和 $\wedge$
\[
(a,b) \mapsto a \vee b \mbox{  和  } (a,b) \mapsto a \wedge b
\]
就成为 $S$ 上的二元运算.
\end{enumerate}
\printbibliography
\end{document}