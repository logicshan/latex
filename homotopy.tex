\documentclass{ctexart}

\usepackage{amsmath,amssymb,amsthm,tikz-cd,mathrsfs}
\usepackage{color}
\DeclareMathOperator{\Kom}{Kom}
\begin{document}
\begin{enumerate}
\item 链复形(Chain complex)

$d^2 = 0$
\item 链映射(Chain map)

It is worth noticing that the concept of chain map reduces to the one of boundary through the construction of the cone of a chain map.

\item 链同伦(Chain homotopy)

Chain homotopies give an important equivalence relation between chain maps. Chain homotopic chain maps induce the same maps on homology groups. A particular case is that homotopic maps between two spaces $X$ and $Y$ induce the same maps from homology of $X$ to homology of $Y$.

\item Quasi-isomorphism

In homological algebra, a branch of mathematics, a quasi-isomorphism is a morphism $A\to B$ of chain complexes (respectively, cochain complexes) such that the induced morphisms
\[
H_n(A_\bullet) \to H_n(B_\bullet)\ (\text{respectively, } H^n(A^\bullet) \to H^n(B^\bullet))
\]
of homology groups (respectively, of cohomology groups) are isomorphisms for all $n$.

\item Homotopy category of chain complexes

In homological algebra in mathematics, the homotopy category $K(A)$ of chain complexes in an additive category $A$ is a framework for working with chain homotopies and homotopy equivalences. It lies intermediate between the category of chain complexes $\Kom(A)$ of $A$ and the derived category $D(A)$ of $A$ when $A$ is abelian; unlike the former it is a triangulated category, and unlike the latter its formation does not require that $A$ is abelian. Philosophically, while $D(A)$ makes isomorphisms of any maps of complexes that are quasi-isomorphisms in $\Kom(A)$, $K(A)$ does so only for those that are quasi-isomorphisms for a "good reason", namely actually having an inverse up to homotopy equivalence. Thus, $K(A)$ is more understandable than $D(A)$.

$\Kom(A)$ - the category of chain complexes in an \textcolor{blue}{abelian category $A$}.

$K(A)$ - homotopy category of chain complexes in an \textcolor{red}{additive category $A$}.

$D(A)$ - the derived category $D(A)$ in an \textcolor{blue}{abelian category $A$}.

A morphism $f:A\rightarrow B$ which is an isomorphism in $K(A)$ is called a \textbf{homotopy equivalence}. In detail, this means there is another map $g : B\rightarrow A$, such that the two compositions are homotopic to the identities: $f\circ g\sim Id_{B}$ and $g\circ f\sim Id_{A}$.

$K(A)$ 中的态射为模去同伦等价的等价类, 所以 $f:A\to B$实际是$[f]:A\to B$(和$f$同伦的等价类), $[f]$是同构, 就是存在另外一个等价类$[g]:B\to A$, 使得 $[f]\circ [g] = [Id_B], [g]\circ [f]=[Id_A]$, 即 $f\circ g\sim Id_{B}$ and $g\circ f\sim Id_{A}$.

Two chain homotopic maps $f$ and $g$ induce the same maps on homology because $(f − g)$ sends cycles to boundaries, which are zero in homology. In particular a homotopy equivalence is a quasi-isomorphism. (The converse is false in general.) This shows that there is a canonical functor $K(A)\rightarrow D(A)$ to the derived category (if $A$ is abelian).

\item Zig-zag lemma

从复形的短正合列得到同调的长正合列.


\item Homotopy extension property

In mathematics, in the area of algebraic topology, the homotopy extension property indicates which homotopies defined on a subspace can be extended to a homotopy defined on a larger space.
\begin{center}
\begin{tikzcd}
Y \arrow[r, leftarrow, "\tilde{f}_0"] \arrow[d, twoheadleftarrow, "p_0"']&X \arrow[ld, dashrightarrow, "\tilde{f}"]\\
Y^I \arrow[r, leftarrow, "f"'] & A \arrow[u, "i"']
\end{tikzcd}
\end{center}
$i : A \to X$ 对于 $Y$ 满足同伦扩张性质, 如果给定交换图, 存在 $\tilde{f} : X\to Y^I$ 使得图交换.

\item Cofibration

In mathematics, in particular homotopy theory, a continuous mapping
\[
i\colon A\to X,
\]
where $A$ and $X$ are topological spaces, is a cofibration if it satisfies the homotopy extension property with respect to \textcolor{red}{all spaces $Y$}. This definition is dual to that of a fibration, which is required to satisfy the homotopy lifting property with respect to all spaces. This duality is informally referred to as Eckmann–Hilton duality.

In fact, if you consider any cofibration $i:Y\to Z$, then we have that $Y$ is homeomorphic to its image under $\mathbf{\mathit {i}}$ . This implies that any cofibration can be treated as an inclusion map.

\item Mapping cylinder

In mathematics, specifically algebraic topology, the mapping cylinder of a function $f$ between topological spaces $X$ and $Y$ is the quotient
\[
M_{f}=(([0,1]\times X)\amalg Y)\,/\,\sim
\]

The mapping cylinder may be viewed as a way to replace an arbitrary map by an equivalent cofibration, in the following sense:

Given a map $f\colon X\to Y$, the mapping cylinder is a space $M_{f}$, together with a cofibration $\tilde f\colon X \to M_f$ and a surjective homotopy equivalence $M_f \to Y$ (indeed, $Y$ is a deformation retract of $M_{f}$), such that the composition $X \to M_f \to Y$ equals $f$.
\begin{center}
\begin{tikzcd}
X \arrow[rr, "f"] \arrow[rd, "\tilde f"'] & & Y\\
& M_f \arrow[ru]
\end{tikzcd}
\end{center}

\item Homotopy lifting property and fibration

Given a map $\pi \colon E\to B$, and a space $X\,$, one says that $(X,\pi )\,$ has the homotopy lifting property, or that $\pi \,$ has the homotopy lifting property with respect to $X\,$, if:

    for any homotopy $f\colon X\times [0,1]\to B\,$, and

    for any map ${\tilde {f}}_{0}\colon X\to E$ lifting $f_{0}=f|_{X\times \{0\}}$ (i.e., so that $f_{0}=\pi \circ {\tilde {f}}_{0}\,$),

there exists a homotopy ${\tilde {f}}\colon X\times [0,1]\to E$ lifting $f\,$ (i.e., so that $f=\pi \circ {\tilde {f}}\,$) which also satisfies ${\tilde {f}}_{0}={\tilde {f}}|_{X\times \{0\}}\,$.

The following diagram depicts this situation.

\begin{center}
\begin{tikzcd}
X \arrow[r, "{\tilde f}_0"] \arrow[d, "X\times \{0\}"']& E \arrow[d, "\pi"]\\
X\times I \arrow[r, "f"'] \arrow[ru, dashrightarrow, "{\tilde f}"] & B
\end{tikzcd}
\end{center}


The outer square (without the dotted arrow) commutes if and only if the hypotheses of the lifting property are true. A lifting ${\tilde {f}}$ corresponds to a dotted arrow making the diagram commute. This diagram is dual to that of the homotopy extension property; this duality is loosely referred to as Eckmann–Hilton duality.

If the map $\pi\,$ satisfies the homotopy lifting property with respect to all spaces $X$, then $\pi\,$ is called a fibration, or one sometimes simply says that $\pi\,$ has the homotopy lifting property.

Note that this is the definition of fibration in the sense of Hurewicz, which is more restrictive than fibration in the sense of Serre, for which homotopy lifting only for $X\,$ a CW complex is required.

\item Mapping cone (homological algebra)

Suppose now that we are working over an abelian category, so that the cohomology of a complex is defined. The main use of the cone is to identify quasi-isomorphisms: if the cone is acyclic, then the map is a quasi-isomorphism.

\item Relative homology


\end{enumerate}
\end{document}