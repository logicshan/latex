\documentclass{amsart}
%\usepackage[UTF8]{ctex}
% The following AMS packages are automatically loaded with
% the amsart documentclass:
%\usepackage{amsmath}
\usepackage{amssymb}
%\usepackage{amsthm}
\usepackage{color}
\usepackage{filecontents}
%\usepackage{amsxtra,amsfonts}
\usepackage{mathrsfs} %\mathscr
% For commutative diagrams you can use
% \usepackage{amscd}
\usepackage[all]{xy}
% We use 2cell for 2-commutative diagrams.
%\xyoption{2cell}
%\UseAllTwocells
%\usepackage{verbatim}
%-----------------------------------------------------------
%\usepackage{makeidx}
%\usepackage{graphicx}

%Two-column Proof
%\usepackage{array}
%\newcolumntype{L}[1]{>{\raggedright\let\newline\\\arraybackslash\hspace{0pt}}p{#1}}
%\newcolumntype{C}[1]{>{\centering\let\newline\\\arraybackslash\hspace{0pt}}p{#1}}
%\newcolumntype{R}[1]{>{\raggedleft\let\newline\\\arraybackslash\hspace{0pt}}p{#1}}
%\newenvironment{twopf}{\tabular{L{10cm}L{10cm}}}{\endtabular}

%\usepackage{multicol}

% For cross-file-references
%\usepackage{xr-hyper}
% Package for hypertext links:
\usepackage[backref=page]{hyperref} %\cite[thm#]{} %\label{key} %\ref{key}
\hypersetup{
    %bookmarks=true,
    colorlinks=true,
    linkcolor=blue,
    citecolor=red,
    %filecolor=magenta,      % color of file links
    %urlcolor=cyan
    }
    \renewcommand\backrefxxx[3]{%
          \hyperlink{page.#1}{$\uparrow$#1}%
        }
%\usepackage{showkeys}
\usepackage[capitalize,nameinlink]{cleveref}
%\usepackage{cleveref} %\cref
%--------------------------------------------------------------------------
% Theorem environments.
%
\numberwithin{equation}{section}
\theoremstyle{plain}
\newtheorem{thm}[equation]{Theorem}

\theoremstyle{definition}
\newtheorem{prop}[equation]{Proposition}
\newtheorem{lem}[equation]{Lemma}
\newtheorem{cor}[equation]{Corollary}
\newtheorem{defn}[equation]{Definition}
\newtheorem{conj}[equation]{Conjecture}
\newtheorem{eg}[equation]{Example}
\newtheorem{xca}[equation]{Exercise}
\newtheorem{as}[equation]{Assumption}
\newtheorem{nt}[equation]{Notation}
\newtheorem{fact}[equation]{Fact}
\newtheorem{ww}[equation]{}
%\theoremstyle{remark}
\newtheorem{rem}[equation]{Remark}
\newtheorem*{rem*}{Remark}

%Fonts
%
%\usepackage[T1]{fontenc}
%\usepackage{arev}
%\usepackage{tgbonum}
%\usepackage{bookman}
%\usepackage{pslatex}
%\usepackage{palatino}
%\usepackage{times}
%\usepackage{avant}
%\usepackage{charter}
%\usepackage{courier}
%\usepackage{helvet}
%\usepackage{newcent}
%\usepackage{punk}%Stupid Font
%\usepackage{chancery}%Fancy
%\usepackage{utopia}
%EndFonts

%\usepackage{setspace}
%\singlespacing
%\onehalfspacing
%\doublespacing
%\setstretch{1.1}

%\usepackage[landscape]{geometry}
%\usepackage{geometry}
%\geometry{papersize={20cm,100cm}}
%\usepackage{color}

%----------------------------------------------------------
%\newcommand{\IFF}{if and only if}
%\newcommand{\ST}{such that}
%\newcommand{\TFAE}{the following are equivalent}
%\newcommand{\WLOG}{without loss of generality}

%\newcommand{\thm}{theorem}
%\newcommand{\lem}{lemma}
%\newcommand{\prop}{proposition}
%\newcommand{\cor}{corollary}
%\newcommand{\rem}{remark}
%\newcommand{\eg}{example}
%\newcommand{\iso}{isomorphism}
\newcommand{\dg}[8]{\ensuremath{
\xymatrix{
#1\ar[r]^{#2}\ar[d]_{#4} & #3\ar[d]^{#5}\\
#6\ar[r]^{#7} & #8\\
}
}}
\newcommand{\mat}[9]{\ensuremath{
\left[
\begin{array}{cccc}
#1 & #2 & \cdots & #3\\
#4 & #5 & \cdots & #6\\
\vdots & \vdots & \ddots & \vdots\\
#7 & #8 & \cdots & #9
\end{array}
\right]
}}
%\newcommand{\mat22}[4]{\ensuremath{
%\left(
%\begin{array}{cc}
%#1 & #2\\
%#3 & #4\\
%\end{array}
%\right)
%}}

%\newcommand{\mat21}[2]{\ensuremath{
%\left(
%\begin{array}{c}
%#1 \\
%#2 \\
%\end{array}
%\right)
%}}
\newcommand{\ses}[3]{0\to #1 \to #2 \to #3 \to 0}
%\newcommand{\les}[4]{\cdots \to #1 \to #2 \to #3 \to #4 \to \cdots}
\newcommand{\ST}{such that }
\newcommand{\TE}{there exists }
\newcommand{\wrt}{with respect to }
\newcommand{\TFAE}{the following are equivalent }

% Macros
%
%\def\lim{\mathop{\rm lim}\nolimits}
%\def\colim{\mathop{\rm colim}\nolimits}
%\def\Spec{\mathop{\rm Spec}}
\def\SheafHom{\mathop{\mathcal{H}\!{\it om}}\nolimits}
\def\SheafEnd{\mathop{\mathcal{E}\!{\it nd}}\nolimits}
%\def\SheafHom{\mathop{\mathcal{H}\!{\it om}}\nolimits}
%\def\Sch{\textit{Sch}}
%\def\Mor{\mathop{\rm Mor}\nolimits}
%\def\Ob{\mathop{\rm Ob}\nolimits}
%\def\Sh{\mathop{\textit{Sh}}\nolimits}
\DeclareMathOperator{\Aut}{Aut}
\DeclareMathOperator{\Ab}{\mathbf{Ab}}
\DeclareMathOperator{\an}{an} %anisotropic part
\DeclareMathOperator{\Br}{Br} %brauer group
\DeclareMathOperator{\Char}{char} %characteristic
\DeclareMathOperator{\codim}{codim} %codimension
\DeclareMathOperator{\cok}{coker}
\DeclareMathOperator{\diag}{diag} %diagonal matrix
\DeclareMathOperator{\disc}{disc}
\DeclareMathOperator{\End}{End} %endomorphism
\DeclareMathOperator{\et}{\acute{e}t}
\DeclareMathOperator{\Frac}{Frac} %field of fractionss
\DeclareMathOperator{\Gal}{Gal} %galois
\DeclareMathOperator{\GL}{GL} %general linear group
\DeclareMathOperator{\Herm}{Herm} %hermitian
\DeclareMathOperator{\Hom}{Hom}
\DeclareMathOperator{\hyp}{hyp} %hyperbolic part
\DeclareMathOperator{\Id}{Id}
\DeclareMathOperator{\Ind}{ind}
\DeclareMathOperator{\Inn}{Inn}
\DeclareMathOperator{\Int}{Int}
\DeclareMathOperator{\Iso}{Iso}
\DeclareMathOperator{\lcm}{lcm}
\DeclareMathOperator{\Mod}{\mathbf{Mod}}
\DeclareMathOperator{\Nrd}{Nrd}
\DeclareMathOperator{\Op}{op}
\DeclareMathOperator{\Open}{\mathbf{Open}}
\DeclareMathOperator{\Or}{O} %orthogonal group
\DeclareMathOperator{\Per}{per}
\DeclareMathOperator{\Pre}{pre}
\DeclareMathOperator{\PreAb}{\mathbf{PreAb}}
\DeclareMathOperator{\PreMod}{\mathbf{PreMod}}
\DeclareMathOperator{\Proj}{Proj}
\DeclareMathOperator{\ram}{ram}
\DeclareMathOperator{\Rank}{Rank}
\DeclareMathOperator{\rdim}{rdim} %reduced dimension
\DeclareMathOperator{\Sh}{\mathbf{Sh}}
\DeclareMathOperator{\SL}{SL}
\DeclareMathOperator{\SO}{SO}
\DeclareMathOperator{\Sp}{Sp}
\DeclareMathOperator{\Spec}{Spec}
\DeclareMathOperator{\Spin}{Spin}
\DeclareMathOperator{\SU}{SU}
\DeclareMathOperator{\Supp}{Supp}
\DeclareMathOperator{\Sym}{Sym}
\DeclareMathOperator{\Tr}{Tr}
%\DeclareMathOperator{\U}{U}
\DeclareMathOperator{\op}{op}
\DeclareMathOperator{\im}{Im}
\DeclareMathOperator{\can}{can}
%
%\newenvironment{ww}
%  {$\left\{\tabular{l}}
%  {\endtabular\right.$}

\begin{document}
\title%[Hermitian u-invariants over semi-global fields]
{Three lectures on sheaf cohomology}
\author{Zhengyao Wu}
\date{3,10,17 November 2016} %no date
\address{
Department of Mathematics\\
Shantou University\\
243 Daxue Road\\
Shantou, Guangdong, China 515063}
\email{wuzhengyao@stu.edu.cn}
%\subjclass[2010]{Primary: 11E39. Secondary: 14H05, 16W10}
%\keywords{hermitian form, u-invariant, p-adic curve}
\begin{abstract}
References: \cite{Hartshorne} and \cite{Gro57}. 
\end{abstract}
\maketitle
\tableofcontents
%不证明的内容或来自其他同学/老师的报告,或给出文献。

\section{Nov 3, Definition of sheaf cohomology}\label{1}

\begin{defn}\label{1.1}
	Let $ A $ be a commutative ring. 
	An $ A $-module $ L $ is \textbf{injective} if it satisfies the following equivalent conditions:
	
	(1) $ \Hom(\bullet,L) $ is right exact. 
	
	(2) For all injective homomorphism $j:X'\to X $ and for all homomorphism $ f': X'\to L $, there exists a homomorphism $ f :X\to L$ such that $ f\circ j=f' $. 
	
	(3) For all ideals $ I $ of $ A $ and for all homomorphism $ f: I\to L $, there exists $ u\in L $ such that $ f(a)=a u $ for all $ a\in I $. 
\end{defn}

\begin{proof}
	We only prove (3) $ \implies $ (2). 
	
	Let $j:X'\hookrightarrow X $ be an injective homomorphism and let $ f': X'\to L $ be a homomorphism. 
	Define the set $ S $ of pairs $ (Y,g) $ of $ A$-module $ Y $ with a homomorphism $ g: Y\to L $ such that $j(X')\subset Y\subset X$ and $(g|_{j(X')})\circ j=f'$. 
	 $ S $ is partially ordered by defining $ (Y,g)\le(Y',g') $ if $Y\subset Y'$ and $g'|_Y=g$.  
	Then $ S\ne\emptyset $ because $ (j(X'),f)\in S $. 
	Also any totally ordered subset $ (Y_{\alpha}, g_{\alpha}) $ of $ S $ has an upper bound $ (\bigcup\limits_{\alpha}Y_{\alpha}, G) $ with $ G|_{Y_{\alpha}}=g_{\alpha} $. 
	By Zorn's lemma, $ S $ has a maximal element $ (Y_0,g_0) $. 
	
	Next, we want to show that $ Y_0=X $. 
	If $ Y_0 \subsetneq X$, take $ x\in X'\setminus Y_0 $. 
	Define an ideal $ I=\{a\in A~|~ax\in Y_0\} $ of $ A $ and a homomorphism $ h: I\to L $, $ h(a)=g_0(ax) $. 
	By (3), there exists $ u\in L $ such that $ h(a)=g_0(ax)=au $ for all $ a\in I $. 
	Let $ Y_1=Y_0+Ax $ and $ g_1: Y_1\to L $, $ g_1(y+ax)=g_0(y)+au $ for all $ y\in Y_0 $, $ a\in A $. 
	Then $ (Y_0,g_0)<(Y_1,g_1) $ and $ (Y_1,g_1)\in S $, which contradicts the maximality of $ (Y_0,g_0) $. 
	Therefore $ Y_0=X $ and we take $ f=g_0 $. 
\end{proof}

\begin{eg}\label{1.2}
	 $ \mathbb Q/\mathbb Z $ is an injective abelian group. 
\end{eg}

\begin{proof}
	Abelian groups are $ \mathbb Z $-modules. 
	We verify \cref{1.1}(3). 
	For zero ideal it is trivial. 
	Let $(n)$ be an nonzero ideal of $ \mathbb Z $. 
	Let $ f:(n)\to \mathbb Q/\mathbb Z $ be a $ \mathbb Z $-homomorphism. 
	Suppose $ f(n)=q+\mathbb Z $ for some $ q\in\mathbb Q\cap[0,1) $. 
	Let $ u=\frac{1}{n}q+\mathbb Z $. 
	Then for all $ m\in\mathbb Z $, $ mn\in I $ and $ f(mn)=mq+\mathbb Z=mnu $. 
\end{proof}

Let $ L $ be an $ A $-module. 
Let $ \widehat{L}=\Hom_{\mathbb Z}(L,\mathbb Q/\mathbb Z) $. 
Let $ \widehat{\widehat{L}}=\Hom_{\mathbb Z}(\widehat{L},\mathbb Q/\mathbb Z) $. 
Then we have a homomorphism $ i_L: L\to \widehat{\widehat{L}} $ defined as follows: For all $ x\in L $ and $ f\in\widehat{L} $, $ (i_L(x))(f)=f(x) $.  

\begin{lem}\label{1.3}
	$ i_L :L\to \widehat{\widehat{L}}$ is an injective homomorphism. 
\end{lem}

\begin{proof}
	Suppose $ x\in L $ and $ x\ne 0 $. 
	Let $ \langle x\rangle $ be the cyclic abelian group generated by $ x $. 
	Then $ \langle x\rangle $ is a subgroup of $ L $. 
	When $ x $ has order $ n $, we define $g:\langle x\rangle\to \mathbb Q/\mathbb Z  $, $ g(x)=\frac{1}{n}+\mathbb Z $. 
	When $ x $ has infinite order, we define $g:\langle x\rangle\to \mathbb Q/\mathbb Z  $, $ g(x)=\frac{1}{2}+\mathbb Z $. 
	%Thus we have a nonzero homomorphism $f:\langle x\rangle\to \mathbb Q/\mathbb Z  $. 
	
	By \cref{1.2}, $ \mathbb Q/\mathbb Z $ is injective, there exists a homomorphism $ f:L\to \mathbb Q/\mathbb Z $ such that $ f|_{\langle x\rangle}=g $ and $ f(x)=g(x)\ne 0 $. %Then $ g $ is nonzero. 
	%If $ \iota(x)=0 $, i.e.~$ f(x)=0 $ for all $ f\in \widehat{L} $, then $ x=0 $. 
\end{proof}

\begin{lem}\label{1.4}
	If $ L $ is projective, then $ \widehat{L} $ is injective. 
\end{lem}

\begin{proof}
	Let $j: X'\to X $ be an injective homomorphism. 
	Let $ f: X'\to \widehat{L} $ be a homomorphism. 
	Since $ \mathbb Q/\mathbb Z $ is injective, by \cref{1.1}(1), $ \widehat{j}: \widehat{X}\to \widehat{X'} $ defined by $ \widehat{j}(g)=g\circ j $ is surjective. 
	We also have $ \widehat{f}:\widehat{\widehat{L}}\to \widehat{X'} $. 
	Since $ \widehat{L} $ is projective, there exists $ h: L\to \widehat{X} $ such that $ \widehat{j}\circ h =\widehat{f}\circ i_L$. 
	\[
	\xymatrix{
		& L\ar[d]^{i_L}\ar[ldd]_{\exists h} &\\
		& \widehat{\widehat{L}}\ar[d]^{\widehat{f}} &\\
		\widehat{X}\ar[r]^{\widehat{j}}&\widehat{X'}\ar[r]&0\\
		}
	\]
	Then we have $\widehat{h}:\widehat{\widehat{X}}\to \widehat{L} $. 
	By \cref{1.3}, $ i_X:X\to \widehat{\widehat{X}} $ is an injective homomorphism. 
	Then for all $ x'\in X' $ and $ l\in L $, we have 
	\[((\widehat{h}\circ i_X\circ j)(x'))(l)=(i_X(j(x')))(h(l))=(h(l))(j(x'))=((\widehat{j}\circ h)(l))(x')\]\[=((\widehat{f}\circ i)(l))(x')=(\iota(l))(f(x'))=f(x')(l)\]
	Therefore $ \widehat{h}\circ i_X\circ j=f $ and hence $ \widehat{L} $ is injective. 
	\[
	\xymatrix{
		0\ar[r]& X'\ar[rr]^{j}\ar[dd]^f&&X\ar[ld]^{i_X}\\
		&&\widehat{\widehat{X}}\ar[ld]^{\widehat{h}}&\\
		&\widehat{L}&\\
		}
	\]
\end{proof}

\begin{prop}\label{1.5}
	Let $ A $ be a commutative ring. 
	Any $ A $-module is isomorphic to a submodule of an injective $ A $-module. 
\end{prop}

\begin{proof}
	Let $ L $ be an $ A $-module. 
	There exists a projective module $ P $ with a surjection $ \pi:P\to \widehat{L} $. 
	Then, $ \widehat{\pi}: \widehat{\widehat{L}}\to \widehat{P}$ is an injection. 
	By \cref{1.3} we have an injection $ \widehat{\pi}\circ i:L\to \widehat{F} $. 
	By \cref{1.4}, $ \widehat{F} $ is an injective $ A $-module. 
\end{proof}

Let $ X $ be a topological space. 

\begin{fact}\label{1.6}
	Let $ F,G,H $ be sheaves of abelian groups on $ X $. 
	Then $ F\xrightarrow{a} G\xrightarrow{b} H$
	is exact iff 
	$ F_x\xrightarrow{a_x} G_x\xrightarrow{b_x} H_x$
	is exact for all $ x\in X $. 
\end{fact}

\begin{proof}
	Omit. 
\end{proof}
%
%\begin{proof}
%	(1) Suppose $ \mathscr F\xrightarrow{i} G\xrightarrow{q} H $ is exact. 
%	
%%	\underline{$ i_x $ is an injection. }
%%	In fact, if $ a\in \mathscr F_x $ such that $ i_x(a) =0\in G_x$, there exists a neighborhood $ U $ of $ x $ and $ s\in \mathscr F(U) $ such that $ s_x=a $. 
%%	Then $ i_U(s)\in G(U) $ and $ i_U(s)_x=i_x(s_x)=i_x(a)=0 $. 
%%	Then there exists an open neighborhood $ V\subset U $ of $ x $ such that $ i_V(s|_V)=i_U(s)|_V=0 $. 
%%	Since $ i_V:\mathscr F(V)\to G(V) $ is injective, we have $ s|_V=0 $. 
%%	Therefore $ a=s_x=(s|_V)_x=0 $. 
%	
%	\underline{$ \im(i_x)\subset \ker(q_x) $. }
%	In fact, let $ a\in \mathscr F_x $, we want to show $ q_x(i_x(a))=0 $. 
%	There exists a neighborhood $ U $ of $ x $ and $ s\in \mathscr F(U) $ such that $ s_x=a $. 
%	Then $ q_U(i_U(s))=0 $ and hence $ q_x(i_x(a))=q_x(i_x(s_x))= q_U(i_U(s))_x=0$. 
%	
%	\underline{$ \im(i_x)\supset \ker(q_x) $. }
%	In fact, suppose $ a\in G_x $ such that $ q_x(a)=0 $
%\end{proof}

Let $ \mathcal{O}_X $ be a sheaf of rings on $ X $. 
Let $ \Mod(\mathcal{O}_X) $ be the category of sheaves of $ \mathcal{O}_X $-modules. 

\begin{thm}\label{1.7}
	$ \Mod(\mathcal{O}_X) $ has enough injectives.
\end{thm}

\begin{proof}
	Let $ \mathscr F $ be a sheaf of $ \mathcal O_X $-module. 
	
	\underline{(1) Construct a sheaf $ I $. }
	Then for all $ x\in X $, $ \mathscr F_x $ is a $ \mathcal O_{X,x} $-module. 
	By \cref{1.5}, there exists an injective $ \mathcal O_{X,x} $-module $ I $ with an injection $ i_x:\mathscr F_x\to I_x $. 
	Then $ I_x $ is a sheaf on $ \{x\} $. 
	Suppose $ j_x:\{x\}\to X $ is the inclusion map. 
	Then $ (j_x)_*I_x $ is a sheaf of $ \mathcal O_X $-module. 
	Let $ I=\prod\limits_{x\in X}(j_x)_*I_x $. 
	Then $ I $ is a sheaf of $ \mathcal O_X $-module. 
	
	\underline{(2) Construct an injection $ \iota: \mathscr F\to I $. }
	Since $$\Hom_{\Mod(\mathcal{O}_X)}(F, I)=\Hom_{\Mod(\mathcal{O}_X)}(F, \prod\limits_{x\in X}(j_x)_*I_x)$$ $$\simeq \prod\limits_{x\in X}\Hom_{\Mod(\mathcal{O}_X)}(F, (j_x)_*I_x)\simeq \prod\limits_{x\in X}\Hom_{\mathcal O_{X,x}}(\mathscr F_x, I_x)$$
	There exists a morphism $ \iota:\mathscr F\to I $ whose morphism on stalks are $ (i_x)_{x\in X} $. 
	Since $ i_x $ are all injections, by \cref{1.6}, $ i $ is an injection. 
	
%	\underline{(3) Show that $ i $ is an injection. }
%	Since $ i_x $ are all injective, $ i_U:\mathscr F(U)\to I(U) $ is injective for all open set $ U $ of $ X $. 
%	In fact, if $ f\in \mathscr F(U) $ such that $ i_U(f)=0 $, then $ i_x(f_x)=i_U(f)_x=0 $ for all $ x\in U $. 
%	Since $ i_x $ is injective, we have $ f_x=0 $ for all $ x\in U $. 
%	Then there exist an open neighborhood $ U_x\subset U $ of $ x $ such that $ f|_{U_x}=0 $, where $ (U_x)_{x\in U} $ form an open covering of $ U_x $. 
%	Since $ \mathscr F $ is a sheaf, we have $ f=0 $. 
%	Therefore $ i_U $ is injective and hence $ \iota:\mathscr F\to I $ is an injection. 
	
	\underline{(3) Show that $ I $ is an injective sheaf. }
	$ \Hom_{\Mod(\mathcal{O}_X)}(\bullet, I) $ is formed by three kinds of functors $ \bullet_x $, $ \Hom_{\mathcal O_{X,x}}(\bullet, I_x) $ and $ \prod\limits_{x\in X}\bullet $. 
	By \cref{1.6}, $ \bullet_x $ is exact for all $ x\in X $. 
	Since $ I_x $ is injective, by \cref{1.1}, $ \Hom_{\mathcal O_{X,x}}(\bullet, I_x) $ is exact for all $ x\in X $. 
	Also, the product $ \prod\limits_{x\in X}\bullet  $ of exact functors is exact. 
	Therefore $ \Hom_{\Mod(\mathcal{O}_X)}(\bullet, I) $ is exact. 
	By \cref{1.1}, $ I $ is injective. 
\end{proof}

Let $ \Ab $ be the category of abelian groups. 
Let $ \Ab(X) $ be the category of sheaves of abelian groups on $ X $. 

\begin{cor}\label{1.8}
	$ \Ab(X) $ has enough injectives.
\end{cor}

\begin{proof}
	Let $ \mathcal O_X $ be the locally constant sheaf of rings $ \underline{\mathbb Z} $. 
	Then the result follows from $ \Ab(X)=\Mod(\underline{\mathbb Z}) $ and \cref{1.7}. 
\end{proof}

\begin{lem}\label{1.9}
	The \textbf{global section} functor $ \Gamma(X, \bullet):\Ab(X)\to\Ab $ such that $ \Gamma(X,\mathscr F)=\mathscr F(X)$ is left exact. 
\end{lem}

\begin{proof}
	Suppose $ 0\to \mathscr F\xrightarrow{i} G\xrightarrow{q} H\to 0 $ is exact. 
	We want to show that $$ 0\to \mathscr F(X)\xrightarrow{i_X} G(X)\xrightarrow{q_X} H(X) $$ is exact.
	
	\underline{$ i_X$ is injective. }
	In fact, suppose $ s\in \mathscr F(X) $ such that $ i_X(s)=0 $. 
	Then $ i_x(s_x)=(i_X(s))_x=0 $ for all $ x\in X $. 
	By \cref{1.6}, $ s_x=0 $ for all $ x\in X $. 
	For each $ x\in X $, there exists an open neighborhood $ U_x $ of $ x $ such that $ s|_{U_x}=0 $. 
	Since $ (U_x)_{x\in X} $ is an open covering of $ X $ and $ \mathscr F $ is a sheaf, $ s=0 $. 
	
	\underline{$ \ker(q_X)\supset\im(i_X) $. }
	In fact, suppose $ s\in \mathscr F(X) $, let $ t=q_X(i_X(s))\in H(X)$. 
	Then $ t_x=q_x(i_x(s_x))=0$. 
	For each $ x\in X $, there exists an open neighborhood $ U_x $ of $ x $ such that $ t|_{U_x}=0 $. 
	Since $ (U_x)_{x\in X} $ is an open covering of $ X $ and $ H $ is a sheaf, $ t=0 $.  
	
	\underline{$ \ker(q_X)\subset\im(i_X) $. }
	In fact, suppose $ s\in G(X) $ such that $ q_X(s)=0 $. 
	Then $ q_x(s_x)=(q_X(s))_x=0 $ for all $ x\in X $. 
	By \cref{1.6}, $ s_x=i_x(t_x) $ for some $ t_x\in \mathscr F_x $ for all $ x\in X $. 
	Then there exists an open neighborhood $ U_x $ of $ x $ and $ t_{U_x}\in \mathscr F(U_x) $ such that $ s|_{U_x}=i_{U_x}(t_{U_x}) $. 
	If $ U_x\cap U_y\ne\emptyset $, then $ i_{U_x\cap U_y}(t_{U_x}|_{U_x\cap U_y}-t_{U_x}|_{U_x\cap U_y})=s|_{U_x\cap U_y}-s|_{U_x\cap U_y}=0 $. 
	Since $ i $ is injective, $ t_{U_x}|_{U_x\cap U_y}=t_{U_x}|_{U_x\cap U_y} $. 
	Since $ \mathscr F $ is a sheaf and $ (U_x)_{x\in X} $ is a covering of $ X $, there exists $ t\in \mathscr F(X) $ such that $ t|_{U_x}=t_{U_x} $. 
	Since $ (i_X(t))|_{U_x}=i_{U_x}(t|_{U_x})=i_{U_x}(t_{U_x})=s|_{U_x} $ for all $ x\in X $, $G $ is a sheaf and $ (U_x)_{x\in X} $ is a covering of $ X $, we have $ i_X(t)=s $. 	
\end{proof}

\begin{eg}\label{1.10}
	$ \Gamma(X,\bullet):\Ab(X)\to \Ab $ is not necessarily right exact. 
	
	Let $ X=\mathbb C^*$ with analytic topology. 
	Let $ \underline{\mathbb Z} $ be the locally constant sheaf associated to $ \mathbb Z $. 
	Let $ \mathcal O $ be the sheaf of holomorphic functions. 
	Let $ \mathcal O^* $ be the sheaf of invertible holomorphic functions. 
	Then we have an exact sequence 
	\[0\to \underline{\mathbb Z}\xrightarrow{\bullet 2\pi\sqrt{-1}}\mathcal O\xrightarrow{\exp}\mathcal O^*\to 1.\]
	For all $ w\in\mathbb C $, $ \underline{\mathbb Z}_w=\mathbb Z $, $ \mathcal O_w= \{f:U_w\to \mathbb C~|~\exists f'(z),\forall z\in U_w\}$ and $ \mathcal O^*_w= \{f:U_w\to \mathbb C~|~\exists f'(z), \forall z\in U_w, f(w)\ne 0, \}$, where $ U_w $ is some open neighborhood of $ w $. 
	\[0\to \mathbb Z\xrightarrow{\bullet 2\pi\sqrt{-1}}\mathcal O_w\xrightarrow{\exp}\mathcal O^*_w\to 0\]
	is exact at $ \mathbb Z $ and $ \mathcal O_w $ because $ e^{2\pi \sqrt{-1}}=0 $. 
	And $ \exp $ is surjective because let $ g(z)=\ln(f(z)) $ on a neighborhood of $ w $ such that $ \ln $ is a well-defined logarithm function on a band neighborhood of $ f(w) $ of width $ 2\pi $. 
	Then $ g'(w)=\dfrac{f'(w)}{f(w)} $ exists and hence $ g $ is a inverse image of $ f $. 
	
	However, $ \exp:\mathcal O(\mathbb C^*)\to \mathcal O^*(\mathbb C^*) $ is not surjective because $ \Id_{\mathbb C^*}\in \mathcal O^*(\mathbb C^*) $ does not have an inverse image. 
	If not, there exists $ f\in \mathcal O(\mathbb C^*) $ such that $ e^f=\Id_{\mathbb C^*} $, then $$ f|_{\mathbb C^*\setminus(-\infty,0)}(re^{\theta\sqrt{-1}})=\ln(r)+\theta\sqrt{-1}+2n\pi\sqrt{-1} $$ for some $ n\in\mathbb Z $, where $ r>0 $ and $ -\pi< \theta<\pi $; $$ f|_{\mathbb C^*\setminus(0,+\infty)}(re^{\theta'\sqrt{-1}})=\ln(r)+\theta'\sqrt{-1}+2m\pi\sqrt{-1} $$ for some $ n\in\mathbb Z $, where $ r>0 $ and $ 0< \theta'<2\pi $. 
	
	If $0<\theta=\theta'<\pi $, then $ n=m $. 
	If $-\pi<\theta'=\theta+2\pi<0 $, then $ n=m-1 $. 
	Contradiction.% to the fact that $ \mathcal O $ is a sheaf. 
\end{eg}

\begin{defn}\label{1.11}
	Let $ X $ be a topological space (By \cref{1.8}, it has enough injectives). 
	Let $ \Gamma(X,\bullet):\Ab(X)\to \Ab $ be the global section functor (By \cref{1.9} it is left exact). 
	Define $$ H^i(X, \bullet)=R^i\Gamma(X,\bullet):\Ab(X)\to \Ab ,~ i=0,1,\ldots $$ be the sequence of right derived functors of $ \Gamma(X,\bullet) $. 
	Let $ \mathscr F $ be a sheaf of abelian groups on $ X $. 
	Then $ H^i(X, \mathscr F) $ is called the $ i $-th \textbf{sheaf cohomology} of $ \mathscr F $. 
\end{defn}

\begin{rem}
 

	(1) \cref{1.11} provides the first way to calculate sheaf cohomology. 
	Let $ \mathscr F $ be a sheaf of $ \mathcal O_X $-module.
	Let $ 0\to \mathscr F\to I^0\to I^1\to \cdots $ be a injective resolution of $ \mathscr F $. 
	Apply $ \Gamma(X, \bullet) $ to the resolution, we obtain a complex 
	\[0\to \Gamma(X, I^0)\to \Gamma(X, I^1)\to \cdots\]
	The usual $ i $-th cohomology of this complex is $ h^i(\Gamma(X,I^{\bullet}))\simeq H^i(X, \mathscr F) $. 
	
	(2) The disadvantage of (1) is that: Despite that there are enough injectives, there are still ``too few''. 
	That is why we need flasque sheaves. 
\end{rem}

\begin{defn}\label{2.1}
	Let $ X $ be a topological space. 
	Let $ \mathscr F $ be a sheaf of abelian groups on $ X $. 
	We say that $ \mathscr F $ is \textbf{flasque} if for all inclusion of open sets $ V\subset U $ in $ X $, the restriction $ \mathscr F(U)\to \mathscr F(V) $ is surjective. 
\end{defn}

\begin{fact}\label{2.2}
	Let $ U $ be an open set of $ X $. 
	Let $ j: U\to X $  be the inclusion map. 
	Let $ j_!(\mathscr F) $ be the sheaf associated to the presheaf 
	$$
	V\mapsto \left\{
	\begin{array}{ll}
	\mathscr F(V),&V\subset U;\\
	0,&V\not\subset U\\
	\end{array}
	\right.
	$$
	
	Then for all $ x\in X $ 
	\[
	j_!(\mathscr F)_x=\left\{
	\begin{array}{ll}
	\mathscr F_x,&x\in U;\\
	0,&x\not\in U.\\
	\end{array}
	\right.
	\]
\end{fact}

\begin{proof}
	Omit. 
%	Suppose $ x\in U $. 
%	If $ a\in j_!(\mathscr F)_x $, then there exists an open neighborhood $ V $ of $ x $ and $ s\in j_!(\mathscr F)(V) $ such that $ s_x=a $. 
%	Since $ x\in V\cap U $, $ s|_{V\cap U}\in j_!(\mathscr F)(V\cap U)=\mathscr F(V\cap U) $ and $(s|_{V\cap U})_x=s_x=a $, we have $ a\in \mathscr F_x $. 
%	
%	Conversely, if $ b\in \mathscr F_x $, then there exists an open neighborhood $ W $ of $ x $ and $ t\in \mathscr F(W) $ such that $ t_x=b $. 
%	Since $ x\in W\cap U $, $ t|_{W\cap U}\in \mathscr F(W\cap U)=j_!(\mathscr F)(W\cap U) $ and $ (t|_{W\cap U})_x=t_x=b $, we have $ b\in j_!(\mathscr F)_x $. 
%	
%	Now suppose $ x\not\in U $. 
%	If $ a\in j_!(\mathscr F)_x $, then there exists an open neighborhood $ V $ of $ x $ and $ s\in j_!(\mathscr F)(V) $ such that $ s_x=a $. 
%	Since $ x\in V\setminus U $, we have $ V\not\subset U $. 
%	Then $ j_!(\mathscr F)(V)=0 $ and hence $ s=0 $, $ a=s_x=0 $. 
\end{proof}

\begin{lem}\label{2.3}
	Let $ \mathcal O_X $ be a sheaf of rings on $ X $. 
	Let $ U $ be an open set of $ X $ with inclusion $ j: U\to X $. 
	Write $ \mathcal O_U=j_!(\mathcal O_X|U) $. 
	Then for all sheaf of $ \mathcal O_X $-modules $ G $, $$ \Hom_{\Mod(\mathcal{O}_X)}(\mathcal O_U, G)\simeq G(U) .$$ 
\end{lem}

\begin{proof}
\color{red}{Define $ \alpha:  \Hom_{\Mod(\mathcal{O}_X)}(\mathcal O_U, G)\to G(U) $. 
For all natural transformation $ f:\mathcal O_U\to G $ and for all open subset $ V\subset U $, there exists $ f_V:\mathcal O_U(V)=\mathcal O_X(V)\to G(V) $ commuting with restrictions. 
In particular, we have $ f_U: \mathcal O_X(U)\to G(U) $, define $ \alpha(f)=f_U(1) $. 

Define $ \beta:  G(U)\to \Hom_{\Mod(\mathcal{O}_X)}(\mathcal O_U, G) $. 
For all $ g\in G(U) $ and for all open subset $ V\subset U $, $ g|V\in G(V) $. 
Define $ \beta(g): \mathcal O_U\to G $ such that $ \beta(g)_V(x)=x\cdot g|V $ for all $ x\in \mathcal O_U(V)=\mathcal O_X(V) $. 

For all $ f\in \Hom_{\Mod(\mathcal{O}_X)}(\mathcal O_U, G) $, $ V$ open in $ U $ and $ x\in \mathcal O_U(V) $, 
$$ \beta(\alpha(f))_V(x)=x\cdot (\alpha(f)|V)=x\cdot f_U(1)|V=x\cdot f_V(1)=f_V(x),$$
Hence $ \beta\circ\alpha=\Id $. 

Conversely, for all $ g\in G(U) $, 
\[\alpha(\beta(g))=\beta(g)_U(1)=1\cdot g|U=g,\]
Hence $ \alpha\circ\beta=\Id $. 

This is basically like the proof of Yoneda lemma.}
\end{proof}

The next lemma shows that there are ``more'' flasque sheaves than injective sheaves. 

\begin{lem}\label{2.4}
	Let $ \mathcal O_X $ be a sheaf of rings on $ X $. 
	Any injective $ \mathcal O_X $-module is flasque. 
\end{lem}

\begin{proof}
	Suppose $ V\subset U $ is an inclusion of open sets in $ X $. 
	By \cref{2.2}, the inclusion gives the canonical injection $ \mathcal O_V \to \mathcal O_U$. 
	Let $ I $ be an injective sheaf of $ \mathcal O_X $-module. 
	By \cref{1.1}, there exists a canonical surjection $$ \Hom_{\Mod(\mathcal{O}_X)}(\mathcal O_U, I)\to \Hom_{\Mod(\mathcal{O}_X)}(\mathcal O_V, I) $$ 
	By \cref{2.3}, this surjection is identified with the restriction $ I(U)\to I(V) $. 
\end{proof}

%\begin{prop}
%	For all sheaf of abelian groups $ \mathscr F\in\Ab(X) $ and open set $ U\subset X $, 
%	\[\mathscr F(U)=\prod\limits_{x\in U}\mathscr F_x.\]
%\end{prop}

%\begin{ww}
%	Let $ \widetilde{\mathscr F} =\bigsqcup\limits_{x\in X}\mathscr F_x$ be the disjoint union. 
%	Define $ p: \widetilde{\mathscr F}\to X$ such that $ p(y)=x $ if $ y\in \mathscr F_x$. 
%	Let $ \widetilde{\mathscr F}(U)=\{q: U\to \widetilde{\mathscr F}~|~p\circ q=\Id_U\} $. 
%	For 
%	
%	For each $ s\in \mathscr F(U) $, there exists a map $ \widetilde{s}:U\to \widetilde{\mathscr F}$ such that $ \widetilde{s}(x)=s_x\in \mathscr F_x $ for all $ x\in U $. We have $ p(\widetilde{s}(x))=p(s_x)=x $, i.e.~$ \widetilde{s}\in \widetilde{\mathscr F}(U) $. 
%	
%	
%	We have defined a 
%	
%	
%	To summarize, we have the following dictionary:
%	
%	\begin{tabular}{|c|c|c|}
%		\hline
%		Sheaves & Sections & Abelian Groups\\
%		\hline
%		&&\\
%		$ \mathscr F $ & $ s $ & $ \mathscr F(U) $\\
%		\hline
%		&&\\
%		$ \widetilde{\mathscr F} $ & $ \widetilde{s} $ & $ \widetilde{\mathscr F}(U) $\\
%		\hline
%	\end{tabular}
%\end{ww}
%
%See also \cite[p.67,~1.16(e)]{Hartshorne} for our last discussion today. 
\newpage
\section{Nov 10, Sheaf cohomology and flasque resolution}\label{2}

Review: Let $ X $ be a topological space. 
Let $ \mathcal O_X $ be a sheaf of rings on $ X $. 

\begin{enumerate}
	\item $ \Mod(\mathcal O_X) $ has enough injectives. e.g.~$ \Ab(X) $. 
	\item $ \Gamma(X, \bullet): \Ab(X)\to \Ab $ is left exact. e.g.~$ 0\to \underline{\mathbb Z}\to \mathcal O\to \mathcal O^*\to 1 $ on $ \mathbb C^* $. 
	\item $ H^n(X,\bullet)=R^n\Gamma(X, \bullet) $. e.g.~$ H^0(X, \mathscr F)=\Gamma(X, \mathscr F) $. 
	\item  A sheaf $ \mathscr F $ is flasque if the restriction $ \mathscr F(U)\to \mathscr F(V) $ is surjective for all inclusions of open sets $ V\subset U $. e.g.~injectives sheaves of $ \mathcal O_X $-modules. 
\end{enumerate}

\begin{defn}\label{2.8}
	A sheaf of $ \mathcal O_X $-module  $ \mathscr F $ is \textbf{acyclic} if $ H^n(X, \mathscr F)=0 $ for all $ n\ge 1 $. 
\end{defn}

\begin{eg}\label{eg-inj}
	Let $ I $ be an injective sheaf of $ \mathcal O_X $-modules. 
	Then $ 0\to I\to I\to 0 $ is an injective resolution of $ I $. 
	We obtain a complex $ 0\to \Gamma(X,I)\to 0 $. 
	Hence $ H^n(X, I)=0 $ for all $ i\ge 1 $. Hence, {injective sheaves of $ \mathcal O_X $-modules are acyclic. }
\end{eg}

Next we want to show that flasque sheaves of $ \mathcal O_X $-modules are acyclic. 
Fix the following notations for \cref{G1}, \cref{G2} and \cref{2.5}. 
Let $ \mathscr F $ be a flasque sheaf of $ \mathcal O_X $-module
By \cref{1.7}, there exists an injective sheaf of $ \mathcal O_X $-module $ I $ with an injection $ i: \mathscr F\to I $. 
Let $ G =\cok( i) $. 
Since $ \Mod(\mathcal{O}_X) $ is an abelian category, $ G $ is a sheaf. 
Then we have an exact sequence \[0\to \mathscr F\xrightarrow{i} I\xrightarrow{q} G\to 0\]

\begin{lem}\label{G1}
	 \[0\to \mathscr F(U)\xrightarrow{i_U} I(U)\xrightarrow{q_U} G(U)\to 0\] is exact for all open set $ U $ of $ X $.
\end{lem}

\begin{proof}
	By \cref{1.9}, it suffices to show that $ q_U $ is a surjection. 
	Suppose $ s\in G(U) $. 
	Let $ T=\{(V,t)~|~V\subset U,~t\in I(V),~q_V(t)=s|_V\} $. 
	Define $ (V,t)\le (V't') $ iff $ V\subset V' $ and $ t'|_V=t $. 
	First, $ T\ne \emptyset $ because $ (\emptyset, 0)\in T $. 
	Second, any totally ordered subset $ (V_{\alpha}, t_{\alpha}) $ has an upper bound $ (\bigcup\limits_{\alpha}V_{\alpha}, t) $ with $ t|_{V_{\alpha}} =t_{\alpha}$ as $ I $ is a sheaf. 
	By Zorn's lemma, $ T $ has a maximal element $ (V_0, t_0) $. 
	
	We want to show that $ V_0=U $ and hence $ t_0 $ is an inverse image of $ s $. 
	If not, then there exists $ x\in U\setminus V_0 $. 
	Since $ q_x: I_x\to G_x $ is surjective, there exists an open neighborhood $ W $ of $ x $ and $ t'\in I(W) $ such that $ q_W(t')=s|_W $. 
	Since $ q_{W\cap V_0}(t_0|_{W\cap V_0}-t'|_{W\cap V_0})=s|_{W\cap V_0}-s|_{W\cap V_0}=0 $, there exists $ r'\in \mathscr F(W\cap V_0) $ such that $ i_{W\cap V_0}(r')=t_0|_{W\cap V_0}-t'|_{W\cap V_0} $. 
	Since $ \mathscr F $ is flasque, the restriction $ \mathscr F(W)\to \mathscr F(W\cap V_0) $ is surjective. 
	Then there exists $ r\in \mathscr F(W) $ such that $ r|_{W\cap V_0}=r' $. 
	Let $ t''=t'+i_W(r) $.
	Then $ t''|_{W\cap V_0}=t'|_{W\cap V_0}+i_W(r)|_{W\cap V_0}=t_0|_{W\cap V_0} $. 
	Since $ I $ is a sheaf, $ t_0 $ and $ t'' $ are glued to $ \widetilde{t}\in I(W\cup V_0) $. 
	We have $ (V_0,t_0)<(W\cup V_0, \widetilde{t})\in T $, a contradiction. 
	Therefore $ V_0=U $ and $ q_U(t)=s $. 
\end{proof}

\begin{lem}\label{G2}
	$ G $ is flasque. 
\end{lem}

\begin{proof}
	Suppose $ V\subset U $ is an inclusion of open sets in $ X $. 
	By \cref{G1}, we have a commutative diagram with exact rows:
	\[
	\xymatrix{
		0\ar[r]& \mathscr F(U)\ar[r]\ar[d]^{a} & I(U)\ar[r]\ar[d]^{b} & G(U)\ar[r]\ar[d]^{c} & 0\\
		0\ar[r]& \mathscr F(V)\ar[r] & I(V)\ar[r] & G(V)\ar[r] & 0\\
	}
	\]
	where $ a,b,c $ are restrictions. 
	By Snake lemma, we have an exact sequence $$ \cdots\to\cok(b)\to \cok(c)\to 0. $$ 
	Since $ I $ is injective, by \cref{2.4}, it is flasque and hence $ \cok(b)=0 $. 
	Therefore $ \cok(c)=0 $, i.e.~$ G $ is flasque. 
\end{proof}

\begin{thm}\label{2.5}
	 Flasque sheaves of $ \mathcal O_X $-modules are acyclic. 
\end{thm}

\begin{proof}
	Let $ \mathscr F $ be a flasque sheaf of $ \mathcal O_X $-module. 
	We want to show that $ H^n(X, \mathscr F)=0 $ for all $ n\ge 1 $. 
	
	We use induction on $ n $. 
	Let $ I,G $ be as above. 
	Let $ I^0=I $. 
	Let $ I^1 $ be an injective $ \mathcal O_X $-module containing $ G $. 
	Then $ 0\to \mathscr F\to I^0\to I^1\to \cdots $ gives a complex $$ 0\to \Gamma(X, I)\xrightarrow{d^0}\Gamma(X, I^1)\xrightarrow{d^1}\cdots $$
	So $$ H^1(U, \mathscr F)=\dfrac{\ker(d^1)}{\im(d^0)}=\dfrac{G(U)}{\im(q_U)}=0 $$ for all open set $ U $ of $ X $. 
	In particular, $ H^1(X, \mathscr F)=0 $. 

	Suppose the $ n $-th cohomology vanishes for all flasque sheaves. 
	In particular, $ H^n(X, \mathscr F)=0 $. 
	We need to show that $ H^{n+1}(X, \mathscr F)=0 $. 
	The short exact sequence $ 0\to \mathscr F\to I\to G\to 0 $ gives a long exact sequence
	\[
	\cdots\to H^n(X, I)\to H^n(X, G)\to H^{n+1}(X, \mathscr F)\to H^{n+1}(X, I)\to \cdots
	\]
	Since $ I $ is injective, by \cref{eg-inj}, $ H^{n+1}(X,I)=0 $. 
	By \cref{G2}, $ G $ is flasque. 
	By inductive hypothesis, $ H^n(X, G)=0$. Therefore $H^{n+1}(X, \mathscr F)=0$. 
\end{proof}



\begin{fact}[Horseshoe]\label{2.6}
	Let $ C $ be an abelian category with enough injectives. 
	Let $ 0\to A_1\xrightarrow{i} A_2\xrightarrow{p} A_3\to 0 $ be an exact sequence in $ C $. 
	Let $ 0\to A_1\xrightarrow{a} I_1^0\xrightarrow{a^0} I_1^1\xrightarrow{a^1} \cdots $ and $ 0\to A_3\xrightarrow{c} I_3^0\xrightarrow{c^0} I_3^1\xrightarrow{c^1} \cdots $ be two injective resolutions. 
	Then there exists an injective resolution $ 0\to A_2\xrightarrow{b} I_2^0\xrightarrow{b^0} I_2^1\xrightarrow{b^1} \cdots $ and horizontal morphisms such that the following diagram is commutative with exact columns and \textbf{split-exact} rows. 
	\[
	\xymatrix{
		& 0\ar[d] & 0\ar[d] & 0\ar[d] & \\
		0\ar[r]& A_1\ar[d]^a\ar[r]^i & A_2\ar[d]^b\ar[r]^p%\ar[dl]^f\ar[dr]^g 
		& A_3\ar[d]^c\ar[r] & 0\\
		0\ar[r]& I_1^0\ar[d]^{a^0}\ar[r]^{i^0} & I_2^0\ar[d]^{b^0}\ar[r]^{p^0}%\ar[dl]^{f^0}\ar[dr]^{g^0} 
		& I_3^0\ar[d]^{c^0}\ar[r] & 0\\
		0\ar[r]& I_1^1\ar[d]^{a^1}\ar[r]^{i^1} & I_2^1\ar[d]^{b^1}\ar[r]^{p^1}%\ar[dl]^{f^1}\ar[dr]^{g^1} 
		& I_3^1\ar[d]^{c^1}\ar[r] & 0\\
		& \vdots & \vdots & \vdots & \\
		}
	\]
\end{fact}

\begin{proof}
	Omit.
\end{proof}

%\begin{proof}
%	For all $ j=0,1,\ldots $, take $ I_2^j=I_1^j\oplus I_3^j $ with the canonical injection $ i^j:I_1^j\to I_1^j\oplus I_3^j   $ and the canonical projection $ p^j:I_1^j\oplus I_3^j \to I_3^j  $ hence rows are split exact and $ I_2^j $ are injective. 
%	
%	Since $ I_1^0 $ is injective, there exists $ f:A\to I_1^0 $ such that $ f\circ i=a $. Let $ g: A_2\to I_3^0 $, $ g=c\circ p $. 
%	Let $ b=f\oplus g $. 
%	For $ j\ge 1 $, since $ I_1^j $ is injective, there exists $ f^j:I_2^j\to I_1^{j+1}$ such that $ f^j\circ i^j=a^{j} $. 
%	Let $ g^j: I_2^{j}\to I_3^{j+1} $, $ g^j=c^j\circ p^j $. 
%	Let $ b^j=f^j\oplus g^j $. 
%	Hence $ b^j $ are constructed and the diagram is commute.
%	
%	Finally we show that the middle column is exact. 
%	That $ b $ is injectiv follows from Snake lemma. 
%	That $ \ker(b^0)=\im(b) $ follows from $ \ker(f^0)=\im(f) $ and $ \ker(g^{0})=\im(g) $. 
%	That $ \ker(b^{j+1})=\im(b^j) $ follows from $ \ker(f^{j+1})=\im(f^j) $ and $ \ker(g^{j+1})=\im(g^j) $. 
%\end{proof}

\begin{lem}\label{2.7}
	Let $ C^{\bullet} $ be a complex in an abelian category with enough injectives. 
	Then there exists an injective resolution
	$$ 0\to C^{\bullet}\to I^{\bullet, 0}\to I^{\bullet, 1}\to \cdots $$
	such that 
	$$0\to  Z^p(C^{\bullet})\to  Z^p(I^{\bullet, 0})\to  Z^p(I^{\bullet, 1})\to \cdots, $$
	$$0\to  B^p(C^{\bullet})\to  B^p(I^{\bullet, 0})\to  B^p(I^{\bullet, 1})\to \cdots, $$
	$$0\to  h^p(C^{\bullet})\to  h^p(I^{\bullet, 0})\to  h^p(I^{\bullet, 1})\to \cdots $$
	are injective resolutions (i.e.~\textbf{Cartan-Eilenberg resolution} exists). 
\end{lem}

\begin{proof}
	We use $ Z $ for kernels, $ B $ for images and $ h $ for usual cohomologies.
	
	First, we select injective resolutions for $ B^p(C^{\bullet}) $ and $ h^p(C^{\bullet}) $ for all $ p $. 
	
	Since $ 0\to B^{p-1}(C^{\bullet})\to Z^p(C^{\bullet})\to h^p(C^{\bullet})\to 0 $, by \cref{2.6}, we obtain an injective resolution for $ Z^p(C^{\bullet}) $ for all $ p $. 
	
	Since $ 0\to Z^{p}(C^{\bullet})\to C^{p}\to B^p(C^{\bullet})\to 0 $, by \cref{2.6}, we obtain an injective resolution for $ C^p $ for all $ p $. 
\end{proof}


Acyclic sheaves of $ \mathcal O_X $-modules provides a second way to calculate sheaf cohomology. 
\begin{thm}[De Rham-Weil]\label{DW}
	Let $ \mathscr F $ be a sheaf of $ \mathcal O_X $-module.
	Let $ 0\to \mathscr F\to J^0\to J^1\to\cdots  $ be a acyclic (e.g.~flasque) resolution of $ \mathscr F $. 
	Apply $ \Gamma(X, \bullet) $ to the resolution, we obtain a complex 
	\[0\to \Gamma(X, J^0)\to \Gamma(X, J^1)\to\cdots \]
	The usual $ i $-th cohomology of this complex is $ h^i(\Gamma(X,J^{\bullet}))\simeq H^i(X, \mathscr F) $. 
	%(2) $ H^i(X,\mathscr F) $ are all $ \Gamma(X, \mathcal O_X) $-modules. 	
	%(3) The disadvantage of (1) is that: Despite that flasque sheaves are better than injective sheaves, they are still hard to compute.
\end{thm}

\begin{proof} 
	By \cref{2.7}, let the following be an injective resolution of $ 0\to \mathscr F\to J^{\bullet} $. 
	\[
	\xymatrix{
		&\vdots &\vdots &\vdots &\vdots &\\
		0\ar[r]& I^1 \ar[u]\ar[r]&I^{0,1} \ar[u]\ar[r]&I^{1,1} \ar[u]\ar[r]&I^{2,1} \ar[u]\ar[r]&\cdots\\
		0\ar[r]& I^0 \ar[u]\ar[r]&I^{0,0} \ar[u]\ar[r]&I^{1,0} \ar[u]\ar[r]&I^{2,0} \ar[u]\ar[r]&\cdots\\
		0\ar[r]& F \ar[u]\ar[r]&J^0 \ar[u]\ar[r]&J^1 \ar[u]\ar[r]&J^2 \ar[u]\ar[r]&\cdots\\
		& 0 \ar[u]&0\ar[u]&0 \ar[u]&0 \ar[u]&\\
		}
	\]
	By \cref{1.9}, $ \Gamma(X,\bullet) $ is left exact, apply it everywhere, and replace the left column and the bottom row with $ 0 $, then we have a double complex $ C^{p,q}=\Gamma(X, I^{p,q}) $ in $ \Ab $.
		
	\begin{equation}\label{double}
	\xymatrix{
		 &\vdots &\vdots &\vdots &\\
		 0\ar[r]&\Gamma(X,I^{0,1}) \ar[u]\ar[r]&\Gamma(X,I^{1,1}) \ar[u]\ar[r]&\Gamma(X,I^{2,1}) \ar[u]\ar[r]&\cdots\\
		 0\ar[r]&\Gamma(X,I^{0,0}) \ar[u]\ar[r]&\Gamma(X,I^{1,0}) \ar[u]\ar[r]&\Gamma(X,I^{2,0}) \ar[u]\ar[r]&\cdots\\
	&0\ar[u]&0 \ar[u]&0\ar[u]&\\
		%&0\ar[u]&0 \ar[u]&0 \ar[u]&\\
	}
	\end{equation}
	
	First, we take the vertical spectral sequence. 
	Let $ {_v}E_0^{p,q}=C^{p,q} $. 
	Since $ J^p $ are all acyclic, we have $$ {_v}E_1^{p,q}=\left\{
	\begin{array}{ll}
	\Gamma(X,J^p), &q=0;\\
	0,&q>0.\\
	\end{array}
	\right. $$
	Then $$ {_v}E_2^{p,q}=\left\{
	\begin{array}{ll}
	h^p(\Gamma(X,J^{\bullet})), &q=0;\\
	0,&q>0.\\
	\end{array}
	\right. $$
	Hence $ {_v}E_2^{p,q}=E_{\infty}^{p,q} $ and $ {_v}E_2^{p,q}\Rightarrow h^{p+q}(\Gamma(X,J^{\bullet}))$. 
	
	Next, we take the horizontal spectral sequence. 
	Let $ {_h}E_0^{p,q}=C^{q,p} $. 
	Since $ 0\to F\xrightarrow{d^{-1}} J^0\xrightarrow{d^0} J^1\to\cdots $ is exact, 
	we have $ Z^p=B^{p-1} $ for all $ p\ge 0 $. 
	Let $ 0\to Z^p\to Z^{p,0}\to Z^{p,1}\to\cdots $ be an injective resolution of $ Z^p $. 
	By the construction of \cref{2.7}, $ I^p=Z^{0,p} $, $ I^{q,p}=Z^{q,p}\oplus Z^{q+1,p} $ and the homomorphism $$ I^{q,p}=Z^{q,p}\oplus Z^{q+1,p}\to I^{q+1,p}=Z^{q+1,p}\oplus Z^{q+2,p},~(x,y)\mapsto (y,0) $$
	has kernel $ Z^{q,p} $ and image $ Z^{q+1,p} $ for all $ p,q\ge 0 $. 
	Hence $ h^q(I^{\bullet,p})=0 $ for all $ q\ge 1 $. 
	Then $ 0\to I^p\to I^{0,p}\to I^{1,p}\to\cdots $ is injective resolution of the injective $ I^p $. By \cref{eg-inj}, 
	\[
	{_h}E_1^{p,q}=h^q(\Gamma(X, I^{\bullet, p}))=\left\{
	\begin{array}{ll}
	\Gamma(X, I^{p}), & q=0;\\
	0,& q>0.\\
	\end{array}
	\right.
	\]
	Then $$ {_h}E_2^{p,q}=\left\{
	\begin{array}{ll}
	h^p(\Gamma(X, I^{\bullet}))=H^p(X,\mathscr F), &p=0;\\
	0,&p>0.\\
	\end{array}
	\right. $$
	Hence $ {_h}E_2^{p,q}={_h}E_{\infty}^{p,q} $ and $ {_h}E_2^{p,q}\Rightarrow H^{p+q}(X,\mathscr F)$. 
	
	Finally, since two spectral sequences of the same double complex have isomorphic abutment, we have $$ h^n(\Gamma(X, J^{\bullet}))\simeq H^n(X,\mathscr F) $$ for all $ n $. 
	
\end{proof}

\begin{eg}\label{eg1}
	If $ X $ is an irreducible topological space with a locally constant sheaf of abelian group $ \underline{A} $, then $ H^n(X, \underline{A}) =0$ for all $n\ge 1 $. 
\end{eg}

\begin{proof}
	Since $ X $ is irreducible, every nonempty open subset of $ X $ is connected. 
	For all inclusion of open subsets $ V\subset U $, $ \underline{A}(U)\to  \underline{A}(V)$ is $ \Id :A\to A$ or $ A\to 0 $ or $ 0\to 0 $. All three possibilities are surjective. 
	Then $ \underline{A} $  is flasque. 
	By \cref{2.5}, $ H^n(X, \underline{A})=0$ for all $ n\ge 1 $. 
\end{proof}

\begin{rem}	
	Sheaf cohomology is ``bad'', because by \cref{eg1}, it does not do anything even on constant sheaves. 
	We need a better one (\'etale cohomology). 
	
	Despite that there are ``more'' flasque sheaves and even ``more'' acyclic sheaves, sheaf cohomology is still hard to compute. 
	That is why we need \v{C}ech cohomology, which are easier to compute.
\end{rem}

\begin{defn}
	%Let $ X $ be a topological space. 
	Let $ \mathscr U=(U_i)_{i\in I} $ be an open covering of $ X $ whose index set $ I $ has a \textit{well-ordering}(i.e.~totally ordered such that every nonempty subset has a least element). 
%	(1) Let $ \mathscr P$ be a \textit{presheaf} of $ \mathcal O_X $-modules on $ X $. 
%	Let 
%	
%	$ C^0(\mathscr U, \mathscr P)=\prod\limits_{i\in I}\Gamma(U_i, \mathscr P) $, 
%	$ C^1(\mathscr U, \mathscr P)=\prod\limits_{i<j\text{ in }I}\Gamma(U_i\cap U_j, \mathscr P) $, \ldots, $$ C^n(\mathscr U, \mathscr P)=\prod\limits_{i_0<i_1<\cdots<i_n\text{ in } I}\Gamma(U_{i_0}\cap U_{i_1}\cap\cdots \cap U_{i_n}, \mathscr P) $$
%	be a complex with $$ d^0((f_i)_{i\in I})=(f_i|_{U_i\cap U_j}-f_j|_{U_i\cap U_j})_{i<j\text{ in }I}$$
%	$$ d^1((f_{ij})_{i<j\text{ in }I})=(f_{jk}|_{U_i\cap U_j\cap U_k}-f_{ik}|_{U_i\cap U_j\cap U_k}+f_{ij}|_{U_i\cap U_j\cap U_k})_{i<j<k\text{ in }I}$$
%	$$ d^n((f_{i_1,i_2\cdots, i_n})_{i_1<i_2<\cdots<i_n\text{ in }I})
%	=\left(\sum\limits_{j=0}^{n}(-1)^j
%	f_{i_0,\cdots,\widehat{i_j},\cdots i_n}
%	|_{U_{i_0}\cap\cdots\cap U_{i_n}}
%	\right)_{i_0<i_1<\cdots<i_n\text{ in }I}$$
%	where $ \widehat{i_j} $ means delete $ i_j $. 
%	We write $ \check{H}^n(\mathscr U, \mathscr P)=h^n(C^{\bullet}(\mathscr U, \mathscr P)) $.
%	
%	(2) 
	Let $ \mathscr F $ be a \textit{sheaf} of $ \mathcal O_X $-modules. 
	If $ W $ is an open set of $ X $ with inclusion $ j_W: W\hookrightarrow X $, then $ (j_W)_*(\mathscr F|W) $ is a sheaf on $ X $ such that $ (j_W)_*(\mathscr F|W)(U)=\mathscr F(U\cap W) $ for all open set $ U\subset X $. 
	%Let $ j_{i_0\ldots i_n}: U_{i_0}\cap\cdots\cap U_{i_n} \to X $ be the inclusion. 
	Let $$ \mathscr C^n(\mathscr U, \mathscr F)=\prod\limits_{i_0<i_1<\cdots<i_n\text{ in } I}(j_{U_{i_0}\cap\cdots \cap U_{i_n}})_*(\mathscr F|U_{i_0}\cap\cdots \cap U_{i_n}). $$
	For all $ x\in X $, define 
	$\partial_x^n:\mathscr C^n(\mathscr U, \mathscr F)_x\to \mathscr C^{n+1}(\mathscr U, \mathscr F)_x$ as follows. 
	Suppose $ x \in U_{i_m}$. Let $ V $ be an open neighborhood of $ x $ in $ U_{i_m} $ and let $$ s\in \mathscr C^n(\mathscr U, \mathscr F)(V)=\prod\limits_{i_0<i_1<\cdots<i_n\text{ in } I}\Gamma(V\cap U_{i_0}\cap\cdots \cap U_{i_n}, \mathscr F) $$
	$$ \partial_x^n(s)_{i_0,\cdots, i_{n+1}}
	=\left(\sum\limits_{j=0}^{n}(-1)^j
	s_{i_0,\cdots,\widehat{i_j},\cdots, i_{n+1}}
	|V\cap U_{i_0}\cap\cdots\widehat{U_{i_j}}\cdots\cap U_{i_{n+1}}
	\right)_x$$
	We simply write
	\[\partial_x^n(V,s)_{i_0\cdots i_{n+1}}
	=\sum\limits_{j=0}^{n}(-1)^j
	(V,s)_{i_0\cdots\widehat{i_j}\cdots i_{n+1}}
	\]
\end{defn}

\begin{fact}
	$ (\mathscr C^n, \partial^n) $ is a complex in $ \Mod(\mathcal O_X) $. 
\end{fact}

\begin{proof}
	Sketch: $ \partial^2=\sum\limits_{j,k}((-1)^{j+(k-1)}+(-1)^{j+k})\cdots=0$
\end{proof}

\begin{lem}
	%If $ \mathscr F\in\Mod(\mathcal O_X) $, then 
	\[0\to \mathscr F\to \mathscr C^0(\mathscr U, \mathscr F)\xrightarrow{\partial^0}\mathscr C^1(\mathscr U, \mathscr F)\xrightarrow{\partial^1}\cdots\]
	is exact (called the \textbf{\v{C}ech resolution} of $ \mathscr F $). 
\end{lem}

\begin{proof}
	It suffices to show that
	\[0\to \mathscr F_x\to \mathscr C^0(\mathscr U, \mathscr F)_x\xrightarrow{\partial_x^0}\mathscr C^1(\mathscr U, \mathscr F)_x\xrightarrow{\partial_x^1}\cdots\]
	is exact for all $ x\in X $. 
	
	Suppose $ x\in U_{i_m} $, define $ k^n_x:\mathscr C^n(\mathscr U, \mathscr F)_x\to \mathscr C^{n-1}(\mathscr U, \mathscr F)_x $: Let $ V $ be an open neighborhood of $ x $ in $ U_{i_m} $ and $ s\in \mathscr C^n(\mathscr U, \mathscr F)(V) $, let $ k^n(s)\in \mathscr C^{n-1}(\mathscr U, \mathscr F)(V) $ satisfy
	\[(k^n_x(s))_{i_0,\ldots,i_{n-1}}=(s|U_{i_m}\cap U_{i_0}\cap\ldots\cap U_{i_{n-1}})_x\]
	We simply write
	\[k_x^n(V,s)_{i_0\cdots i_{n-1}}
	=	(V,s)_{i_m,i_0\cdots i_{n-1}}
	\]
	Then \[
	\begin{array}{ll}
	&((\partial_x^{n-1}\circ k_x^n+k_x^{n+1}\circ \partial_x ^n)(V,s))_{i_0,\ldots,i_{n}}\\
	=&\sum\limits_{j=0}^{n}(-1)^jk_x^n(V,s)_{i_0\cdots \widehat{i_j}\cdots i_n}+\partial_x^n(V,s)_{i_m,i_0\cdots i_n}\\
	=&\sum\limits_{j=0}^{n}(-1)^j(V,s)_{i_m,i_0\cdots \widehat{i_j}\cdots i_n}+s_{i_0\cdots i_n}+\sum\limits_{j=1}^{n+1}(-1)^j(V,s)_{i_m,i_0\cdots\widehat{i_{j-1}}\cdots i_n}\\
	=&\sum\limits_{j=0}^{n}(-1)^j(V,s)_{i_m,i_0\cdots \widehat{i_j}\cdots i_n}+s_{i_0\cdots i_n}+\sum\limits_{\substack{l=0\\l=j-1}}^{n}(-1)^{l+1}(V,s)_{i_m,i_0\cdots\widehat{i_{k}}\cdots i_n}\\
	=&(V,s)_{i_0\cdots i_n}\\
	\end{array}
	\]
	Then $ \partial^{n-1}\circ k^n+k^{n+1}\circ \partial^n=\Id $. 
	Hence $ \Id_{h^n(\mathscr C^\bullet(\mathscr U,\mathscr F))}=h^n(\Id_{\mathscr C^\bullet(\mathscr U,\mathscr F)})=h^n(0)=0 $. Therefore $ h^n(\mathscr C^\bullet(\mathscr U,\mathscr F))=0 $ for all $ n\ge 1 $. 
	The sequence is exact at first two terms since $ \mathscr F $ is a sheaf. 
\end{proof}


\begin{defn}
	Apply $ \Gamma(X, \bullet) $ to the \v{C}ech resolution, we have a complex $ \Gamma(X, \mathscr C^{\bullet}(\mathscr U, \mathscr F)) $. 
	We write $ \check{H}^n(\mathscr U, \mathscr F)=h^n(\Gamma(X, \mathscr C^{\bullet}(\mathscr U, \mathscr F))) $.  
\end{defn}

%\begin{proof}
%	Let $ 0\to \mathscr F\to I^0\to I^1\to \cdots$ be an injective resolution. 
%	Then $$ 0\to \mathscr C^n(\mathscr U, \mathscr F)\to \mathscr C^n(\mathscr U, I^0)\to \mathscr C^n(\mathscr U, I^1)\to\cdots $$ is an injective resolution for all $ n\ge 1 $. 
%\end{proof}

\newpage
\section{Nov 17, Sheaf cohomology and \v{C}ech cohomology}\label{3}

Review: Let $ X $ be a topological space. 
Let $ \mathcal O_X $ be a sheaf of rings on $ X $. 

\begin{enumerate}
	\item An $ \mathcal O_X $-module $ \mathscr F $ is acyclic if $ H^n(X, \mathscr F)=0 $ for all $ n\ge 1 $. e.g.~flasque sheaves of $ \mathcal O_X $-modules; the locally constant sheaf $ \underline{A} $ on an irreducible topological space $ X $, where $ A $ is an abelian group. 
	\item De Rham-Weil theorem: If $ 0\to \mathscr F\to J^0\to J^1\to \cdots $ is an acyclic resolution, then $ h^n(\Gamma(X, J^{\bullet}))\simeq H^n(X, \mathscr F) $. 
	\item Let $ \mathscr U=(U_i) $ be an open covering of $ X $. 
	Let $$ 0\to \mathscr F\to \mathscr C^0(\mathscr U, \mathscr F)\to \mathscr C^1(\mathscr U, \mathscr F) \to \cdots$$ be the \v{C}ech resolution. Let $ C^n(\mathscr U, \mathscr F)=\Gamma(X,\mathscr C^n(\mathscr U, \mathscr F)) $. The $ n $-th \v{C}ech cohomology is defined to be $ \check{H}^n(\mathscr U, \mathscr F)=h^n(C^{\bullet}(\mathscr U, \mathscr F)) $
\end{enumerate}

\begin{fact}\label{F0}
	Let $ \mathscr F $ be a sheaf on $ X $. 
	Let $ U $ be an open set of $ X $. 
	If $ \mathscr F $ is flasque, then $ \mathscr F|U $ is flasque. 
\end{fact}

\begin{proof}
	For all inclusion of open sets $ V\subset W $ in $ U $, the restriction
	\[(\mathscr F|U)(W)=\mathscr F(W)\to \mathscr F(V)=(\mathscr F|U)(V)\]
	is surjective. 
\end{proof}

\begin{fact}\label{F1}
	Let $ f:X\to Y $ be a continuous map. 
	Let $ \mathscr F $ be a sheaf on $ X $. 
	Let $ f_*\mathscr F $ be the sheaf on $ Y $ such that $ f_*\mathscr F(U)=F(f^{-1}(U)) $ for all open set $ U $ of $ X $. 
	If $ \mathscr F $ is flasque, then $ f_*\mathscr F $ is flasque. 
\end{fact}

\begin{proof}
	For all inclusion of open sets $ V\subset U $ in $ Y $, since $ f $ is continuous, $ f^{-1}(V)\subset f^{-1}(U) $ is an inclusion of open sets in $ X $. 
	Since $ F $ is flasque, the restriction $ \mathscr F(f^{-1}(U))\to \mathscr F(f^{-1}(V)) $ is surjective, i.e.~$ f_*\mathscr F(U)\to f_*\mathscr F(V) $ is surjective. 
\end{proof}

\begin{fact}\label{F2}
	If $ \mathscr F_i $ are flasque sheaves for all $ i\in I $, then $ \prod\limits_{i\in I} \mathscr F_i$ is a flasque sheaf. 
\end{fact}

\begin{proof}
	For all inclusion of open sets $ V\subset U $ in $ X $, since $ \mathscr F_i $ is flasque, $ \mathscr F_i(U)\to \mathscr F_i(V) $ is surjective. 
	Then $ \prod\limits_{i\in I}\mathscr F_i(U)\to  \prod\limits_{i\in I}\mathscr F_i(V)$  surjective. 
\end{proof}

\begin{prop}\label{derived}
	$ \check{H}^n(\mathscr U, \bullet)=R^n\check{H}^0(\mathscr U, \bullet):\Mod(\mathcal O_X)\to \Ab $. 
\end{prop}

\begin{proof}
	It suffices to show that $ \check{H}^n(\mathscr U, \bullet) $ is a universal $ \delta $-functor. It is a $ \delta $-functor since it is cohomology of a cochain complex. 
	We show that $ \check{H}^n(\mathscr U, \bullet) $ is effacable. 
	Since $ \Mod(\mathcal O_X) $ has enough injectives, it suffices to show that for all injective $ \mathcal O_X $-module $ I $, $\check{H}^n(\mathscr U, I)=0$ for all $ n\ge 1 $. 
	
	By \cref{F0}, \cref{F1} and \cref{F2}, $$ \mathscr C^n(\mathscr U, \mathscr F)=\prod\limits_{i_0<i_1<\cdots<i_n\text{ in } I}(j_{U_{i_0}\cap\cdots \cap U_{i_n}})_*(\mathscr F|U_{i_0}\cap\cdots \cap U_{i_n}) $$ is flasque. 
	By \cref{DW}, $\check{H}^n(\mathscr U, I)=H^n(X, I)$. 
	Finally, by \cref{eg-inj}, $ H^n(X,I)=0 $. Therefore $ \check{H}^n(\mathscr U, \bullet) $ is effacable. 
\end{proof}

Note that $ C^n(\mathscr U, \mathscr F)=\prod\limits_{i_0<\cdots<i_n}\mathscr F(U_{i_0}\cap\cdots\cap U_{i_n}) $, from now on, we extend our definition of \v{C}ech cohomology to \[\check{H}^n(\mathscr U, \bullet): \PreMod(\mathcal O_X)\to \Ab\]
Let $ \iota:\Mod(\mathcal O_X)\to\PreMod(\mathcal O_X) $ be the inclusion/forgetful functor. 
Then $ \Gamma(U,\bullet)=\Gamma_{\Pre}(U, \bullet)\circ\iota $ where $ \Gamma_{\Pre}(U, \bullet)$ is the presheaf global section functor for all open set $ U $ of $ X $,. 

\begin{lem}\label{3.5}
	%Let $ \mathscr U=(U_i)_{i\in I} $ be an open covering of $ X $. 
	
	Then for all $ \mathscr \mathscr F\in\Mod(\mathcal O_X) $ and $ \Gamma_{\Pre}(U, R^q\iota (\mathscr F))\simeq H^q(U, \mathscr F) $. 
\end{lem}

\begin{proof}
	Let $ G_1=\iota $, it is left exact. 
	Let $ G_2=\Gamma_{pre}(U, \bullet):\PreMod(\mathcal O_X)\to \Mod(\mathcal O_X) $. 
	Then $ G_2 $ is exact and hence $ R^pG_2=0 $ for all $ p\ge 1 $. 
	If $ I $ is an injective $ \mathcal O_X $-module, then $ \iota(I) $ is $ \Gamma_{\Pre}(U, \bullet) $-acyclic. 
	
	By Grothendieck spectral sequence, $$ E_2^{p,q}=R^pG_2(R^qG_1(\mathscr F))\Rightarrow L^{p+q}=R^{p+q}(G_2\circ G_1)(\mathscr F), $$
	where $ E_2^{p,q}=0 $ for all $ p\ge 1 $. 
	Then $ E_2^{p,q}=E_{\infty}^{p,q}=\dfrac{F^pL^{p+q}}{F^{p+1}L^{p+q}} $. 
	We have $$ F^1L^{p+q}=F^2L^{p+q}=\cdots=F^{p+q+1}L^{p+q}=0 $$. 
	
	When $ p=0 $, we have $ \Gamma(U, R^q\iota (\mathscr F)) =R^0G_2(R^qG_1(\mathscr F))=E_2^{0,q}=E_{\infty}^{0,q}=\dfrac{F^0L^{p+q}}{F^1L^{p+q}}=L^{p+q}=R^q(\Gamma(U,\iota(\mathscr F)))=R^q(\Gamma(U,\mathscr F))=H^q(U, \mathscr F)$. 	
\end{proof}




\begin{lem}\label{3.4}
	Let $ \mathscr U=(U_i)_{i\in I} $ be an open covering of $ X $. 
	%Let $ \iota:\Ab(X)\to\PreAb(X) $ be the inclusion functor. 
	There exists an exact sequence 
	\[E_2^{p,q}=\check{H}^p(\mathscr U, R^q\iota (\mathscr F))\Rightarrow H^{p+q}(X, \mathscr F).\]
\end{lem}

\begin{proof}
	Let $ G_1=i $ and $ G_2=\check{H}^0(\mathscr U, \bullet)$. 
	If $ I $ is an injective $ \mathcal O_X $-module, then by \cref{derived}, $ \iota(I) $ is $ \check{H}^0(\mathscr U, \bullet) $-acyclic. 
	
	By the Grothendieck spectral sequence, $$ E_2^{p,q}=R^pG_2(R^qG_1(\mathscr F))\Rightarrow L^{p+q}=R^{p+q}(G_2\circ G_1)(\mathscr F), $$
	By \cref{derived}, $ E_2^{p,q}=R^p\check{H}^0(\mathscr U,\bullet)( R^q\iota (\mathscr F))=\check{H}^p(\mathscr U, R^q\iota (\mathscr F))$. Also 
	 $$ L^{p+q}=R^{p+q}(\check{H}^0(\mathscr U, \bullet)\circ\iota)(\mathscr F)=R^{p+q}(\Gamma(X,\bullet))(\mathscr F)=H^{p+q}(X,\mathscr F).$$
\end{proof}

\begin{thm}\label{3.11}
	If $ H^q(U_i, \mathscr F)=0 $ for all $ U_i $ and $ q\ge 1 $, then $ \check{H}^p(\mathscr U, \iota(\mathscr F))\simeq H^p(X, \mathscr F)$ for all $ p\ge 0 $. 
\end{thm}


\begin{proof}
	By \cref{3.5}, $ \Gamma_{\Pre}(U, R^q\iota (\mathscr F))\simeq H^q(U, \mathscr F) =0$ for all $ q\ge 1 $. 
	Then $ C^{\bullet}(\mathscr U, R^q\iota (\mathscr F))=0 $.
	By \cref{3.4}, $$E_2^{p,q}=\check{H}^p(\mathscr U, R^q\iota (\mathscr F))\Rightarrow L^{p+q}=H^{p+q}(X, \mathscr F).$$ 
	Hence $ E_2^{p,q}=0 $ for all $ q\ge 1 $. 
	Then $ 0=E_2^{p,q}=E_{\infty}^{p,q}=\dfrac{F^pL^{p+q}}{F^{p+1}L^{p+q}} $ for all $ p $ and for all $ q\ge 1 $. Then $ L^p=F^0L^p=F^1L^p=\cdots=F^pL^p $. 
	
	Then $ \check{H}^p(\mathscr U, \iota(\mathscr F))=E_2^{p,0}=E_{\infty}^{p,0}=\dfrac{F^pL^p}{F^{p+1}L^p}=L^p=H^p(X,\mathscr F) $. 
\end{proof}

\begin{eg}
	$ H^1(\mathbb P^1_{\mathbb C}, \underline{\mathbb Z})= \mathbb Z$. 
\end{eg}

\begin{proof}
	Let $ \pi:\mathbb C^2\setminus\{(0,0)\}\to \mathbb P_{\mathbb C}^1 $ be the canonical quotient map. 
	Let $ U_0=\{\pi(x_0,x_1)~|~x_0\ne 0\} $. % and $ U_1=\{(x_0,x_1)\in \mathbb C^2~|~x_1\ne 0\} $. 
	Then $ U_0\simeq \mathbb A_{\mathbb C}^1 ,~\pi(x_0,x_1)\mapsto \frac{x_1}{x_0}:=x$. 
	Let $ U_1=\{\pi(x_0,x_1)~|~x_1\ne 0\} $. % and $ U_1=\{(x_0,x_1)\in \mathbb C^2~|~x_1\ne 0\} $. 
	Then $ U_1\simeq \mathbb A_{\mathbb C}^1 ,~\pi(x_0,x_1)\mapsto \frac{x_0}{x_1}:=y$. 
	We have $ U_0\cap U_1=\mathbb{P}_{\mathbb C}^1\setminus\{\pi(1,0),\pi(0,1)\} $ has two connected components. 
	
	Since $ U_0,U_1\simeq\mathbb A_{\mathbb C}^1 $ are irreducible, by \cref{eg1}, $ H^i(U_0, \underline{Z})=0 $ and $ H^i(U_1, \underline{Z})=0 $ for all $ i\ge 1 $. 
	Let $ \mathscr U=(U_0,U_1) $. 
	Then $$ C^0(\mathscr U, \underline{Z})=\Gamma(U_0, \underline{Z})\oplus \Gamma(U_1, \underline{Z})=\mathbb Z\oplus\mathbb Z,$$
	$$ C^1(\mathscr U, \underline{Z})=\Gamma(U_0\cap U_1, \underline{Z})=\mathbb Z\oplus\mathbb Z,$$
	\[d^0(a,b)=(a-b,a-b)\]
	The complex is
	\[0\to C^0(\mathscr U, \underline{\mathbb Z})\xrightarrow{d^0}C^1(\mathscr U, \underline{\mathbb Z})\to 0\]
	Finally, by \cref{3.11}, $$ H^1(\mathbb P^1_{\mathbb C}, \underline{\mathbb Z})\simeq\check H^1(\mathscr U, \underline{\mathbb Z}) =\dfrac{C^1(\mathscr U, \underline{\mathbb Z})}{\im(d^0)}\simeq\dfrac{\mathbb Z\oplus\mathbb Z}{\mathbb Z}\simeq \mathbb Z .$$
	%	Here $ \infty=\pi(0,1)$, $\{\infty\}=\mathbb P_{\mathbb C}^1\setminus V_0$. 
	%	Let $ j: \{\infty\}\to  \mathbb P_{\mathbb C}^1$ be the inclusion. Let $ \underline{\mathbb Z}_{\infty}= j_*(\underline{\mathbb Z}|_{\{\infty\}})$. 
	%	Let $\underline{\mathbb Z}_U$ be the sheaf associated to the presheaf for all open set $ V $ in $ X $: $$ V\mapsto
	%	\left\{
	%	\begin{array}{ll}
	%	\underline{\mathbb Z}(V),&V\subset U,\\
	%	0,&V\not\subset U.\\
	%	\end{array}
	%	\right.
	%	 $$
	%	 Then there exists a short exact sequence of sheaves on $ \mathbb P_{\mathbb C}^1 $
	%	 \[0\to \underline{\mathbb Z}_U\to \underline{\mathbb Z}\to \underline{\mathbb Z}_\infty\to 0\]
	%	 \[0\to \underline{\mathbb Z}_U\to \underline{\mathbb Z}\to \underline{\mathbb Z}_\infty\to H^1(\mathbb P_{\mathbb C}^1, \underline{\mathbb Z}_U)\to H^1(\mathbb P_{\mathbb C}^1, \underline{\mathbb Z}_U)\to H^1(\mathbb P_{\mathbb C}^1, \underline{\mathbb Z}_\infty)\]
	
	%	$$ V_1=\pi(U_1)\simeq \mathbb A_{\mathbb C}^1 ,~(x_0:x_1)\mapsto \dfrac{x_0}{x_1}:=y, $$
	%	\[V_0\cap V_1=\pi(U_0\cap U_1),~y=\dfrac{1}{x}\]
\end{proof}

\begin{defn}
	An open covering $ \mathscr U'=(U_j')_{j\in J} $ is called a \textbf{refinement} of $ \mathscr U=(U_i)_{i\in I} $ if there exists a map $\varphi: J\to I $ such that $ U_{j}'\subset U_{\varphi(j)} $. 
	%We write $ \mathscr U'<\mathscr U $. 
\end{defn}

\begin{fact}
	(1) If $ \varphi:J\to I $ gives a {refinement} $ \mathscr U'=(U_j')_{j\in J} $ of $ \mathscr U=(U_i)_{i\in I} $, then there exists a homomorphism $ \varphi_*: \check{H}^n(\mathscr U', \mathscr F)\to \check{H}^n(\mathscr U, \mathscr F) $. 
	
	(2) If $ \varphi_1,\varphi':J\to I $ give two {refinements} $ \mathscr U'=(U_j')_{j\in J} $ of $ \mathscr U=(U_i)_{i\in I} $, they give the same homomorphism $ \varphi_*=\varphi_*': \check{H}^n(\mathscr U', \mathscr F)\to \check{H}^n(\mathscr U, \mathscr F) $. 
	
	(3) Cohomology groups of open coverings and corresponding homomorphisms of refinements form a directed system. 
\end{fact}

\begin{proof}
	Omit.
\end{proof}

\begin{defn}
	The \textbf{\v{C}ech cohomology} of presheaf $ \mathscr F $ of $ \mathcal O_X $-modules is $$ \check{H}^n(X,\mathscr F)=\lim\limits_{\longrightarrow} \check{H}^n(\mathscr U, \mathscr F) .$$ 
\end{defn}

\begin{lem}\label{3.7}
	%Let $ \mathscr U=(U_i)_{i\in I} $ be an open covering of $ X $. 
	There exists an exact sequence 
	\[E_2^{p,q}=\check{H}^p(X, R^q\iota (\mathscr F))\Rightarrow H^{p+q}(X, \mathscr F).\]
\end{lem}

\begin{proof}
	Let $ G_1=i $ and $ G_2=\check{H}^0(X, \bullet)$. 
	Let $ I $ be an injective sheaf of $ \mathcal O_X $-modules. 
	Since $ \iota(I) $ is $ \check{H}^0(\mathscr U, \bullet) $-acyclic for all $ \mathscr U $, $ \iota(I) $ is $ \check{H}^0(X, \bullet) $-acyclic. 
	
	By Grothendieck spectral sequence, $$ E_2^{p,q}=R^pG_2(R^qG_1(\mathscr F))\Rightarrow L^{p+q}=R^{p+q}(G_2\circ G_1)(\mathscr F), $$
	where $ E_2^{p,q}=R^p\check{H}^0(X, \bullet)(R^q\iota(\mathscr F))=\check{H}^p(X, R^q\iota (\mathscr F))$ and 	
	$$ L^{p+q}=R^{p+q}(\check{H}^0(X, \bullet)\circ\iota)(\mathscr F)=R^{p+q}(\Gamma(X,\bullet))(\mathscr F)=H^{p+q}(X,\mathscr F).$$
\end{proof}

\begin{defn}
	Let $ \mathscr F $ be a presheaf. 
	Its \textbf{sheafification} $ \#\mathscr F $ is a sheaf, for all open set $ U $, $ (\#\mathscr F)(U) $ consists of sections $ s:U\to \prod\limits_{P\in U}\mathscr F_P$  such that 
	
	(1) $ s(P)\in \mathscr F_P $ for all $ P\in U $. 
	
	(2) For all $ P\in U $, there exists an open neighborhood $ V $ of $ P $ in $ U $ and $ t\in \mathscr F(V) $ such that $ t_Q=s(Q) $ for all $ Q\in V $. 
\end{defn}

\begin{fact}\label{3.13}
	The sheafification functor $ \#:\PreMod(\mathcal O_X)\to \Mod(\mathcal O_X) $ is exact. 
\end{fact}

\begin{proof}
	Sketch: Since $ \lim\limits_{\longrightarrow} $ is left exact and (2) above.
%	Suppose $ 0\to F(U)\to G(U)\xrightarrow{q_U} H(U)\to 0 $ is exact for all open set $ U $. \textbf{Since $ \lim\limits_{\longrightarrow} $ is left exact},  $ 0\to F_x\to G_x\to H_x$ is exact. 
%	
%	We only show that $ G_x\xrightarrow{q_x} H_x $ is surjective for all $ x\in X $. 
\end{proof}

\begin{lem}\label{3.8}
	$ \check{H}^0(X, R^q\iota (\mathscr F))=0 $ for all $ q\ge 1 $. 
\end{lem}

\begin{proof}
	%Let $ \#:\PreAb\to \Ab $ be the sheafification functor.  
	%Then $ \#(\iota(\mathscr F))=\mathscr F $ and 
	By \cref{3.13} $ R^p\#=0 $ for all $ p\ge 1 $. 
	In particular, if $ I $ is an injective sheaf of $ \mathcal O_X $-modules, then $ \iota(\mathscr F) $ is $ \# $-acyclic. 
	
	By the Grothendieck spectral sequence, $$ E_2^{p,q}=R^p\#(R^q\iota (\mathscr F))\Rightarrow L^{p+q}=R^{p+q}(\#\circ i)(\mathscr F), $$
	where $ E_2^{p,q}= 0$ for all $ p\ge 1 $. 
	Also $ L^n =R^{n}\Id(\mathscr F) =0$ for all $ n\ge 1 $ because $ \Id $ is exact. 
	%Then $ 0=E_2^{p,q}=E_{\infty}^{p,q}=\dfrac{F^pL^{p+q}}{F^{p+1}L^{p+q}} $
	Then $ \#(R^q\iota (\mathscr F))=E_2^{0,q}=E_{\infty}^{0,q}=0$ for all $ q\ge 1 $. 
	Therefore $ \check{H}^0(X, R^q\iota (\mathscr F))=\Gamma_{\Pre}(X, R^q\iota (\mathscr F))=\Gamma(X, \#(R^q\iota (\mathscr F)))=0 $ for all $ q\ge 1 $. 
\end{proof}

\begin{thm}\label{main}
	For all $ \mathscr \mathscr F\in \Ab(X) $, 
	
	(a) $\check{H}^0(X,\iota(\mathscr F))\simeq H^0(X, \mathscr F)$
	
	(b) $\check{H}^1(X,\iota(\mathscr F))\simeq H^1(X, \mathscr F)$
	
	(c) There exists and exact sequence $$0\to \check{H}^2(X,\iota(\mathscr F))\to H^2(X, \mathscr F)\to \check{H}^1(X, R^1\iota(\mathscr F))\to \check{H}^3(X, \iota(\mathscr F))$$
	
	%(d) If $ E_2^{p,q}=0 $ for all $ q\ge 1 $, then $\check{H}^p(X,\iota(\mathscr F))\simeq H^p(X, \mathscr F)$ for all $ p $. 
\end{thm}

\begin{proof}
	%$\check{H}^0(X,\iota(\mathscr F))\simeq \Gamma(X, \mathscr F)\simeq H^0(X, \mathscr F)$. 
	
	By \cref{3.7} and \cref{3.8}, there exists an exact sequence
	\[E_2^{p,q}=\check{H}^p(X, R^q\iota (\mathscr F))\Rightarrow L^{p+q}=H^{p+q}(X, \mathscr F).\]
	such that $ E_2^{0,q}=0 $ for all $ q\ge 1 $. 
	\begin{equation}
		\xymatrix{
			 &\vdots&\vdots &\vdots &\vdots &&\\
			&0\ar[rrd]&0 \ar[rrd]&E_2^{1,2}&E_2^{2,2} & \cdots&\\
			0\ar[rrd]&0\ar[rrd]&0 \ar[rrd]&E_2^{1,1} \ar[rrd]&E_2^{2,1}\ar[rrd] &E_2^{3,1}&\cdots\\
			&0&E_2^{0,0} \ar[rrd]&E_2^{1,0}\ar[rrd] &E_2^{2,0}\ar[rrd]  &E_2^{3,0}&\cdots\\
			&0&0&0 &0& 0&0\\
		}
	\end{equation}
	
	(a) $ \check{H}^0(X,\iota(\mathscr F))=E_2^{0,0}= E_{\infty}^{0,0}=\dfrac{F^0L^0}{F^1L^0}=L^0=H^0(X,\mathscr F) $. 
	
	(b) $ \check{H}^1(X,\iota(\mathscr F))=E_2^{1,0}= E_{\infty}^{1,0}=\dfrac{F^1L^1}{F^2L^1}=F^1L^1 $. 
	
	Since $0=E_2^{0,1}= E_{\infty}^{0,1}=\dfrac{F^0L^1}{F^1L^1}=\dfrac{H^1(X,\mathscr F)}{F^1L^1} $, we have $ H^1(X,\mathscr F)=F^1L^1 $. 
	
	Hence $ \check{H}^1(X,\iota(\mathscr F))=F^1L^1=H^1(X,\mathscr F)$. 
	
	(c) $ \check{H}^2(X,\iota(\mathscr F))=E_2^{2,0}= E_{\infty}^{2,0}=\dfrac{F^2L^2}{F^3L^2}=F^2L^2 $.  
	
	Since $ 0=E_2^{0,2}= E_{\infty}^{0,2}=\dfrac{F^0L^2}{F^1L^2}=\dfrac{H^2(X,\mathscr F)}{F^1L^2}$, we have $ H^2(X,\mathscr F)=F^1L^2 $. 
	
	Since $ 0=E_3^{-2,3}\to E_3^{1,1}\to  E_3^{4,-1}=0  $, $ E_3^{1,1}= E_{\infty}^{1,1}=\dfrac{F^1L^2}{F^2L^2}$. 	
	
	Since $ 0\to F^2L^2\to F^1L^2\to \dfrac{F^1L^2}{F^2L^2}\to 0 $ is exact, $$0\to \check{H}^2(X,\iota(\mathscr F))\to H^2(X, \mathscr F)\to E_3^{1,1}\to 0$$ is exact. 
	By $E_3^{1,1}=\ker(d_2^{1,1}:E_2^{1,1}\to E_2^{3,0})=\ker(\check{H}^1(X, R^1\iota(\mathscr F))\to \check{H}^3(X, \iota(\mathscr F)))$, we have that 
	$$0\to \check{H}^2(X,\iota(\mathscr F))\to H^2(X, \mathscr F)\to \check{H}^1(X, R^1\iota(\mathscr F))\to \check{H}^3(X, \iota(\mathscr F))$$is exact. 
%	
%	Finally, since $ 0=E_3^{0,2}\to E_3^{3,0}\to  E_3^{5,-2}=0  $, $ E_3^{3,0}= E_{\infty}^{3,0}=\dfrac{F^3L^3}{F^4L^3}=F^3L^3$. 	
	%(d) $ \check{H}^p(X,\iota(\mathscr F))=E_2^{p,0}= E_{\infty}^{p,0}=\dfrac{F^pL^p}{F^{p+1}L^p}=F^pL^p $.  
	%
	%For all $ m\ge 1 $, $ 0=E_2^{n,m}=E_{\infty}^{n,m}=\dfrac{F^nL^{n+m}}{F^{n+1}F^{n+m}} $. 
	%Let $ m=p-n $. We have $ F^0L^p=F^1L^p=\cdots=F^pL^p $. 
	%
	%Hence $ \check{H}^p(X,\iota(\mathscr F))=F^pL^p=F^0L^p=L^p=H^p(X,\mathscr F)$. 
\end{proof}



\begin{eg}
	Grothendieck \cite[3.8.3]{Gro57} gives an example such that 
	
	\begin{center}
		$ \check H^2(X, \iota(\mathscr F))=0$ and $ H^2(X, \mathscr F)=\mathbb Z $. 
	\end{center}%Details are omitted. 

	Let $ X=\mathbb A_{\mathbb C}^2 $, $ Y_1=Z(x^2+y^2-1) $, $ Y_2=Z(x^2-2x+y^2) $, $ Y=Y_1\cup Y_2 $. Then $ Y_1 $ and $ Y_2 $ are irreducible and $ Y_1\cap Y_2=\{(\frac{1}{2}, \frac{\sqrt{3}}{2}), (\frac{1}{2}, -\frac{\sqrt{3}}{2})\} $
	Let $ \mathbb Z_X $ be the locally constant sheaf on $ X $ associated to $ \mathbb Z $. 
	If $ j: F\hookrightarrow X $ is the inclusion of a closed set, then we define $ \mathbb Z_F=j_*(\mathbb Z_X|F) $. 
	If $ i: U\hookrightarrow X $ is the inclusion of an open set, then we define $ \mathbb Z_{U}=i_!(\mathbb Z_X|U) $.
	
	\underline{(1) $ H^2(X, \mathbb Z_{X\setminus Y})\simeq H^1(X, \mathbb Z_Y) $.} From the short exact sequence 
	\[0\to \mathbb Z_{X\setminus Y}\to \mathbb Z_X\to \mathbb Z_Y\to 0,\]
	we obtain a long exact sequence 
	\[
	\cdots\to H^1(X, \mathbb Z_X)\to H^1(X, \mathbb Z_Y)\to H^2(X, \mathbb Z_{X\setminus Y})\to H^2(X, \mathbb Z_X)\to \cdots
	\]
	Since $ X $ is irreducible, by \cref{eg1}, we have $ H^1(X, \mathbb Z_X)=0 $ and $ H^2(X, \mathbb Z_X)=0 $. Then $ H^2(X, \mathbb Z_{X\setminus Y})\simeq H^1(X, \mathbb Z_Y) $. 
	
	\underline{(2) $ H^1(X, \mathbb Z_Y)\simeq \mathbb Z $.} From the short exact sequence 
	\[0\to \mathbb Z_Y\to \mathbb Z_{Y_1}\oplus\mathbb Z_{Y_2}\to \mathbb Z_{Y_1\cap Y_2}\to 0\]
	we obtain a long exact sequence 
	\[
	\cdots\to H^0(X, \mathbb Z_{Y_1}\oplus\mathbb Z_{Y_2})\xrightarrow{f} H^0(X, \mathbb Z_{Y_1\cap Y_2})\to H^1(X, \mathbb Z_{Y})\to H^1(X, \mathbb Z_{Y_1}\oplus\mathbb Z_{Y_2})\to \cdots
	\]
	Where $ H^0(X, \mathbb Z_{Y_1}\oplus\mathbb Z_{Y_2})\simeq  H^0(X, \mathbb Z_{Y_1})\oplus  H^0(X, \mathbb Z_{Y_2})= \Gamma(X, \mathbb Z_{Y_1})\oplus\Gamma(X, \mathbb Z_{Y_2})\simeq\mathbb Z\oplus\mathbb Z$. 
	Since $ Y_1\cap Y_2 $ has two connected components, $ H^0(X, \mathbb Z_{Y_1\cap Y_2})=\Gamma(X, \mathbb Z_{Y_1\cap Y_2})\simeq \mathbb Z\oplus\mathbb Z $. 
	The map $ f: \mathbb Z\oplus\mathbb Z\to  \mathbb Z\oplus\mathbb Z$ is defined as $ f(m,n)=(m-n,m-n) $ for all $ m,n\in\mathbb Z $. 
	By \cref{eg1}, $ H^1(X, \mathbb Z_{Y_1}\oplus\mathbb Z_{Y_2}) \simeq H^1(X, \mathbb Z_{Y_1})\oplus H^1(X, \mathbb Z_{Y_2})\simeq 0\oplus 0=0$. 
	Hence, $ H^1(X, \mathbb Z_Y)=\cok(f)=\dfrac{\mathbb Z\oplus\mathbb Z}{\im(f)}\simeq\mathbb Z $. 
	
	It follows from (1)(2) that $ H^2(X, \mathbb Z_{X\setminus Y})\simeq H^1(X, \mathbb Z_Y)\simeq\mathbb Z $. 
	
	\underline{(3) Let $ U $ be an open set such that $ |U\cap Y_1\cap Y_2|\le 1 $, we calculate $ H^1(U, \mathbb Z_{U\setminus Y}) $. }
	From the short exact sequence
	\[0\to \mathbb Z_{U\setminus Y}\to \mathbb Z_U\to \mathbb Z_{U\cap Y}\to 0,\]
	we obtain a long exact sequence 
	\[
	0\to H^0(U, \mathbb Z_{U\setminus Y})\to H^0(U, \mathbb Z_{U})\xrightarrow{g} H^0(U, \mathbb Z_{U\cap Y})\to H^1(U, \mathbb Z_{U\setminus Y})\to H^1(U, \mathbb Z_{U})\to \cdots
	\]
	where $ H^0(U, \mathbb Z_U)\simeq \mathbb Z $ and by \cref{eg1}, $ H^1(U, \mathbb Z_U)= 0 $. 
	
	(3a) Suppose $ U\cap Y_1\ne\emptyset $, $ U\cap Y_2\ne\emptyset $ and $ U\cap Y_1\cap Y_2=\emptyset $. 
	Then $ U\cap Y $ has two connected components, $ H^0(U, \mathbb Z_{U\cap Y})=\mathbb Z\oplus\mathbb Z $. 
	Also $ g(m)=(m,m) $ for all $ m\in \mathbb Z $. Hence $ H^1(U, \mathbb Z_{U\setminus Y})\simeq \dfrac{\mathbb Z\oplus\mathbb Z}{\im(g)}\simeq\mathbb Z $. 
	
	(3b) Suppose $ U\cap Y_1\ne\emptyset $ or $ U\cap Y_2\ne\emptyset $ or $ U\cap Y_1\cap Y_2\ne\emptyset $. Then $ U\cap Y $ has at most one connected component. If $ U\cap Y=\emptyset $, then $ H^0(U, \mathbb Z_{U\cap Y})=0$ and hence $ H^1(U, \mathbb Z_{U\setminus Y})=0 $; If $ U\cap Y$ has only one connected component, then $ H^0(U, \mathbb Z_{U\cap Y})=\mathbb Z$, $ g=\Id_{\mathbb Z} $ and hence $ H^1(U, \mathbb Z_{U\setminus Y})=0 $.  
	
	\underline{(4) We show that $ \check{H}^1(X, R^1\iota(\mathbb Z_{X\setminus Y}))\simeq \mathbb Z $.} 
	Let $ \mathscr U=(U_i)_{i\in I} $ be an open covering such that 
	there exists a unique $ a\in I $ such that $ (\frac{1}{2},\frac{\sqrt{3}}{2})\in U_a $; there exists a unique $ b\in I, b\ne a $ such that $ (\frac{1}{2},-\frac{\sqrt{3}}{2})\in U_b $; 
	for all $ i\in I\setminus\{a,b\} $, $ U_i\cap Y_1=\emptyset $ or $ U_i\cap Y_2=\emptyset $. Then $ U_i$ satisfies (3b) for all $ i\in I $. 
	
	Consider the \v{C}ech complex
	\[C^0(\mathscr U, R^1\iota(\mathbb Z_{X\setminus Y}))\xrightarrow{d^0} C^1(\mathscr U, R^1\iota(\mathbb Z_{X\setminus Y}))\xrightarrow{d^1} C^2(\mathscr U, R^1\iota(\mathbb Z_{X\setminus Y}))\to\cdots.\]
	Then $$ 
	\begin{array}{llll}
	C^0(\mathscr U, R^1\iota(\mathbb Z_{X\setminus Y}))&=&\prod\limits_{i\in I}\Gamma_{\Pre}(U_i, R^1\iota(\mathbb Z_{X\setminus Y}))&\\
	&\simeq &\prod\limits_{i\in I}H^1(U_i, \mathbb Z_{U_i\setminus Y})&\text{ by \cref{3.5}}\\
	&=&0 &\text{ by (3b)}\\.
	\end{array}
	$$
	
	$$ 
	\begin{array}{llll}
	C^1(\mathscr U, R^1\iota(\mathbb Z_{X\setminus Y}))&=&\prod\limits_{i<j}\Gamma_{\Pre}(U_i\cap U_j, R^1\iota(\mathbb Z_{X\setminus Y}))&\\
	&\simeq& \prod\limits_{i<j}H^1(U_i\cap U_j, \mathbb Z_{U_i\cap U_j\setminus Y}) &\text{ by \cref{3.5}}\\
	&\simeq& H^1(U_a\cap U_b, \mathbb Z_{U_s\cap U_t\setminus Y})&\text{ by (3b)}\\
	&\simeq& \mathbb Z&\text{ by (3a)}\\
	\end{array}
	$$
	
	For all $ i\in I\setminus\{a,b\} $, $ U_i\cap U_a\cap U_b$ satisfies (3b), then
	
	$$ 
	\begin{array}{llll}
	C^2(\mathscr U, R^1\iota(\mathbb Z_{X\setminus Y}))&=&\prod\limits_{i<j<k}\Gamma_{\Pre}(U_i\cap U_j\cap U_k, R^1\iota(\mathbb Z_{X\setminus Y}))&\\
	&\simeq& \prod\limits_{i<j<k}H^1(U_i\cap U_j\cap U_k, \mathbb Z_{U_i\cap U_j\cap U_k\setminus Y}) &\text{ by \cref{3.5}}\\
	&=&0&\text{ by (3b)}\\
	\end{array}
	$$
	
	Then the the \v{C}ech complex becomes
	$0\xrightarrow{d^0} \mathbb Z\xrightarrow{d^1} 0\to\cdots$ and hence $ \check{H}^1(X, R^1\iota(\mathbb Z_{X\setminus Y}))\simeq \mathbb Z $. 
	
	\underline{(5) Finally, we show that $ \check{H}^2(X, \iota(\mathbb Z_{X\setminus Y}))=0 $.} 
By \cref{main}(c), we have a commutative diagram with exact rows $$
\xymatrix{
0\ar[r]&\check{H}^2(X,\iota(\mathscr F))\ar[r]\ar@{=}[d]&H^2(X, \mathscr F)\ar[r]\ar@{=}[d]^{\text{(1)(2)}} &\check{H}^1(X, R^1\iota(\mathscr F))\ar[r]\ar@{=}[d]^{\text{(4)}}&\check{H}^3(X, \iota(\mathscr F))\ar@{=}[d]^{\dim(X)=2}\\
0\ar[r]& \check{H}^2(X,\iota(\mathscr F))\ar[r] &\mathbb Z\ar[r]&\mathbb Z\ar[r]&0\\
}
$$
Therefore $ \check{H}^2(X, \iota(\mathbb Z_{X\setminus Y}))=0 $. 

To summarize, let $ \mathscr F= \mathbb Z_{X\setminus Y}$, we have $ \check H^2(X, \iota(\mathscr F))=0\ne H^2(X, \mathscr F)=\mathbb Z  $.
\end{eg}

%\section{bib}
\begin{filecontents}{wu.bib}
	@article {Gro57,
		AUTHOR = {Grothendieck, Alexander},
		TITLE = {Sur quelques points d'alg\`ebre homologique},
		JOURNAL = {T\^ohoku Math. J. (2)},
		FJOURNAL = {The Tohoku Mathematical Journal. Second Series},
		VOLUME = {9},
		YEAR = {1957},
		PAGES = {119--221},
		ISSN = {0040-8735},
		MRCLASS = {18.00},
		MRNUMBER = {0102537},
		MRREVIEWER = {D. Buchsbaum},
	}
	
	@book {Hartshorne,
		AUTHOR = {Hartshorne, Robin},
		TITLE = {Algebraic geometry},
		NOTE = {Graduate Texts in Mathematics, No. 52},
		PUBLISHER = {Springer-Verlag, New York-Heidelberg},
		YEAR = {1977},
		PAGES = {xvi+496},
		ISBN = {0-387-90244-9},
		MRCLASS = {14-01},
		MRNUMBER = {0463157},
		MRREVIEWER = {Robert Speiser},
	}
	
	@book {Ive86,
		AUTHOR = {Iversen, Birger},
		TITLE = {Cohomology of sheaves},
		SERIES = {Universitext},
		PUBLISHER = {Springer-Verlag, Berlin},
		YEAR = {1986},
		PAGES = {xii+464},
		ISBN = {3-540-16389-1},
		MRCLASS = {14F05 (14-01 18-01 54-01)},
		MRNUMBER = {842190},
		MRREVIEWER = {G. Horrocks},
		DOI = {10.1007/978-3-642-82783-9},
		URL = {http://dx.doi.org/10.1007/978-3-642-82783-9},
	}
		
	@article {Ste65,
		AUTHOR = {Steinberg, Robert},
		TITLE = {Regular elements of semisimple algebraic groups},
		JOURNAL = {Inst. Hautes \'Etudes Sci. Publ. Math.},
		FJOURNAL = {Institut des Hautes \'Etudes Scientifiques. Publications
			Math\'ematiques},
		NUMBER = {25},
		YEAR = {1965},
		PAGES = {49--80},
		ISSN = {0073-8301},
		MRCLASS = {14.50 (20.75)},
		MRNUMBER = {0180554},
		MRREVIEWER = {J. L. Koszul},
	}
	
	@article {Lang56,
		AUTHOR = {Lang, Serge},
		TITLE = {Algebraic groups over finite fields},
		JOURNAL = {Amer. J. Math.},
		FJOURNAL = {American Journal of Mathematics},
		VOLUME = {78},
		YEAR = {1956},
		PAGES = {555--563},
		ISSN = {0002-9327},
		MRCLASS = {14.0X},
		MRNUMBER = {0086367},
		MRREVIEWER = {P. Roquette},
	}
	
	@book{Spr63,
		author    = "Springer, T. A. and P. Eysenbach",
		title     = "Oktaven, Jordan-Algebren und Ausnahmegruppen",
		publisher = "Vorlesungsausarbeitung",
		address  = "Göttingen",
		year      = "1963",
	}
	
	@article {Land35,
		AUTHOR = {Landherr, Walther},
		TITLE = {\"{U}ber einfache {L}iesche {R}inge},
		JOURNAL = {Abh. Math. Sem. Univ. Hamburg},
		FJOURNAL = {Abhandlungen aus dem Mathematischen Seminar der Universit\"at
			Hamburg},
		VOLUME = {11},
		YEAR = {1935},
		NUMBER = {1},
		PAGES = {41--64},
		ISSN = {0025-5858},
		CODEN = {AMHAAJ},
		MRCLASS = {Contributed Item},
		MRNUMBER = {3069642},
		DOI = {10.1007/BF02940712},
		URL = {http://dx.doi.org/10.1007/BF02940712},
	}
	
	@incollection {Tate62,
		AUTHOR = {Tate, John},
		TITLE = {Duality theorems in {G}alois cohomology over number fields},
		BOOKTITLE = {Proc. {I}nternat. {C}ongr. {M}athematicians ({S}tockholm,
			1962)},
		PAGES = {288--295},
		PUBLISHER = {Inst. Mittag-Leffler, Djursholm},
		YEAR = {1963},
		MRCLASS = {10.66},
		MRNUMBER = {0175892},
	}
	
	@article {Kne65,
		AUTHOR = {Kneser, Martin},
		TITLE = {Starke {A}pproximation in algebraischen {G}ruppen. {I}},
		JOURNAL = {J. Reine Angew. Math.},
		FJOURNAL = {Journal f\"ur die Reine und Angewandte Mathematik},
		VOLUME = {218},
		YEAR = {1965},
		PAGES = {190--203},
		ISSN = {0075-4102},
		MRCLASS = {14.50 (20.75)},
		MRNUMBER = {0184945},
		MRREVIEWER = {T. Ono},
	}
	
	@article {Che55,
		AUTHOR = {Chevalley, C.},
		TITLE = {Sur certains groupes simples},
		JOURNAL = {T\^ohoku Math. J. (2)},
		FJOURNAL = {The Tohoku Mathematical Journal. Second Series},
		VOLUME = {7},
		YEAR = {1955},
		PAGES = {14--66},
		ISSN = {0040-8735},
		MRCLASS = {20.0X},
		MRNUMBER = {0073602},
		MRREVIEWER = {F. I. Mautner},
	}
	
	@incollection {Tits62,
		AUTHOR = {Tits, J.},
		TITLE = {Groupes simples et g\'eom\'etries associ\'ees},
		BOOKTITLE = {Proc. {I}nternat. {C}ongr. {M}athematicians ({S}tockholm,
			1962)},
		PAGES = {197--221},
		PUBLISHER = {Inst. Mittag-Leffler, Djursholm},
		YEAR = {1963},
		MRCLASS = {20.75 (50.05)},
		MRNUMBER = {0175903},
	}
	
	@article {Spr55,
		AUTHOR = {Springer, T. A.},
		TITLE = {Quadratic forms over fields with a discrete valuation. {I}.
			{E}quivalence classes of definite forms},
		JOURNAL = {Nederl. Akad. Wetensch. Proc. Ser. A. {\bf 58} = Indag. Math.},
		VOLUME = {17},
		YEAR = {1955},
		PAGES = {352--362},
		MRCLASS = {10.2X},
		MRNUMBER = {0070664},
		MRREVIEWER = {B. W. Jones},
	}
	
	@article {Jac38,
		AUTHOR = {Jacobson, N.},
		TITLE = {Simple {L}ie algebras over a field of characteristic zero},
		JOURNAL = {Duke Math. J.},
		FJOURNAL = {Duke Mathematical Journal},
		VOLUME = {4},
		YEAR = {1938},
		NUMBER = {3},
		PAGES = {534--551},
		ISSN = {0012-7094},
		CODEN = {DUMJAO},
		MRCLASS = {Contributed Item},
		MRNUMBER = {1546073},
		DOI = {10.1215/S0012-7094-38-00444-2},
		URL = {http://dx.doi.org/10.1215/S0012-7094-38-00444-2},
	}
	
	@book {SerGalCoh,
		AUTHOR = {Serre, Jean-Pierre},
		TITLE = {Cohomologie galoisienne},
		SERIES = {Cours au Coll\`ege de France},
		VOLUME = {1962},
		PUBLISHER = {Springer-Verlag, Berlin-Heidelberg-New York},
		YEAR = {1962/1963},
		PAGES = {vii+212 pp. (not consecutively paged)},
		MRCLASS = {12.50 (10.68)},
		MRNUMBER = {0180551},
		MRREVIEWER = {M. Greenberg},
	}
	
	
	@book {SerLocFie,
		AUTHOR = {Serre, Jean-Pierre},
		TITLE = {Corps locaux},
		SERIES = {Publications de l'Institut de Math\'ematique de l'Universit\'e
			de Nancago, VIII},
		PUBLISHER = {Actualit\'es Sci. Indust., No. 1296. Hermann, Paris},
		YEAR = {1962},
		PAGES = {243},
		MRCLASS = {10.67 (10.68)},
		MRNUMBER = {0150130},
		MRREVIEWER = {T. Nakayama},
	}
	
	
	
@book {Bcomm89, 
    AUTHOR = {Bourbaki, N.},
     TITLE = {\'{E}l\'ements de math\'ematique. {A}lg\`ebre commutative.
              {C}hapitres 8 et 9},
      NOTE = {Reprint of the 1983 original},
 PUBLISHER = {Springer, Berlin},
      YEAR = {2006},
     PAGES = {ii+200},
      ISBN = {978-3-540-33942-7; 3-540-33942-6},
   MRCLASS = {13-01},
  MRNUMBER = {2284892 (2007h:13001)},
}
%MR2284892,

@article {TW, 
    AUTHOR = {Tignol, J.-P. and Wadsworth, A. R.},
     TITLE = {Totally ramified valuations on finite-dimensional division
              algebras},
   JOURNAL = {Trans. Amer. Math. Soc.},
  FJOURNAL = {Transactions of the American Mathematical Society},
    VOLUME = {302},
      YEAR = {1987},
    NUMBER = {1},
     PAGES = {223--250},
      ISSN = {0002-9947},
     CODEN = {TAMTAM},
   MRCLASS = {16A39 (12E15 16A10)},
  MRNUMBER = {887507 (88j:16025)},
MRREVIEWER = {David J. Saltman},
       DOI = {10.2307/2000907},
       URL = {http://dx.doi.org/10.2307/2000907},
}
%MR887507,

@article {LU, 
    AUTHOR = {Lewis, D. W. and Unger, T.},
     TITLE = {Hermitian {M}orita theory: a matrix approach},
   JOURNAL = {Irish Math. Soc. Bull.},
  FJOURNAL = {Irish Mathematical Society Bulletin},
    NUMBER = {62},
      YEAR = {2008},
     PAGES = {37--41},
      ISSN = {0791-5578},
   MRCLASS = {16K20 (16W10)},
  MRNUMBER = {2532179 (2010g:16030)},
MRREVIEWER = {Mohammad G. Mahmoudi},
}
%MR2532179,

@book {albert, 
    AUTHOR = {Albert, A. A.},
     TITLE = {Structure of algebras},
    SERIES = {Revised printing. American Mathematical Society Colloquium
              Publications, Vol. XXIV},
 PUBLISHER = {American Mathematical Society, Providence, R.I.},
      YEAR = {1961},
     PAGES = {xi+210},
   MRCLASS = {16.50},
  MRNUMBER = {0123587 (23 \#A912)},
}

%MR0123587,

@unpublished{AdJ,
	author	= {Artin, M. and de Jong, A. J.},
	title	= {Stable orders over surfaces}, 
	year	= {2004},
	month	= {June},
	note	= {\url{http://www.math.lsa.umich.edu/courses/711/ordersms-num.pdf}},
}




@article {AB2, 
    AUTHOR = {Auslander, M. and Buchsbaum, D. A.},
     TITLE = {On ramification theory in noetherian rings},
   JOURNAL = {Amer. J. Math.},
  FJOURNAL = {American Journal of Mathematics},
    VOLUME = {81},
      YEAR = {1959},
     PAGES = {749--765},
      ISSN = {0002-9327},
   MRCLASS = {16.00 (14.00)},
  MRNUMBER = {0106929 (21 \#5659)},
MRREVIEWER = {M. Nagata},
}
%MR0106929,

@article {AB, 
    AUTHOR = {Auslander, M. and Buchsbaum, D. A.},
     TITLE = {Unique factorization in regular local rings},
   JOURNAL = {Proc. Nat. Acad. Sci. U.S.A.},
  FJOURNAL = {Proceedings of the National Academy of Sciences of the United
              States of America},
    VOLUME = {45},
      YEAR = {1959},
     PAGES = {733--734},
      ISSN = {0027-8424},
   MRCLASS = {16.00},
  MRNUMBER = {0103906 (21 \#2669)},
MRREVIEWER = {M. Nagata},
}
%MR0103906,


@article {AG, 
    AUTHOR = {Auslander, M. and Goldman, O.},
     TITLE = {The {B}rauer group of a commutative ring},
   JOURNAL = {Trans. Amer. Math. Soc.},
  FJOURNAL = {Transactions of the American Mathematical Society},
    VOLUME = {97},
      YEAR = {1960},
     PAGES = {367--409},
      ISSN = {0002-9947},
   MRCLASS = {18.00 (16.00)},
  MRNUMBER = {0121392 (22 \#12130)},
MRREVIEWER = {T. Nakayama},
}
%MR0121392,


@article {BP, 
    AUTHOR = {Bayer-Fluckiger, E. and Parimala, R.},
     TITLE = {Galois cohomology of the classical groups over fields of
              cohomological dimension {$\le 2$}},
   JOURNAL = {Invent. Math.},
  FJOURNAL = {Inventiones Mathematicae},
    VOLUME = {122},
      YEAR = {1995},
    NUMBER = {2},
     PAGES = {195--229},
      ISSN = {0020-9910},
     CODEN = {INVMBH},
   MRCLASS = {11E72 (11E39 11E57 20G10)},
  MRNUMBER = {1358975 (96i:11042)},
MRREVIEWER = {Adrian R. Wadsworth},
       DOI = {10.1007/BF01231443},
       URL = {http://dx.doi.org/10.1007/BF01231443},
}
%MR1358975,

@unpublished{B,
	author	= {Beke, S.},
	title	= {Specialisation and good reduction for algebras with involution},
	year	= {2013},

	note	= {\url{http://www.math.uni-bielefeld.de/lag/man/488.pdf}},
}
%	month	= "",



@article {BG, 
    AUTHOR = {Beke, S. and Van Geel, J.},
     TITLE = {An {I}somorphism {P}roblem for {A}zumaya {A}lgebras with
              {I}nvolution over {S}emilocal {B}\'ezout {D}omains},
   JOURNAL = {Algebr. Represent. Theory},
  FJOURNAL = {Algebras and Representation Theory},
    VOLUME = {17},
      YEAR = {2014},
    NUMBER = {6},
     PAGES = {1635--1655},
      ISSN = {1386-923X},
   MRCLASS = {16H05 (12J20 16K20 20G35)},
  MRNUMBER = {3284324},
       DOI = {10.1007/s10468-013-9463-6},
       URL = {http://dx.doi.org/10.1007/s10468-013-9463-6},
}
%MR3284324,

@book {Balg9, 
    AUTHOR = {Bourbaki, N.},
     TITLE = {\'{E}l\'ements de math\'ematique. {A}lg\`ebre. {C}hapitre 9},
      NOTE = {Reprint of the 1959 original},
 PUBLISHER = {Springer-Verlag, Berlin},
      YEAR = {2007},
     PAGES = {211},
      ISBN = {978-3-540-35338-6; 3-540-35338-0},
   MRCLASS = {15-01 (01A75)},
  MRNUMBER = {2325344 (2008f:15001)},
}
%MR2325344,

@book {Bcomm17, 
    AUTHOR = {Bourbaki, N.},
     TITLE = {Commutative algebra. {C}hapters 1--7},
    SERIES = {Elements of Mathematics (Berlin)},
      NOTE = {Translated from the French,
              Reprint of the 1989 English translation},
 PUBLISHER = {Springer-Verlag, Berlin},
      YEAR = {1998},
     PAGES = {xxiv+625},
      ISBN = {3-540-64239-0},
   MRCLASS = {13-XX (00A05)},
  MRNUMBER = {1727221 (2001g:13001)},
}
%MR1727221,

@book {Bomm, 
    AUTHOR = {Bourbaki, N.},
     TITLE = {\'{E}l\'ements de math\'ematique. {A}lg\`ebre commutative.
              {C}hapitres 8 et 9},
      NOTE = {Reprint of the 1983 original},
 PUBLISHER = {Springer, Berlin},
      YEAR = {2006},
     PAGES = {ii+200},
      ISBN = {978-3-540-33942-7; 3-540-33942-6},
   MRCLASS = {13-01},
  MRNUMBER = {2284892 (2007h:13001)},
}
%MR2284892,

@book {CF, 
    editor = {J. W. S. Cassels and A. Fr\"ohlich}, 
     TITLE = {Algebraic number theory},
    SERIES = {Proceedings of an instructional conference organized by the
              London Mathematical Society (a NATO Advanced Study Institute)
              with the support of the International Mathematical Union.
             },
 PUBLISHER = {Academic Press, London; Thompson Book Co., Inc., Washington,
              D.C.},
      YEAR = {1967},
     PAGES = {xviii+366},
   MRCLASS = {00.04 (10.00)},
  MRNUMBER = {0215665 (35 \#6500)},
}
%MR0215665,

@article {CM, 
    AUTHOR = {Chernousov, V. and Merkurjev, A.},
     TITLE = {{$R$}-equivalence and special unitary groups},
   JOURNAL = {J. Algebra},
  FJOURNAL = {Journal of Algebra},
    VOLUME = {209},
      YEAR = {1998},
    NUMBER = {1},
     PAGES = {175--198},
      ISSN = {0021-8693},
     CODEN = {JALGA4},
   MRCLASS = {20G15 (14E05 14G05 14M20)},
  MRNUMBER = {1652122 (99m:20101)},
MRREVIEWER = {Jean-Claude Douai},
       DOI = {10.1006/jabr.1998.7534},
       URL = {http://dx.doi.org/10.1006/jabr.1998.7534},
}
%MR1652122,

@article {CP, 
    AUTHOR = {Chernousov, V. I. and Platonov, V. P.},
     TITLE = {The rationality problem for semisimple group varieties},
   JOURNAL = {J. Reine Angew. Math.},
  FJOURNAL = {Journal f\"ur die Reine und Angewandte Mathematik},
    VOLUME = {504},
      YEAR = {1998},
     PAGES = {1--28},
      ISSN = {0075-4102},
     CODEN = {JRMAA8},
   MRCLASS = {14L35 (14M20 20G15)},
  MRNUMBER = {1656830 (99i:14056)},
MRREVIEWER = {Andy R. Magid},
}
%MR1656830,

@article {CTPS, 
    AUTHOR = {Colliot-Th{\'e}l{\`e}ne, J.-L. and Parimala, R. and
              Suresh, V.},
     TITLE = {Patching and local-global principles for homogeneous spaces
              over function fields of {$p$}-adic curves},
   JOURNAL = {Comment. Math. Helv.},
  FJOURNAL = {Commentarii Mathematici Helvetici. A Journal of the Swiss
              Mathematical Society},
    VOLUME = {87},
      YEAR = {2012},
    NUMBER = {4},
     PAGES = {1011--1033},
      ISSN = {0010-2571},
   MRCLASS = {11G99 (11E12 11E72 14G05 14G20 20G35)},
  MRNUMBER = {2984579},
MRREVIEWER = {Jan van Geel},
       DOI = {10.4171/CMH/276},
       URL = {http://dx.doi.org/10.4171/CMH/276},
}
%MR2984579,

@article {CTS, 
    AUTHOR = {Colliot-Th{\'e}l{\`e}ne, J.-L. and Sansuc, J.-J.},
     TITLE = {Fibr\'es quadratiques et composantes connexes r\'eelles},
   JOURNAL = {Math. Ann.},
  FJOURNAL = {Mathematische Annalen},
    VOLUME = {244},
      YEAR = {1979},
    NUMBER = {2},
     PAGES = {105--134},
      ISSN = {0025-5831},
     CODEN = {MAANA3},
   MRCLASS = {14G05 (10C04)},
  MRNUMBER = {550842 (81c:14010)},
MRREVIEWER = {Daniel Coray},
       DOI = {10.1007/BF01420486},
       URL = {http://dx.doi.org/10.1007/BF01420486},
}
%MR550842,

@book {EKM, 
    AUTHOR = {Elman, R. and Karpenko, N. and Merkurjev, A.},
     TITLE = {The algebraic and geometric theory of quadratic forms},
    SERIES = {American Mathematical Society Colloquium Publications},
    VOLUME = {56},
 PUBLISHER = {American Mathematical Society, Providence, RI},
      YEAR = {2008},
     PAGES = {viii+435},
      ISBN = {978-0-8218-4329-1},
   MRCLASS = {11Exx (11-02 11E04 11E81 14C15 14C25)},
  MRNUMBER = {2427530 (2009d:11062)},
MRREVIEWER = {Andrzej S{\l}adek},
}
%MR2427530,

@article {EGA1, 
    AUTHOR = {Grothendieck, A.},
     TITLE = {\'{E}l\'ements de g\'eom\'etrie alg\'ebrique. {I}. {L}e
              langage des sch\'emas},
   JOURNAL = {Inst. Hautes \'Etudes Sci. Publ. Math.},
  FJOURNAL = {Institut des Hautes \'Etudes Scientifiques. Publications
              Math\'ematiques},
    NUMBER = {4},
      YEAR = {1960},
     PAGES = {228},
      ISSN = {0073-8301},
   MRCLASS = {14.55},
  MRNUMBER = {0217083 (36 \#177a)},
}
%MR0217083,

@article {EGA4.2, 
    AUTHOR = {Grothendieck, A.},
     TITLE = {\'{E}l\'ements de g\'eom\'etrie alg\'ebrique. {IV}. \'{E}tude
              locale des sch\'emas et des morphismes de sch\'emas. {II}},
   JOURNAL = {Inst. Hautes \'Etudes Sci. Publ. Math.},
  FJOURNAL = {Institut des Hautes \'Etudes Scientifiques. Publications
              Math\'ematiques},
    NUMBER = {24},
      YEAR = {1965},
     PAGES = {231},
      ISSN = {0073-8301},
   MRCLASS = {14.00},
  MRNUMBER = {0199181 (33 \#7330)},
MRREVIEWER = {H. Hironaka},
}
%MR0199181,

@article {EGA4.3, 
    AUTHOR = {Grothendieck, A.},
     TITLE = {\'{E}l\'ements de g\'eom\'etrie alg\'ebrique. {IV}. \'{E}tude
              locale des sch\'emas et des morphismes de sch\'emas. {III}},
   JOURNAL = {Inst. Hautes \'Etudes Sci. Publ. Math.},
  FJOURNAL = {Institut des Hautes \'Etudes Scientifiques. Publications
              Math\'ematiques},
    NUMBER = {28},
      YEAR = {1966},
     PAGES = {255},
      ISSN = {0073-8301},
   MRCLASS = {14.55},
  MRNUMBER = {0217086 (36 \#178)},
MRREVIEWER = {J. P. Murre},
}
%MR0217086,

@article {EGA4.4, 
    AUTHOR = {Grothendieck, A.},
     TITLE = {\'{E}l\'ements de g\'eom\'etrie alg\'ebrique. {IV}. \'{E}tude
              locale des sch\'emas et des morphismes de sch\'emas {IV}},
   JOURNAL = {Inst. Hautes \'Etudes Sci. Publ. Math.},
  FJOURNAL = {Institut des Hautes \'Etudes Scientifiques. Publications
              Math\'ematiques},
    NUMBER = {32},
      YEAR = {1967},
     PAGES = {361},
      ISSN = {0073-8301},
   MRCLASS = {14.55},
  MRNUMBER = {0238860 (39 \#220)},
MRREVIEWER = {J. P. Murre},
}
%MR0238860,

@book {SGA1, 
     TITLE = {Rev\^etements \'etales et groupe fondamental},
      NOTE = {S{\'e}minaire de G{\'e}om{\'e}trie Alg{\'e}brique du Bois
              Marie 1960--1961 (SGA 1),
              Dirig{\'e} par Alexandre Grothendieck. Augment{\'e} de deux
              expos{\'e}s de M. Raynaud,
              Lecture Notes in Mathematics, Vol. 224},
 PUBLISHER = {Springer-Verlag, Berlin-New York},
      YEAR = {1971},
     PAGES = {xxii+447},
   MRCLASS = {14-06 (14E20)},
  MRNUMBER = {0354651 (50 \#7129)},
}
%MR0354651,

@book {GS, 
    AUTHOR = {Gille, Philippe and Szamuely, Tam{\'a}s},
     TITLE = {Central simple algebras and {G}alois cohomology},
    SERIES = {Cambridge Studies in Advanced Mathematics},
    VOLUME = {101},
 PUBLISHER = {Cambridge University Press, Cambridge},
      YEAR = {2006},
     PAGES = {xii+343},
      ISBN = {978-0-521-86103-8; 0-521-86103-9},
   MRCLASS = {16K20 (14F22 19C30)},
  MRNUMBER = {2266528 (2007k:16033)},
MRREVIEWER = {Gr{\'e}gory Berhuy},
       DOI = {10.1017/CBO9780511607219},
       URL = {http://dx.doi.org/10.1017/CBO9780511607219},
}
%MR2266528,

@article {HHK4, 
    AUTHOR = {Harbater, D. and Hartmann, J. and Krashen, D.},
     TITLE = {Local-global principles for {G}alois cohomology},
   JOURNAL = {Comment. Math. Helv.},
  FJOURNAL = {Commentarii Mathematici Helvetici. A Journal of the Swiss
              Mathematical Society},
    VOLUME = {89},
      YEAR = {2014},
    NUMBER = {1},
     PAGES = {215--253},
      ISSN = {0010-2571},
   MRCLASS = {11E72 (12G05 13F25 14H25 20G15)},
  MRNUMBER = {3177913},
MRREVIEWER = {Skip Garibaldi},
       DOI = {10.4171/CMH/317},
       URL = {http://dx.doi.org/10.4171/CMH/317},
}
%MR3177913,

@article {HHK3, 
    AUTHOR = {Harbater, D. and Hartmann, J. and Krashen, D.},
     TITLE = {Weierstrass preparation and algebraic invariants},
   JOURNAL = {Math. Ann.},
  FJOURNAL = {Mathematische Annalen},
    VOLUME = {356},
      YEAR = {2013},
    NUMBER = {4},
     PAGES = {1405--1424},
      ISSN = {0025-5831},
   MRCLASS = {13J10 (11E04 14H25 16K50 16W60 32B05)},
  MRNUMBER = {3072806},
MRREVIEWER = {Rosario Strano},
       DOI = {10.1007/s00208-012-0888-8},
       URL = {http://dx.doi.org/10.1007/s00208-012-0888-8},
}
%MR3072806,

@article {HHK, 
    AUTHOR = {Harbater, D. and Hartmann, J. and Krashen, D.},
     TITLE = {Patching subfields of division algebras},
   JOURNAL = {Trans. Amer. Math. Soc.},
  FJOURNAL = {Transactions of the American Mathematical Society},
    VOLUME = {363},
      YEAR = {2011},
    NUMBER = {6},
     PAGES = {3335--3349},
      ISSN = {0002-9947},
     CODEN = {TAMTAM},
   MRCLASS = {12F12 (16K20 16K50 16S35)},
  MRNUMBER = {2775810 (2012c:12008)},
MRREVIEWER = {Ivan D. Chipchakov},
       DOI = {10.1090/S0002-9947-2010-05229-8},
       URL = {http://dx.doi.org/10.1090/S0002-9947-2010-05229-8},
}
%MR2775810,

@article {HH, 
    AUTHOR = {Harbater, D. and Hartmann, J.},
     TITLE = {Patching over fields},
   JOURNAL = {Israel J. Math.},
  FJOURNAL = {Israel Journal of Mathematics},
    VOLUME = {176},
      YEAR = {2010},
     PAGES = {61--107},
      ISSN = {0021-2172},
     CODEN = {ISJMAP},
   MRCLASS = {12F10 (16K50)},
  MRNUMBER = {2653187 (2012a:12009)},
MRREVIEWER = {Montserrat Vela},
       DOI = {10.1007/s11856-010-0021-1},
       URL = {http://dx.doi.org/10.1007/s11856-010-0021-1},
}
%MR2653187,

@article {HHK1, 
    AUTHOR = {Harbater, D. and Hartmann, J. and Krashen, D.},
     TITLE = {Applications of patching to quadratic forms and central simple
              algebras},
   JOURNAL = {Invent. Math.},
  FJOURNAL = {Inventiones Mathematicae},
    VOLUME = {178},
      YEAR = {2009},
    NUMBER = {2},
     PAGES = {231--263},
      ISSN = {0020-9910},
     CODEN = {INVMBH},
   MRCLASS = {11E04 (16K20)},
  MRNUMBER = {2545681 (2010j:11058)},
MRREVIEWER = {Mohammad G. Mahmoudi},
       DOI = {10.1007/s00222-009-0195-5},
       URL = {http://dx.doi.org/10.1007/s00222-009-0195-5},
}
%MR2545681,

@unpublished{HHK2,
	author	= {Harbater, D. and Hartmann, J. and Krashen, D.},
	title	= {Local-global principles for torsors over arithmetic curves},
	year	= {2011},
	month	= {August},
	note	= {arXiv:1108.3323v4},
}


@article {JW, 
    AUTHOR = {Jacob, B. and Wadsworth, A.},
     TITLE = {Division algebras over {H}enselian fields},
   JOURNAL = {J. Algebra},
  FJOURNAL = {Journal of Algebra},
    VOLUME = {128},
      YEAR = {1990},
    NUMBER = {1},
     PAGES = {126--179},
      ISSN = {0021-8693},
     CODEN = {JALGA4},
   MRCLASS = {12E15 (12J10 12J20 16K20)},
  MRNUMBER = {1031915 (91d:12006)},
MRREVIEWER = {Jean-Pierre Tignol},
       DOI = {10.1016/0021-8693(90)90047-R},
       URL = {http://dx.doi.org/10.1016/0021-8693(90)90047-R},
}
%MR1031915,

@article {Kar, 
    AUTHOR = {Karpenko, N. A.},
     TITLE = {Cohomology of relative cellular spaces and of isotropic flag
              varieties},
   JOURNAL = {Algebra i Analiz},
  FJOURNAL = {Rossi\u\i skaya Akademiya Nauk. Algebra i Analiz},
    VOLUME = {12},
      YEAR = {2000},
    NUMBER = {1},
     PAGES = {3--69},
      ISSN = {0234-0852},
   MRCLASS = {14M15 (57T15)},
  MRNUMBER = {1758562 (2001c:14076)},
MRREVIEWER = {I. Dolgachev},
}
%MR1758562,

@book {Kneser, 
    AUTHOR = {Kneser, M.},
     TITLE = {Lectures on {G}alois cohomology of classical groups},
      NOTE = {With an appendix by T. A. Springer,
              Notes by P. Jothilingam,
              Tata Institute of Fundamental Research Lectures on
              Mathematics, No. 47},
 PUBLISHER = {Tata Institute of Fundamental Research, Bombay},
      YEAR = {1969},
     PAGES = {ii+158},
   MRCLASS = {20G10 (12A60)},
  MRNUMBER = {0340440 (49 \#5195)},
MRREVIEWER = {R. Steinberg},
}
%MR0340440,

@book {inv, 
    AUTHOR = {Knus, M.-A. and Merkurjev, A. and Rost, M. and
              Tignol, J.-P.},
     TITLE = {The book of involutions},
    SERIES = {American Mathematical Society Colloquium Publications},
    VOLUME = {44},
      NOTE = {With a preface in French by J. Tits},
 PUBLISHER = {American Mathematical Society, Providence, RI},
      YEAR = {1998},
     PAGES = {xxii+593},
      ISBN = {0-8218-0904-0},
   MRCLASS = {16K20 (11E39 11E57 11E72 11E88 16W10 20G10)},
  MRNUMBER = {1632779 (2000a:16031)},
MRREVIEWER = {A. R. Wadsworth},
}
%MR1632779,

@book {Knus, 
    AUTHOR = {Knus, M.-A.},
     TITLE = {Quadratic and {H}ermitian forms over rings},
    SERIES = {Grundlehren der Mathematischen Wissenschaften [Fundamental
              Principles of Mathematical Sciences]},
    VOLUME = {294},
      NOTE = {With a foreword by I. Bertuccioni},
 PUBLISHER = {Springer-Verlag, Berlin},
      YEAR = {1991},
     PAGES = {xii+524},
      ISBN = {3-540-52117-8},
   MRCLASS = {11Exx (11E39 11E81 16E20 19Gxx)},
  MRNUMBER = {1096299 (92i:11039)},
MRREVIEWER = {Rudolf Scharlau},
       DOI = {10.1007/978-3-642-75401-2},
       URL = {http://dx.doi.org/10.1007/978-3-642-75401-2},
}
%MR1096299,

@article {L1, 
    AUTHOR = {Larmour, D. W.},
     TITLE = {A {S}pringer theorem for {H}ermitian forms},
   JOURNAL = {Math. Z.},
  FJOURNAL = {Mathematische Zeitschrift},
    VOLUME = {252},
      YEAR = {2006},
    NUMBER = {3},
     PAGES = {459--472},
      ISSN = {0025-5874},
     CODEN = {MAZEAX},
   MRCLASS = {16K20 (13A18)},
  MRNUMBER = {2207754 (2007b:16041)},
MRREVIEWER = {Patrick J. Morandi},
       DOI = {10.1007/s00209-005-0775-z},
       URL = {http://dx.doi.org/10.1007/s00209-005-0775-z},
}
%MR2207754,

@book {L2, 
    AUTHOR = {Larmour, D. W.},
     TITLE = {A {S}pringer theorem for {H}ermitian forms and involutions},
      NOTE = {Thesis (Ph.D.)--New Mexico State University},
 PUBLISHER = {ProQuest LLC, Ann Arbor, MI},
      YEAR = {1999},
     PAGES = {79},
      ISBN = {978-0599-52417-0},
   MRCLASS = {Thesis},
  MRNUMBER = {2699972},
       URL = {http://gateway.proquest.com/openurl?url_ver=Z39.88-2004&rft_val_fmt=info:ofi/fmt:kev:mtx:dissertation&res_dat=xri:pqdiss&rft_dat=xri:pqdiss:9949220},
}
%MR2699972,

@incollection {Lip75, 
    AUTHOR = {Lipman, J.},
     TITLE = {Introduction to resolution of singularities},
 BOOKTITLE = {Algebraic geometry ({P}roc. {S}ympos. {P}ure {M}ath., {V}ol.
              29, {H}umboldt {S}tate {U}niv., {A}rcata, {C}alif., 1974)},
     PAGES = {187--230},
 PUBLISHER = {Amer. Math. Soc., Providence, R.I.},
      YEAR = {1975},
   MRCLASS = {14E15 (14J15)},
  MRNUMBER = {0389901 (52 \#10730)},
MRREVIEWER = {Jean Giraud},
}
%MR0389901,

@article {Lip78, 
    AUTHOR = {Lipman, J.},
     TITLE = {Desingularization of two-dimensional schemes},
   JOURNAL = {Ann. Math. (2)},
    VOLUME = {107},
      YEAR = {1978},
    NUMBER = {1},
     PAGES = {151--207},
   MRCLASS = {14J10},
  MRNUMBER = {0491722 (58 \#10924)},
MRREVIEWER = {A. H. Wallace},
}
%MR0491722,

@book {Liu, 
    AUTHOR = {Liu, Q.},
     TITLE = {Algebraic geometry and arithmetic curves},
    SERIES = {Oxford Graduate Texts in Mathematics},
    VOLUME = {6},
      NOTE = {Translated from the French by Reinie Ern{\'e},
              Oxford Science Publications},
 PUBLISHER = {Oxford University Press, Oxford},
      YEAR = {2002},
     PAGES = {xvi+576},
      ISBN = {0-19-850284-2},
   MRCLASS = {14-01 (11G30 14A05 14A15 14Gxx 14Hxx)},
  MRNUMBER = {1917232 (2003g:14001)},
MRREVIEWER = {C{\'{\i}}cero Fernandes de Carvalho},
}
%MR1917232,

@article {Mer, 
    AUTHOR = {Merkurjev, A. S.},
     TITLE = {{$R$}-equivalence and rationality problem for semisimple
              adjoint classical algebraic groups},
   JOURNAL = {Inst. Hautes \'Etudes Sci. Publ. Math.},
  FJOURNAL = {Institut des Hautes \'Etudes Scientifiques. Publications
              Math\'ematiques},
    NUMBER = {84},
      YEAR = {1996},
     PAGES = {189--213 (1997)},
      ISSN = {0073-8301},
     CODEN = {PMIHA6},
   MRCLASS = {14L10 (14M20 16K20 20G15)},
  MRNUMBER = {1441008 (98d:14055)},
MRREVIEWER = {B. Sury},
       URL = {http://www.numdam.org/item?id=PMIHES_1996__84__189_0},
}
%MR1441008,

@article {M, 
    AUTHOR = {Morandi, P.},
     TITLE = {The {H}enselization of a valued division algebra},
   JOURNAL = {J. Algebra},
  FJOURNAL = {Journal of Algebra},
    VOLUME = {122},
      YEAR = {1989},
    NUMBER = {1},
     PAGES = {232--243},
      ISSN = {0021-8693},
     CODEN = {JALGA4},
   MRCLASS = {12E15 (12J20 16A39)},
  MRNUMBER = {994945 (90h:12007)},
MRREVIEWER = {Jean-Pierre Tignol},
       DOI = {10.1016/0021-8693(89)90247-0},
       URL = {http://dx.doi.org/10.1016/0021-8693(89)90247-0},
}
 %MR994945,

@article {nag,
    AUTHOR = {Nagata, M.},
     TITLE = {A general theory of algebraic geometry over {D}edekind
              domains. {II}. {S}eparably generated extensions and regular
              local rings},
   JOURNAL = {Amer. J. Math.},
  FJOURNAL = {American Journal of Mathematics},
    VOLUME = {80},
      YEAR = {1958},
     PAGES = {382--420},
      ISSN = {0002-9327},
   MRCLASS = {13.00 (14.00)},
  MRNUMBER = {0094344 (20 \#862)},
MRREVIEWER = {P. Samuel},
}
 %MR0094344,

@article {O, 
    AUTHOR = {Oukhtite, L.},
     TITLE = {Witt group of {H}ermitian forms over a noncommutative discrete
              valuation ring},
   JOURNAL = {Int. J. Math. Math. Sci.},
  FJOURNAL = {International Journal of Mathematics and Mathematical
              Sciences},
      YEAR = {2005},
    NUMBER = {7},
     PAGES = {1141--1147},
      ISSN = {0161-1712},
   MRCLASS = {11E81},
  MRNUMBER = {2172991 (2006g:11082)},
MRREVIEWER = {Marius M. Somodi},
       DOI = {10.1155/IJMMS.2005.1141},
       URL = {http://dx.doi.org/10.1155/IJMMS.2005.1141},
}
%MR2172991,

@article {PSS, 
    AUTHOR = {Parimala, R. and Sridharan, R. and Suresh, V.},
     TITLE = {Hermitian analogue of a theorem of {S}pringer},
   JOURNAL = {J. Algebra},
  FJOURNAL = {Journal of Algebra},
    VOLUME = {243},
      YEAR = {2001},
    NUMBER = {2},
     PAGES = {780--789},
      ISSN = {0021-8693},
     CODEN = {JALGA4},
   MRCLASS = {11E39},
  MRNUMBER = {1850658 (2002g:11043)},
MRREVIEWER = {Jean-Pierre Tignol},
       DOI = {10.1006/jabr.2001.8830},
       URL = {http://dx.doi.org/10.1006/jabr.2001.8830},
}
%MR1850658,

@article {RS, 
    AUTHOR = {Reddy, B. S. and Suresh, V.},
     TITLE = {Admissibility of groups over function fields of p-adic curves},
   JOURNAL = {Adv. Math.},
  FJOURNAL = {Advances in Mathematics},
    VOLUME = {237},
      YEAR = {2013},
     PAGES = {316--330},
      ISSN = {0001-8708},
   MRCLASS = {20D20 (14H25 16Kxx)},
  MRNUMBER = {3028580},
MRREVIEWER = {Zinovy Reichstein},
       DOI = {10.1016/j.aim.2012.12.017},
       URL = {http://dx.doi.org/10.1016/j.aim.2012.12.017},
}
%MR3028580,

@book {Rei, 
    AUTHOR = {Reiner, I.},
     TITLE = {Maximal orders},
    SERIES = {London Mathematical Society Monographs. New Series},
    VOLUME = {28},
      NOTE = {Corrected reprint of the 1975 original,
              With a foreword by M. J. Taylor},
 PUBLISHER = {The Clarendon Press, Oxford University Press, Oxford},
      YEAR = {2003},
     PAGES = {xiv+395},
      ISBN = {0-19-852673-3},
   MRCLASS = {16H05 (11R54 16K20)},
  MRNUMBER = {1972204 (2004c:16026)},
}
%MR1972204,

@article {Sal1, 
    AUTHOR = {Saltman, D. J.},
     TITLE = {Division algebras over {$p$}-adic curves},
   JOURNAL = {J. Ramanujan Math. Soc.},
  FJOURNAL = {Journal of the Ramanujan Mathematical Society},
    VOLUME = {12},
      YEAR = {1997},
    NUMBER = {1},
     PAGES = {25--47},
      ISSN = {0970-1249},
   MRCLASS = {16H05 (12E15 13A20 16K20)},
  MRNUMBER = {1462850 (98d:16032)},
MRREVIEWER = {Timothy J. Ford},
}
%MR1462850,

@article {Sal1.5, 
    AUTHOR = {Saltman, D. J.},
     TITLE = {Correction to: ``{D}ivision algebras over {$p$}-adic curves''
              [{J}. {R}amanujan {M}ath. {S}oc. {\bf 12} (1997), no. 1,
              25--47; {MR}1462850 (98d:16032)]},
   JOURNAL = {J. Ramanujan Math. Soc.},
  FJOURNAL = {Journal of the Ramanujan Mathematical Society},
    VOLUME = {13},
      YEAR = {1998},
    NUMBER = {2},
     PAGES = {125--129},
      ISSN = {0970-1249},
   MRCLASS = {16H05 (12E15 13A20 16K20)},
  MRNUMBER = {1666362 (99k:16036)},
MRREVIEWER = {Timothy J. Ford},
}
%MR1666362,

@book {Sal0, 
    AUTHOR = {Saltman, D. J.},
     TITLE = {Lectures on division algebras},
    SERIES = {CBMS Regional Conference Series in Mathematics},
    VOLUME = {94},
 PUBLISHER = {Published by American Mathematical Society, Providence, RI; on
              behalf of Conference Board of the Mathematical Sciences,
              Washington, DC},
      YEAR = {1999},
     PAGES = {viii+120},
      ISBN = {0-8218-0979-2},
   MRCLASS = {16K20 (12G05 16H05 16S35)},
  MRNUMBER = {1692654 (2000f:16023)},
MRREVIEWER = {Patrick J. Morandi},
}
%MR1692654,

@article {Sal3, 
    AUTHOR = {Saltman, D. J.},
     TITLE = {Cyclic algebras over {$p$}-adic curves},
   JOURNAL = {J. Algebra},
  FJOURNAL = {Journal of Algebra},
    VOLUME = {314},
      YEAR = {2007},
    NUMBER = {2},
     PAGES = {817--843},
      ISSN = {0021-8693},
     CODEN = {JALGA4},
   MRCLASS = {16K20 (11S15 16K50)},
  MRNUMBER = {2344586 (2008i:16018)},
MRREVIEWER = {Jan van Geel},
       DOI = {10.1016/j.jalgebra.2007.03.003},
       URL = {http://dx.doi.org/10.1016/j.jalgebra.2007.03.003},
}
%MR2344586,

@article {Sam, 
    AUTHOR = {Samuel, P.},
     TITLE = {Anneaux gradu\'es factoriels et modules r\'eflexifs},
   JOURNAL = {Bull. Soc. Math. France},
  FJOURNAL = {Bulletin de la Soci\'et\'e Math\'ematique de France},
    VOLUME = {92},
      YEAR = {1964},
     PAGES = {237--249},
      ISSN = {0037-9484},
   MRCLASS = {13.15 (16.00)},
  MRNUMBER = {0186702 (32 \#4160)},
MRREVIEWER = {M. Nagata},
}
%MR0186702,

@book {Sch, 
    AUTHOR = {Scharlau, W.},
     TITLE = {Quadratic and {H}ermitian forms},
    SERIES = {Grundlehren der Mathematischen Wissenschaften [Fundamental
              Principles of Mathematical Sciences]},
    VOLUME = {270},
 PUBLISHER = {Springer-Verlag, Berlin},
      YEAR = {1985},
     PAGES = {x+421},
      ISBN = {3-540-13724-6},
   MRCLASS = {11Exx (11-02 12D15 15A63 16A16 16A28)},
  MRNUMBER = {770063 (86k:11022)},
MRREVIEWER = {R. Ware},
       DOI = {10.1007/978-3-642-69971-9},
       URL = {http://dx.doi.org/10.1007/978-3-642-69971-9},
}
%MR770063,

@article {Sch2, 
    AUTHOR = {Scharlau, W.},
     TITLE = {Klassifikation hermitescher {F}ormen \"uber lokalen
              {K}\"orpern},
   JOURNAL = {Math. Ann.},
  FJOURNAL = {Mathematische Annalen},
    VOLUME = {186},
      YEAR = {1970},
     PAGES = {201--208},
      ISSN = {0025-5831},
   MRCLASS = {16.99 (15.00)},
  MRNUMBER = {0263874 (41 \#8473)},
MRREVIEWER = {C. F. Moppert},
}
%MR0263874,

@book {Ser, 
    AUTHOR = {Serre, J.-P.},
     TITLE = {Local fields},
    SERIES = {Graduate Texts in Mathematics},
    VOLUME = {67},
      NOTE = {Translated from the French by Marvin Jay Greenberg},
 PUBLISHER = {Springer-Verlag, New York-Berlin},
      YEAR = {1979},
     PAGES = {viii+241},
      ISBN = {0-387-90424-7},
   MRCLASS = {12Bxx},
  MRNUMBER = {554237 (82e:12016)},
}
%MR554237,

@book {Spr,
    AUTHOR = {Springer, T. A.},
     TITLE = {Linear algebraic groups},
    SERIES = {Progress in Mathematics},
    VOLUME = {9},
   EDITION = {Second},
 PUBLISHER = {Birkh\"auser Boston, Inc., Boston, MA},
      YEAR = {1998},
     PAGES = {xiv+334},
      ISBN = {0-8176-4021-5},
   MRCLASS = {20G15 (14L10)},
  MRNUMBER = {1642713 (99h:20075)},
MRREVIEWER = {Andy R. Magid},
       DOI = {10.1007/978-0-8176-4840-4},
       URL = {http://dx.doi.org/10.1007/978-0-8176-4840-4},
}
 %MR1642713,

@incollection {W2, 
    AUTHOR = {Wadsworth, A. R.},
     TITLE = {Valuations on tensor products of symbol algebras},
 BOOKTITLE = {Azumaya algebras, actions, and modules ({B}loomington, {IN},
              1990)},
    SERIES = {Contemp. Math.},
    VOLUME = {124},
     PAGES = {275--289},
 PUBLISHER = {Amer. Math. Soc., Providence, RI},
      YEAR = {1992},
   MRCLASS = {16K20 (16W60)},
  MRNUMBER = {1144041 (93a:16015)},
MRREVIEWER = {Jean-Pierre Tignol},
       DOI = {10.1090/conm/124/1144041},
       URL = {http://dx.doi.org/10.1090/conm/124/1144041},
}
%MR1144041,

@article {MPW1, 
    AUTHOR = {Merkurjev, A. S. and Panin, I. A. and Wadsworth, A. R.},
     TITLE = {Index reduction formulas for twisted flag varieties. {I}},
   JOURNAL = {$K$-Theory},
  FJOURNAL = {$K$-Theory. An Interdisciplinary Journal for the Development,
              Application, and Influence of $K$-Theory in the Mathematical
              Sciences},
    VOLUME = {10},
      YEAR = {1996},
    NUMBER = {6},
     PAGES = {517--596},
      ISSN = {0920-3036},
     CODEN = {KTHEEO},
   MRCLASS = {16K20 (14L30 19E08 20G05)},
  MRNUMBER = {1415325 (98c:16018)},
MRREVIEWER = {Jean-Pierre Tignol},
       DOI = {10.1007/BF00537543},
       URL = {http://dx.doi.org/10.1007/BF00537543},
}
%MR1415325,

@article {MPW2, 
    AUTHOR = {Merkurjev, A. S. and Panin, I. A. and Wadsworth, A. R.},
     TITLE = {Index reduction formulas for twisted flag varieties. {II}},
   JOURNAL = {$K$-Theory},
  FJOURNAL = {$K$-Theory. An Interdisciplinary Journal for the Development,
              Application, and Influence of $K$-Theory in the Mathematical
              Sciences},
    VOLUME = {14},
      YEAR = {1998},
    NUMBER = {2},
     PAGES = {101--196},
      ISSN = {0920-3036},
     CODEN = {KTHEEO},
   MRCLASS = {16K20 (14L30 19E08 20G05)},
  MRNUMBER = {1628279 (99k:16037)},
MRREVIEWER = {Jean-Pierre Tignol},
       DOI = {10.1023/A:1007793218556},
       URL = {http://dx.doi.org/10.1023/A:1007793218556},
}
%MR1628279,

@incollection {Tits, 
    AUTHOR = {Tits, J.},
     TITLE = {Classification of algebraic semisimple groups},
 BOOKTITLE = {Algebraic {G}roups and {D}iscontinuous {S}ubgroups ({P}roc.
              {S}ympos. {P}ure {M}ath., {B}oulder, {C}olo., 1965)},
     PAGES = {33--62},
 PUBLISHER = {Amer. Math. Soc., Providence, R.I., 1966},
      YEAR = {1966},
   MRCLASS = {20.27},
  MRNUMBER = {0224710 (37 \#309)},
MRREVIEWER = {R. Steinberg},
}
%MR0224710,

@article {COP, 
    AUTHOR = {Colliot-Th{\'e}l{\`e}ne, J.-L. and Gille, P. and Parimala, R.},
     TITLE = {Arithmetic of linear algebraic groups over 2-dimensional
              geometric fields},
   JOURNAL = {Duke Math. J.},
  FJOURNAL = {Duke Mathematical Journal},
    VOLUME = {121},
      YEAR = {2004},
    NUMBER = {2},
     PAGES = {285--341},
      ISSN = {0012-7094},
     CODEN = {DUMJAO},
   MRCLASS = {11E76 (14G35 20G35)},
  MRNUMBER = {2034644 (2005f:11063)},
       DOI = {10.1215/S0012-7094-04-12124-4},
       URL = {http://dx.doi.org/10.1215/S0012-7094-04-12124-4},
}
%MR2034644,

@inproceedings{C9, 
	author		= {Chevalley, Claude},
	title		= {Le normalisateur d'un groupe de Borel},
	booktitle	= {S\'eminaire Claude Chevalley},
	volume		= {1},
	number		= {9},
	year		= {1956-1958},
}

@inproceedings{C17, 
	author		= {Chevalley, Claude},
	title		= {Les sous-groupes radiciel},
	booktitle	= {S\'eminaire Claude Chevalley},
	volume		= {2},
	number		= {17},
	year		= {1956-1958},
}

@book {SGA3.1, 
     TITLE = {Sch\'emas en groupes. {I}: {P}ropri\'et\'es g\'en\'erales des
              sch\'emas en groupes},
    SERIES = {S\'eminaire de G\'eom\'etrie Alg\'ebrique du Bois Marie
              1962/64 (SGA 3). Dirig\'e par M. Demazure et A. Grothendieck.
              Lecture Notes in Mathematics, Vol. 151},
 PUBLISHER = {Springer-Verlag, Berlin-New York},
      YEAR = {1970},
     PAGES = {xv+564},
   MRCLASS = {14.50},
  MRNUMBER = {0274458 (43 \#223a)},
}
%MR0274458,

@book {SGA3.2, 
     TITLE = {Sch\'emas en groupes. {II}: {G}roupes de type multiplicatif,
              et structure des sch\'emas en groupes g\'en\'eraux},
    SERIES = {S\'eminaire de G\'eom\'etrie Alg\'ebrique du Bois Marie
              1962/64 (SGA 3). Dirig\'e par M. Demazure et A. Grothendieck.
              Lecture Notes in Mathematics, Vol. 152},
 PUBLISHER = {Springer-Verlag, Berlin-New York},
      YEAR = {1970},
     PAGES = {ix+654},
   MRCLASS = {14.50},
  MRNUMBER = {0274459 (43 \#223b)},
}
%MR0274459,

@book {SGA3.3, 
     TITLE = {Sch\'emas en groupes. {III}: {S}tructure des sch\'emas en
              groupes r\'eductifs},
    SERIES = {S\'eminaire de G\'eom\'etrie Alg\'ebrique du Bois Marie
              1962/64 (SGA 3). Dirig\'e par M. Demazure et A. Grothendieck.
              Lecture Notes in Mathematics, Vol. 153},
 PUBLISHER = {Springer-Verlag, Berlin-New York},
      YEAR = {1970},
     PAGES = {viii+529},
   MRCLASS = {14.50},
  MRNUMBER = {0274460 (43 \#223c)},
}
%MR0274460,

@article {BT, 
    AUTHOR = {Borel, Armand and Tits, Jacques},
     TITLE = {Compl\'ements \`a l'article: ``{G}roupes r\'eductifs''},
   JOURNAL = {Inst. Hautes \'Etudes Sci. Publ. Math.},
  FJOURNAL = {Institut des Hautes \'Etudes Scientifiques. Publications
              Math\'ematiques},
    NUMBER = {41},
      YEAR = {1972},
     PAGES = {253--276},
      ISSN = {0073-8301},
   MRCLASS = {20G15 (14L15 22E20)},
  MRNUMBER = {0315007 (47 \#3556)},
MRREVIEWER = {F. D. Veldkamp},
}
%MR0315007,

@book {Borel, 
    AUTHOR = {Borel, Armand},
     TITLE = {Linear algebraic groups},
    SERIES = {Graduate Texts in Mathematics},
    VOLUME = {126},
   EDITION = {Second},
 PUBLISHER = {Springer-Verlag, New York},
      YEAR = {1991},
     PAGES = {xii+288},
      ISBN = {0-387-97370-2},
   MRCLASS = {20-01 (20Gxx)},
  MRNUMBER = {1102012 (92d:20001)},
MRREVIEWER = {F. D. Veldkamp},
       DOI = {10.1007/978-1-4612-0941-6},
       URL = {http://dx.doi.org/10.1007/978-1-4612-0941-6},
}
%MR1102012,

@book {Balg8, 
    AUTHOR = {Bourbaki, N.},
     TITLE = {\'{E}l\'ements de math\'ematique. {A}lg\`ebre. {C}hapitre 8.
              {M}odules et anneaux semi-simples},
      NOTE = {Second revised edition of the 1958 edition [MR0098114]},
 PUBLISHER = {Springer, Berlin},
      YEAR = {2012},
     PAGES = {x+489},
      ISBN = {978-3-540-35315-7; 978-3-540-35316-4},
   MRCLASS = {16-01},
  MRNUMBER = {3027127},
       DOI = {10.1007/978-3-540-35316-4},
       URL = {http://dx.doi.org/10.1007/978-3-540-35316-4},
}
%MR3027127,

@article {albert32, 
    AUTHOR = {Albert, A. Adrian},
     TITLE = {Normal division algebras of degree four over an algebraic
              field},
   JOURNAL = {Trans. Amer. Math. Soc.},
  FJOURNAL = {Transactions of the American Mathematical Society},
    VOLUME = {34},
      YEAR = {1932},
    NUMBER = {2},
     PAGES = {363--372},
      ISSN = {0002-9947},
     CODEN = {TAMTAM},
   MRCLASS = {16K20 (12E15)},
  MRNUMBER = {1501642},
       DOI = {10.2307/1989546},
       URL = {http://dx.doi.org/10.2307/1989546},
}
%MR1501642,

@article {EGA2, 
    AUTHOR = {Grothendieck, A.},
     TITLE = {\'{E}l\'ements de g\'eom\'etrie alg\'ebrique. {II}. \'{E}tude
              globale \'el\'ementaire de quelques classes de morphismes},
   JOURNAL = {Inst. Hautes \'Etudes Sci. Publ. Math.},
  FJOURNAL = {Institut des Hautes \'Etudes Scientifiques. Publications
              Math\'ematiques},
    NUMBER = {8},
      YEAR = {1961},
     PAGES = {222},
      ISSN = {0073-8301},
   MRCLASS = {14.55},
  MRNUMBER = {0217084 (36 \#177b)},
}
%MR0217084,

@book {DI, 
    AUTHOR = {DeMeyer, Frank and Ingraham, Edward},
     TITLE = {Separable algebras over commutative rings},
    SERIES = {Lecture Notes in Mathematics, Vol. 181},
 PUBLISHER = {Springer-Verlag, Berlin-New York},
      YEAR = {1971},
     PAGES = {iv+157},
   MRCLASS = {13.70},
  MRNUMBER = {0280479 (43 \#6199)},
MRREVIEWER = {H. F. Kreimer},
}
%MR0280479,

@article {AG0, 
    AUTHOR = {Auslander, Maurice and Goldman, Oscar},
     TITLE = {Maximal orders},
   JOURNAL = {Trans. Amer. Math. Soc.},
  FJOURNAL = {Transactions of the American Mathematical Society},
    VOLUME = {97},
      YEAR = {1960},
     PAGES = {1--24},
      ISSN = {0002-9947},
   MRCLASS = {16.00 (18.00)},
  MRNUMBER = {0117252 (22 \#8034)},
MRREVIEWER = {C. W. Curtis},
}
%MR0117252,

@book {PR, 
    AUTHOR = {Platonov, Vladimir and Rapinchuk, Andrei},
     TITLE = {Algebraic groups and number theory},
    SERIES = {Pure and Applied Mathematics},
    VOLUME = {139},
      NOTE = {Translated from the 1991 Russian original by Rachel Rowen},
 PUBLISHER = {Academic Press, Inc., Boston, MA},
      YEAR = {1994},
     PAGES = {xii+614},
      ISBN = {0-12-558180-7},
   MRCLASS = {11E57 (11-02 20Gxx)},
  MRNUMBER = {1278263 (95b:11039)},
}
%MR1278263,

@article {LPS, 
    AUTHOR = {Lieblich, Max and Parimala, R. and Suresh, V.},
     TITLE = {Colliot-{T}helene's conjecture and finiteness of
              {$u$}-invariants},
   JOURNAL = {Math. Ann.},
  FJOURNAL = {Mathematische Annalen},
    VOLUME = {360},
      YEAR = {2014},
    NUMBER = {1-2},
     PAGES = {1--22},
      ISSN = {0025-5831},
   MRCLASS = {14F22},
  MRNUMBER = {3263156},
       DOI = {10.1007/s00208-014-1022-x},
       URL = {http://dx.doi.org/10.1007/s00208-014-1022-x},
}
%MR3263156,

@book {Bcomm10, 
    AUTHOR = {Bourbaki, N.},
     TITLE = {\'{E}l\'ements de math\'ematique. {A}lg\`ebre commutative.
              {C}hapitre 10},
      NOTE = {Reprint of the 1998 original},
 PUBLISHER = {Springer-Verlag, Berlin},
      YEAR = {2007},
     PAGES = {ii+187},
      ISBN = {978-3-540-34394-3; 3-540-34394-6},
   MRCLASS = {13-01},
  MRNUMBER = {2333539 (2008h:13001)},
MRREVIEWER = {Anne-Marie Simon},
}
%MR2333539,

@article {Az, 
    AUTHOR = {Azumaya, Gor{\^o}},
     TITLE = {On maximally central algebras},
   JOURNAL = {Nagoya Math. J.},
  FJOURNAL = {Nagoya Mathematical Journal},
    VOLUME = {2},
      YEAR = {1951},
     PAGES = {119--150},
      ISSN = {0027-7630},
   MRCLASS = {09.1X},
  MRNUMBER = {0040287 (12,669g)},
MRREVIEWER = {I. Kaplansky},
}
%MR0040287,

@book {Lam, 
    AUTHOR = {Lam, T. Y.},
     TITLE = {Introduction to quadratic forms over fields},
    SERIES = {Graduate Studies in Mathematics},
    VOLUME = {67},
 PUBLISHER = {American Mathematical Society, Providence, RI},
      YEAR = {2005},
     PAGES = {xxii+550},
      ISBN = {0-8218-1095-2},
   MRCLASS = {11Exx},
  MRNUMBER = {2104929 (2005h:11075)},
MRREVIEWER = {K. Szymiczek},
}
%MR2104929,

@book {Balg47, 
    AUTHOR = {Bourbaki, Nicolas},
     TITLE = {Algebra {II}. {C}hapters 4--7},
    SERIES = {Elements of Mathematics (Berlin)},
      NOTE = {Translated from the 1981 French edition by P. M. Cohn and J. Howie,
              Reprint of the 1990 English edition [Springer, Berlin;
              MR1080964 (91h:00003)]},
 PUBLISHER = {Springer-Verlag, Berlin},
      YEAR = {2003},
     PAGES = {viii+461},
      ISBN = {3-540-00706-7},
   MRCLASS = {00A05 (12-01 13-01)},
  MRNUMBER = {1994218},
       DOI = {10.1007/978-3-642-61698-3},
       URL = {http://dx.doi.org/10.1007/978-3-642-61698-3},
}
%MR1994218,

@article {P14, 
    AUTHOR = {Parimala, R.},
     TITLE = {A {H}asse principle for quadratic forms over function fields},
   JOURNAL = {Bull. Amer. Math. Soc. (N.S.)},
  FJOURNAL = {American Mathematical Society. Bulletin. New Series},
    VOLUME = {51},
      YEAR = {2014},
    NUMBER = {3},
     PAGES = {447--461},
      ISSN = {0273-0979},
   MRCLASS = {11E04 (11-03 11G35 11R58)},
  MRNUMBER = {3196794},
MRREVIEWER = {Pete L. Clark},
       DOI = {10.1090/S0273-0979-2014-01443-0},
       URL = {http://dx.doi.org/10.1090/S0273-0979-2014-01443-0},
}
%MR3196794,

@book {Pie, 
    AUTHOR = {Pierce, Richard S.},
     TITLE = {Associative algebras},
    SERIES = {Graduate Texts in Mathematics},
    VOLUME = {88},
      NOTE = {Studies in the History of Modern Science, 9},
 PUBLISHER = {Springer-Verlag, New York-Berlin},
      YEAR = {1982},
     PAGES = {xii+436},
      ISBN = {0-387-90693-2},
   MRCLASS = {16-01 (12-01)},
  MRNUMBER = {674652 (84c:16001)},
MRREVIEWER = {S. S. Page},
}
%MR674652,

@article {BM, 
    AUTHOR = {Becher, Karim Johannes and Mahmoudi, Mohammad G.},
     TITLE = {The orthogonal {$u$}-invariant of a quaternion algebra},
   JOURNAL = {Bull. Belg. Math. Soc. Simon Stevin},
  FJOURNAL = {Bulletin of the Belgian Mathematical Society. Simon Stevin},
    VOLUME = {17},
      YEAR = {2010},
    NUMBER = {1},
     PAGES = {181--192},
      ISSN = {1370-1444},
   MRCLASS = {11E39 (11E04 11E81 16K20)},
  MRNUMBER = {2656680 (2011d:11089)},
MRREVIEWER = {Detlev W. Hoffmann},
       URL = {http://projecteuclid.org/euclid.bbms/1267798507},
}
%MR2656680,

@article {Land, 
    AUTHOR = {Landherr, W.},
     TITLE = {Liesche ringe vom typus a \"uber einem algebraischen
              zahlk\"orper (die lineare gruppe) und hermitesche formen
              \"uber einem schiefk\"orper},
   JOURNAL = {Abh. Math. Sem. Univ. Hamburg},
  FJOURNAL = {Abhandlungen aus dem Mathematischen Seminar der Universit\"at
              Hamburg},
    VOLUME = {12}, 
      YEAR = {1937},
    NUMBER = {1},
     PAGES = {200--241},
      ISSN = {0025-5858},
     CODEN = {AMHAAJ},
   MRCLASS = {Contributed Item},
  MRNUMBER = {3069686},
       DOI = {10.1007/BF02948944},
       URL = {http://dx.doi.org/10.1007/BF02948944},
}
%MR3069686,

@article {Mah, 
    AUTHOR = {Mahmoudi, M. G.},
     TITLE = {Hermitian forms and the {$u$}-invariant},
   JOURNAL = {Manuscripta Math.},
  FJOURNAL = {Manuscripta Mathematica},
    VOLUME = {116},
      YEAR = {2005},
    NUMBER = {4},
     PAGES = {493--516},
      ISSN = {0025-2611},
     CODEN = {MSMHB2},
   MRCLASS = {11E39 (11E04 11E57 11E81 16K20 16W10)},
  MRNUMBER = {2140216 (2005k:11078)},
MRREVIEWER = {Detlev W. Hoffmann},
       DOI = {10.1007/s00229-005-0541-x},
       URL = {http://dx.doi.org/10.1007/s00229-005-0541-x},
}
%MR2140216,

@article {PS,
    AUTHOR = {Parihar, Sudeep S. and Suresh, V.},
     TITLE = {On the {$u$}-invariant of {H}ermitian forms},
   JOURNAL = {Proc. Indian Acad. Sci. Math. Sci.},
  FJOURNAL = {Indian Academy of Sciences. Proceedings. Mathematical
              Sciences},
    VOLUME = {123},
      YEAR = {2013},
    NUMBER = {3},
     PAGES = {303--313},
      ISSN = {0253-4142},
   MRCLASS = {11E81 (11E04 16K20)},
  MRNUMBER = {3102374},
MRREVIEWER = {Andrew Dolphin},
       DOI = {10.1007/s12044-013-0131-x},
       URL = {http://dx.doi.org/10.1007/s12044-013-0131-x},
}

@article {L, 
    AUTHOR = {Leep, David B.},
     TITLE = {The {$u$}-invariant of {$p$}-adic function fields},
   JOURNAL = {J. Reine Angew. Math.},
  FJOURNAL = {Journal f\"ur die Reine und Angewandte Mathematik. [Crelle's
              Journal]},
    VOLUME = {679},
      YEAR = {2013},
     PAGES = {65--73},
      ISSN = {0075-4102},
   MRCLASS = {11D72 (11D79 11D88)},
  MRNUMBER = {3065154},
MRREVIEWER = {Timothy D. Browning},
}
%MR3065154,

@article {PS10,
    AUTHOR = {Parimala, R. and Suresh, V.},
     TITLE = {The {$u$}-invariant of the function fields of {$p$}-adic
              curves},
   JOURNAL = {Ann. of Math. (2)},
  FJOURNAL = {Annals of Mathematics. Second Series},
    VOLUME = {172},
      YEAR = {2010},
    NUMBER = {2},
     PAGES = {1391--1405},
      ISSN = {0003-486X},
     CODEN = {ANMAAH},
   MRCLASS = {11E04 (11E08 11E81 12G05)},
  MRNUMBER = {2680494 (2011g:11074)},
MRREVIEWER = {Detlev W. Hoffmann},
       DOI = {10.4007/annals.2010.172.1397},
       URL = {http://dx.doi.org/10.4007/annals.2010.172.1397},
}
%MR2680494,

@unpublished{Wu,
	author	= {Wu, Zhengyao},
	title	= {Hasse principle for Hermitian 
	spaces over semi-global fields},
	year	= {2015},
	note	= {\href{http://arxiv.org/abs/1510.04640}{arXiv:1510.04640}},
}
%\bibitem[Wu14]{W}
%Z.~Wu, 
%\textit{A Hasse principle for hermitian forms}, 
%(2014) in preparation.

@article {T, 
    AUTHOR = {Tsukamoto, Takashi},
     TITLE = {On the local theory of quaternionic anti-hermitian forms},
   JOURNAL = {J. Math. Soc. Japan},
  FJOURNAL = {Journal of the Mathematical Society of Japan},
    VOLUME = {13},
      YEAR = {1961},
     PAGES = {387--400},
      ISSN = {0025-5645},
   MRCLASS = {10.15 (10.16)},
  MRNUMBER = {0136662 (25 \#127)},
MRREVIEWER = {J. Dieudonn{\'e}},
}

@article {CDTWY,
    AUTHOR = {Chacron, M. and Dherte, H. and Tignol, J.-P. and Wadsworth, A.
              R. and Yanchevski{\u\i}, V. I.},
     TITLE = {Discriminants of involutions on {H}enselian division algebras},
   JOURNAL = {Pacific J. Math.},
  FJOURNAL = {Pacific Journal of Mathematics},
    VOLUME = {167},
      YEAR = {1995},
    NUMBER = {1},
     PAGES = {49--79},
      ISSN = {0030-8730},
     CODEN = {PJMAAI},
   MRCLASS = {16K20 (12E15 12J25)},
  MRNUMBER = {1318164 (95k:16020)},
MRREVIEWER = {B. Fein},
       URL = {http://projecteuclid.org/euclid.pjm/1102620975},
}

@article {Leep-sys,
    AUTHOR = {Leep, David B.},
     TITLE = {Systems of quadratic forms},
   JOURNAL = {J. Reine Angew. Math.},
  FJOURNAL = {Journal f\"ur die Reine und Angewandte Mathematik},
    VOLUME = {350},
      YEAR = {1984},
     PAGES = {109--116},
      ISSN = {0075-4102},
     CODEN = {JRMAA8},
   MRCLASS = {11D72 (11E08)},
  MRNUMBER = {743536 (85j:11038)},
MRREVIEWER = {D. J. Lewis},
       DOI = {10.1515/crll.1984.350.109},
       URL = {http://dx.doi.org/10.1515/crll.1984.350.109},
}

@Article{Chevalley,
    Author = {Claude {Chevalley}},
    Title = {{D\'emonstration d'une hypoth\`ese de M. Artin.}},
    FJournal = {{Abhandlungen aus dem Mathematischen Seminar der Universit\"at Hamburg}},
    Journal = {{Abh. Math. Semin. Univ. Hamb.}},
    ISSN = {0025-5858; 1865-8784/e},
    Volume = {11},
    Pages = {73--75},
    Year = {1935},
    Publisher = {Springer, Berlin/Heidelberg},
    Language = {French},
    DOI = {10.1007/BF02940714},
    Zbl = {0011.14504}
}

@Article{Che58,
	Author = {Claude {Chevalley}},
	Title = {{Classification des groupes de Lie alg\'ebriques}},
	Journal = {{S\'em. Ec. Norm. Sup.}},
	Year = {1958},
	Publisher = {Paris},
}

@Article{Warning,
    Author = {Ewald {Warning}},
    Title = {{Bemerkung zur vorstehenden Arbeit von Herrn Chevalley.}},
    FJournal = {{Abhandlungen aus dem Mathematischen Seminar der Universit\"at Hamburg}},
    Journal = {{Abh. Math. Semin. Univ. Hamb.}},
    ISSN = {0025-5858; 1865-8784/e},
    Volume = {11},
    Pages = {76--83},
    Year = {1935},
    Publisher = {Springer, Berlin/Heidelberg},
    Language = {German},
    DOI = {10.1007/BF02940715},
    Zbl = {0011.14601}
}

@article {ABHN1,
	AUTHOR = {Brauer, R. and Noether, E. and Hasse, H.},
	TITLE = {Beweis eines {H}auptsatzes in der {T}heorie der {A}lgebren},
	JOURNAL = {J. Reine Angew. Math.},
	FJOURNAL = {Journal f\"ur die Reine und Angewandte Mathematik. [Crelle's
		Journal]},
	VOLUME = {167},
	YEAR = {1932},
	PAGES = {399--404},
	ISSN = {0075-4102},
	MRCLASS = {Contributed Item},
	MRNUMBER = {1581351},
	DOI = {10.1515/crll.1932.167.399},
	URL = {http://dx.doi.org/10.1515/crll.1932.167.399},
}

@article {ABHN2,
	AUTHOR = {Albert, A. Adrian and Hasse, Helmut},
	TITLE = {A determination of all normal division algebras over an
		algebraic number field},
	JOURNAL = {Trans. Amer. Math. Soc.},
	FJOURNAL = {Transactions of the American Mathematical Society},
	VOLUME = {34},
	YEAR = {1932},
	NUMBER = {3},
	PAGES = {722--726},
	ISSN = {0002-9947},
	CODEN = {TAMTAM},
	MRCLASS = {11R52 (12E15 16K20)},
	MRNUMBER = {1501659},
	DOI = {10.2307/1989375},
	URL = {http://dx.doi.org/10.2307/1989375},
}

@incollection {BrIII,
    AUTHOR = {Grothendieck, Alexander},
     TITLE = {Le groupe de {B}rauer. {III}. {E}xemples et compl\'ements},
 BOOKTITLE = {Dix expos\'es sur la cohomologie des sch\'emas},
    SERIES = {Adv. Stud. Pure Math.},
    VOLUME = {3},
     PAGES = {88--188},
 PUBLISHER = {North-Holland, Amsterdam},
      YEAR = {1968},
   MRCLASS = {14F22},
  MRNUMBER = {244271},
MRREVIEWER = {James Milne},
}
		
@incollection {BrII,
    AUTHOR = {Grothendieck, Alexander},
     TITLE = {Le groupe de {B}rauer. {II}. {T}h\'eorie cohomologique},
 BOOKTITLE = {Dix expos\'es sur la cohomologie des sch\'emas},
    SERIES = {Adv. Stud. Pure Math.},
    VOLUME = {3},
     PAGES = {67--87},
 PUBLISHER = {North-Holland, Amsterdam},
      YEAR = {1968},
   MRCLASS = {14F22},
  MRNUMBER = {244270},
MRREVIEWER = {James Milne},
}
		
@incollection {BrI,
    AUTHOR = {Grothendieck, Alexander},
     TITLE = {Le groupe de {B}rauer. {I}. {A}lg\`ebres d'{A}zumaya et
              interpr\'etations diverses},
 BOOKTITLE = {Dix expos\'es sur la cohomologie des sch\'emas},
    SERIES = {Adv. Stud. Pure Math.},
    VOLUME = {3},
     PAGES = {46--66},
 PUBLISHER = {North-Holland, Amsterdam},
      YEAR = {1968},
   MRCLASS = {14F22},
  MRNUMBER = {244269},
MRREVIEWER = {James Milne},
}

@incollection {Abh,
  AUTHOR = {Abhyankar, Shreeram Shankar},
   TITLE = {Resolution of singularities of algebraic surfaces},
 BOOKTITLE = {Algebraic {G}eometry ({I}nternat. {C}olloq., {T}ata {I}nst.
       {F}und. {R}es., {B}ombay, 1968)},
   PAGES = {1--11},
 PUBLISHER = {Oxford Univ. Press, London},
   YEAR = {1969},
  MRCLASS = {14.18},
 MRNUMBER = {0257080 (41 \#1734)},
MRREVIEWER = {Joseph Lipman},
}

@book {OM,
    AUTHOR = {O'Meara, O. T.},
     TITLE = {Introduction to quadratic forms},
    SERIES = {Die Grundlehren der mathematischen Wissenschaften, Bd. 117},
 PUBLISHER = {Academic Press, Inc., Publishers, New York; Springer-Verlag,
              Berlin-G\"ottingen-Heidelberg},
      YEAR = {1963},
     PAGES = {xi+342},
   MRCLASS = {10.16 (12.70)},
  MRNUMBER = {0152507 (27 \#2485)},
MRREVIEWER = {K. Iwasawa},
}

@book {TW15,
    AUTHOR = {Tignol, Jean-Pierre and Wadsworth, Adrian R.},
     TITLE = {Value functions on simple algebras, and associated graded
              rings},
    SERIES = {Springer Monographs in Mathematics},
 PUBLISHER = {Springer, Cham},
      YEAR = {2015},
     PAGES = {xvi+643},
      ISBN = {978-3-319-16359-8; 978-3-319-16360-4},
   MRCLASS = {16K20 (12Fxx 16K50 16W50 16W60 19Bxx)},
  MRNUMBER = {3328410},
}

@article {Heath-Brown,
    AUTHOR = {Heath-Brown, D. R.},
     TITLE = {Zeros of systems of {$p$}-adic quadratic forms},
   JOURNAL = {Compos. Math.},
  FJOURNAL = {Compositio Mathematica},
    VOLUME = {146},
      YEAR = {2010},
    NUMBER = {2},
     PAGES = {271--287},
      ISSN = {0010-437X},
   MRCLASS = {11E08 (11D72 11D79 11D88)},
  MRNUMBER = {2601629 (2011e:11066)},
MRREVIEWER = {Alexandra Shlapentokh},
       DOI = {10.1112/S0010437X09004497},
       URL = {http://dx.doi.org/10.1112/S0010437X09004497},
}

@article {Lang,
	AUTHOR = {Lang, Serge},
	TITLE = {On quasi algebraic closure},
	JOURNAL = {Ann. of Math. (2)},
	FJOURNAL = {Annals of Mathematics. Second Series},
	VOLUME = {55},
	YEAR = {1952},
	PAGES = {373--390},
	ISSN = {0003-486X},
	MRCLASS = {10.0X},
	MRNUMBER = {0046388 (13,726d)},
	MRREVIEWER = {T. Nakayama},
}

@article {MS,
    AUTHOR = {Merkurjev, A. S. and Suslin, A. A.},
     TITLE = {{$K$}-cohomology of {S}everi-{B}rauer varieties and the norm
              residue homomorphism},
   JOURNAL = {Izv. Akad. Nauk SSSR Ser. Mat.},
  FJOURNAL = {Izvestiya Akademii Nauk SSSR. Seriya Matematicheskaya},
    VOLUME = {46},
      YEAR = {1982},
    NUMBER = {5},
     PAGES = {1011--1046, 1135--1136},
      ISSN = {0373-2436},
   MRCLASS = {12A62 (12G05 14C35 14F12 14F15 14G25 18F25)},
  MRNUMBER = {675529 (84i:12007)},
MRREVIEWER = {Maksymilian Boraty{\'n}ski},
}

@article {Mer81,
    AUTHOR = {Merkurjev, A. S.},
     TITLE = {On the norm residue symbol of degree {$2$}},
   JOURNAL = {Dokl. Akad. Nauk SSSR},
  FJOURNAL = {Doklady Akademii Nauk SSSR},
    VOLUME = {261},
      YEAR = {1981},
    NUMBER = {3},
     PAGES = {542--547},
      ISSN = {0002-3264},
   MRCLASS = {12A60 (13A20)},
  MRNUMBER = {638926 (83h:12015)},
MRREVIEWER = {Thomas Zink},
}

@article {W1,
    AUTHOR = {Wadsworth, Adrian R.},
     TITLE = {Extending valuations to finite-dimensional division algebras},
   JOURNAL = {Proc. Amer. Math. Soc.},
  FJOURNAL = {Proceedings of the American Mathematical Society},
    VOLUME = {98},
      YEAR = {1986},
    NUMBER = {1},
     PAGES = {20--22},
      ISSN = {0002-9939},
     CODEN = {PAMYAR},
   MRCLASS = {16A39 (12E15)},
  MRNUMBER = {848866 (87i:16025)},
MRREVIEWER = {P. M. Cohn},
       DOI = {10.2307/2045758},
       URL = {http://dx.doi.org/10.2307/2045758},
}

@book {Die,
	AUTHOR = {Dieudonn{\'e}, Jean},
	TITLE = {La g\'eom\'etrie des groupes classiques},
	SERIES = {Seconde \'edition, revue et corrig\'ee},
	PUBLISHER = {Springer-Verlag, Berlin-G\"ottingen-Heidelberg},
	YEAR = {1963},
	PAGES = {viii+125},
	MRCLASS = {20.70},
	MRNUMBER = {0158011},
}

@article {Jac,
	AUTHOR = {Jacobson, N.},
	TITLE = {A {N}ote on {T}opological {F}ields},
	JOURNAL = {Amer. J. Math.},
	FJOURNAL = {American Journal of Mathematics},
	VOLUME = {59},
	YEAR = {1937},
	NUMBER = {4},
	PAGES = {889--894},
	ISSN = {0002-9327},
	CODEN = {AJMAAN},
	MRCLASS = {Contributed Item},
	MRNUMBER = {1507289},
	DOI = {10.2307/2371355},
	URL = {http://dx.doi.org/10.2307/2371355},
}

@article {Weil,
	AUTHOR = {Weil, Andr{\'e}},
	TITLE = {Algebras with involutions and the classical groups},
	JOURNAL = {J. Indian Math. Soc. (N.S.)},
	VOLUME = {24},
	YEAR = {1960},
	PAGES = {589--623 (1961)},
	MRCLASS = {22.50 (16.46)},
	MRNUMBER = {0136682},
	MRREVIEWER = {J. Dieudonn{\'e}},
}
\end{filecontents}
\bibliographystyle{amsalpha}
\newpage
\bibliography{wu}
\end{document}