\documentclass[UTF8]{ctexart}

\usepackage{amsmath,amssymb,amsthm}
\usepackage{mathrsfs}
\usepackage{color}
\usepackage{tikz-cd}
\usepackage{url}
\usepackage{filecontents}

\DeclareMathOperator{\Soc}{Soc}
\DeclareMathOperator{\Rad}{Rad}
\DeclareMathOperator{\Aut}{Aut}
\DeclareMathOperator{\Gal}{Gal}

\usepackage[hyperref=true,backend=biber,sorting=none,backref=true]{biblatex}
\addbibresource{abstract_algebra_notes.bib}
\usepackage{hyperref}

\begin{document}
\begin{enumerate}
\item \textbf{定义 4.3.1} 设 $p$ 是素数. 若群 $G$ 的阶是 $p$ 的方幂, 即 $\left\lvert G\right\lvert = [G : e] = p^k(k \in \mathbb{N})$, $e$ 为 $G$ 的幺元, 则称 $G$ 是一个 $p$ \textbf{群}.
\item \textbf{定理 4.3.1} 设 $p$ 群 $G$ 作用在集合 $X$ 上, $\left\lvert X\right\rvert = n$, 则有下列结论:
\begin{enumerate}
  \item $t \equiv n\;(\bmod\;p)$, 其中, $t = \left\lvert\{x \mid x\in X, gx=x,\forall g\in G\}\right\rvert$
  \item 当 $(n,p) = 1$ 时, $t\geqslant1$, 即 $\exists x\in X$, 使 $g(x) = x(\forall g\in G)$;
  \item $G$ 的中心 $C(G)\neq\{e\}$.
\end{enumerate}

\item \textbf{引理 4.3.1} 设 $p$ 是素数, $n = p^lm, (m,p) = 1$. 若 $k\in\mathbb{N},k\leqslant l$, 则
\[
p^{l-k}\|C_n^{p^k},
\]
其中, $\|$ 表示恰能整除, $C_n^{p^k}$ 是组合数.

\item \textbf{定理 4.3.2}(Sylow 第一定理) 设 $G$ 是一个阶为 $p^lm$ 的群, 其中, $p$ 为素数, $l \geqslant 1, (p,m) = 1$, 则对任何 $1 \leqslant k \leqslant l$, $G$ 中一定有 $p^k$ 阶子群.

\item \textbf{定义 4.3.2} 设群 $G$ 的阶为 $p^lm$, $p$ 为素数且 $(p,m) = 1$, 则 $G$ 的 $p^l$ 阶子群称为 $G$ 的\textbf{Sylow} $p$ \textbf{子群}.

\item \textbf{定理 4.3.3} (Sylow 第二定理) 设群 $G$ 的阶为 $p^lm$, $p$ 为素数, $(p,m) = 1$. 又 $P$ 是 $G$ 的一个 Sylow $p$ 子群, $H$ 是 $G$ 的一个 $p^k$ 阶子群, 则 $\exists g\in G$, 使 $H\subseteq gPg^{-1}$.

\item \textbf{定理 4.3.4} (Sylow 第三定理) 设群 $G$ 的阶为 $p^lm$, $p$ 为素数, $(p,m) = 1$. 又设 $G$ 中 Sylow $p$ 子群的个数为 $k$, 则有
    \begin{enumerate}
      \item 当且仅当 $k = 1$ 时, $G$ 的 Sylow $p$ 子群 $P\vartriangleleft G$;
      \item $k|m, k \equiv 1\;(\bmod\;p)$.
    \end{enumerate}

\item Sylow 定理在群论中有许多应用, 其一就是判断某些有限群不是\textbf{单群}(一个群如果没有非平凡的正规子群就称为\textbf{单群}).

\item \textbf{定义 4.4.1} 若有限群 $G$ 无非平凡的正规子群, 则称 $G$ 为\textbf{有限单群}.

\item \textbf{定理 4.4.1} 设 $G$ 为 Abel 群且 $G\neq\{e\}$, $e$ 为 $G$ 的幺元, 则 $G$ 为单群的充分必要条件是 $G$ 的阶为素数. 这时 $G$ 必为循环群.

\item \textbf{定理 4.5.3} 设 $A,B$ 是 $G$ 的子群.
\begin{enumerate}
  \item $G=AB,A\cap B=\{1\}$ 当且仅当 $\forall g\in G, \exists a\in A,b\in B$, 使得 $g=ab$ 且这种表示唯一.
  \item 若 $G=AB, A\cap B=\{1\}$, 则 $A,B$ 都是 $G$ 的正规子群的充分必要条件是 $ab=ba(\forall a\in A,b\in B)$, 此时 $G=A\otimes B$.
\end{enumerate}

\item \textbf{定义 4.6.1} 设 $g_1,g_2$ 是群 $G$ 中的两个元素, 称
\[
[g_1,g_2] = g_1^{-1}g_2^{-1}g_1g_2
\]
为 $g_1$ 与 $g_2$ 的\textbf{换位子}.

\item \textbf{定义 4.6.2} 若 $H,K$ 是群 $G$ 的两个子群, 称
\[
[H,K] = \langle\{[h,k]\mid h\in H,k\in K\}\rangle
\]
为 $H$ 与 $K$ 的\textbf{换位子群}.

以上引用自\parencite{孟道骥2010抽象代数}.

\noindent\makebox[\linewidth]{\rule{\paperwidth}{0.4pt}}

\item 最小多项式(Minimal polynomial (field theory))

域论中的最小多项式是相对于一个域扩张(field extension) $E/F$ 和 $E$ 中的一个元素来定义的. 一个元素的最小多项式如果存在的话, 是 $F[x]$ 中的一个多项式. 给定 $E$ 中的一个元素 $\alpha$, 令 $J_\alpha$ 为 $F[x]$ 中所有满足 $f(\alpha) = 0$ 的多项式 $f(x)$ 组成的集合. $\alpha$ 就被称为 $J_\alpha$ 中每一个多项式的根(root)或零点(zero). $J_\alpha$ 是 $F[x]$ 的一个理想. 如果 $J_\alpha$ 含有任何一个非零多项式, 那么 $\alpha$ 就被称为 $F$ 上的一个代数元(algebraic element), 并且在 $J_\alpha$ 中存在一个次数最小的首一多项式(monic polynomial). 这就是 $\alpha$ 相对于 $E/F$ 的最小多项式. 它是唯一的, 并且相对于 $F$ 不可约(irreducible). 如果 $J_\alpha$ 只含有零多项式, 那么 $\alpha$ 被称为 $F$ 上的一个超越元(transcendental element), 并且相对于 $E/F$ 没有最小多项式.

最小多项式在构造和分析域扩张的时候是有用的. 当 $\alpha$ 是代数的, 并且其最小多项式 为 $a(x)$ 时, 同时包含 $F$ 和 $\alpha$ 的最小的域同构于商环 $F[x]/\langle a(x)\rangle$, 其中 $\langle a(x)\rangle$ 是由 $a(x)$ 生成的 $F[x]$ 的理想.

\item Artinian 环上的模

Let $M$ be a left module over a left Artinian ring. Then the following are equivalent (Hopkins' theorem): (i) $M$ is finitely generated, (ii) $M$ has finite length (i.e., has composition series), (iii) $M$ is Noetherian, (iv) $M$ is Artinian.

群环 $kG$ 都为 artinian. 进一步如果域的特征为零或与群的阶互素(Maschke's theorem), $kG$ 是半单的, 而半单环既是 artinian 又是 noetherian.

以下考虑的环都为 Artinian.

\item 区分半单(semisimple)和完全可约(complete reducible)

模 $M$ 是半单的, 如果它是单模的直和.

模 $M$ 是完全可约的, 如果对所有的子模 $U\subset M$, 存在一个补子模 $V\subset M$, 满足 $M = U\oplus V$(也就是 $M=U+V$ 并且 $U\cap V = 0$.

\item 环的 Jacobson radical

\[
\Rad(A) = \{ x\mid xS = 0 \text{ for all simple modules } S \}.
\]

$\Rad(A)$ 是最大的幂零理想.

$\Rad(A)$ 是所有极大左(右)理想的交.

$\Rad(A)$ 是使得 $A/\Rad(A)$ 半单的最小的左理想.

$M$ 为任意的 $A$-模, 我们定义 $R(M) := \Rad(A)M$.

$M$ 为 $A$-模, 则 $\Rad(M)$ 是使得 $M/\Rad(M)$ 半单的最小子模, 并且它是 $M$ 的所有极大左子模的交.

若 $A$ 为 Artinian 环, 则 $A$ 上的有限生成的不可分解投射模和 $A$上的单模一一对应.

有限群的群代数是Artinian环.
\item Frobenius algebra

$P$ 是不可分解投射模当且仅当 $P/\Rad(P)\cong \Soc(P)$.

\item Finitely generated abelian group\parencite{finitely_generated_abelian_group}

In abstract algebra, an abelian group $(G, +)$ is called \textbf{finitely generated} if there exist finitely many elements $x_1,\ldots, x_s$ in $G$ such that every $x$ in $G$ can be written in the form
\[
    x = n_1x_1 + n_2x_2 + ... + n_sx_s
\]
with integers $n_1,\ldots, n_s$. In this case, we say that the set $\{x_1,\ldots, x_s\}$ is a generating set of $G$ or that $x_1,\ldots, x_s$ generate $G$.

Clearly, every finite abelian group is finitely generated. The finitely generated abelian groups are of a rather simple structure and can be completely classified, as will be explained below.

Classification

The \textbf{fundamental theorem of finitely generated abelian groups} (which is a special case of the structure theorem for finitely generated modules over a principal ideal domain) can be stated two ways (analogously with principal ideal domains):

\textbf{Primary decompostion}

The primary decomposition formulation states that every finitely generated abelian group $G$ is isomorphic to a direct sum of primary cyclic groups and infinite cyclic groups. A primary cyclic group is one whose order is a power of a prime. That is, every finitely generated abelian group is isomorphic to a group of the form
\[
\mathbb {Z} ^{n}\oplus \mathbb {Z} _{q_{1}}\oplus \cdots \oplus \mathbb {Z} _{q_{t}},
\]
where the rank $n \geq 0$, and the numbers $q_1,\ldots, q_t$ are powers of (not necessarily distinct) prime numbers. In particular, $G$ is finite if and only if $n = 0$. The values of $n, q_1,\ldots, q_t$ are (up to rearranging the indices) uniquely determined by $G$.

\textbf{Invariant factor decomposition}

We can also write any finitely generated abelian group $G$ as a direct sum of the form
\[
\mathbb {Z} ^{n}\oplus \mathbb {Z} _{k_{1}}\oplus \cdots \oplus \mathbb {Z} _{k_{u}},
\]
where $k_1$ divides $k_2$, which divides $k_3$ and so on up to $k_u$. Again, the rank $n$ and the invariant factors $k_1,\ldots, k_u$ are uniquely determined by $G$ (here with a unique order).

\textbf{Equivalence}

These statements are equivalent because of the Chinese remainder theorem, which here states that $\mathbb {Z} _{m}\simeq \mathbb {Z} _{j}\oplus \mathbb {Z} _{k}$ if and only if $j$ and $k$ are coprime and $m = jk$.

\item 域扩张\parencite{wiki域扩张}

\textbf{域扩张}(field extensions)是数学分支抽象代数之域论中的主要研究对象, 基本想法是从一个基域开始以某种方式构造包含它的"更大"的域. 域扩张可以推广为环扩张.

定义

设 $K$ 和 $L$ 是两个域. 如果存在从 $K$ 到 $L$ 的域同态 $\iota$, 则称$(L,\iota)$是$K$的一个\textbf{域扩张}, 记作$L/K$或$K\subseteq L$、$K\subset L$. $K$ 称为域扩张的\textbf{基域}, $L$称为$K$的\textbf{扩域}. 如果某个域$F$既是$K$的扩域, 又是$L$的子域, 则称域扩张 $F/K$ 是域扩张$L/K$的\textbf{子扩张}, 称$F$(域扩张$L/K$的)\textbf{中间域}.

域扩张的记法$L/K$只是形式上的标记, 不表示存在任何商环或商群等代数结构. 有些文献中也会将域扩张记为$L:K$.

另外, \textcolor{red}{因为$\iota$是域同态, 所以$\iota$是单射($\iota$把$K$的乘法单位元映为$L$的乘法单位元). 由于$K$是域, 所以$\iota(K)$是一个$L$的同构于$K$的子域. 很多时候也直接省略$\iota$, 直接将$K$视为$L$的一个子域.} 为了记叙方便, 下文中将依情况使用这种省略方式.

设有域扩张$L/K$, 给定一个由$L$中不属于$\iota(K)$的元素组成的集合$S$, 考虑$L$中所有同时包含$\iota(K)$和$S$的子域, 其中有一个"最小的", 称为"在$K$中添加(集合)$S$生成的扩域", 记作$K(S)$. 它是所有同时包含$\iota(K)$和$S$的域的子域. 如果集合$S$只有一个元素$a$, 则称域扩张$K(S)/K$为\textbf{单扩张}, 对应的扩域一般简记作$K(a)$. $a$ 称为这个域扩张的\textbf{本原元}.

每个域扩张中, 扩域可以看作是以基域为系数域的向量空间. 设有域扩张$L/K$, 将$L$中元素看作向量, $K$ 中元素看作系数, 可以定义$L$中的域加法运算作为向量的加法运算, 同时可以定义$K$中元素作为系数与$L$中元素的数乘运算. 可以验证, 在这样的定义下, $L$是一个$K$-向量空间. 它的维数称为域扩张的\textbf{次数}或\textbf{度数}, 一般记作$[L:K]$. 次数为$1$的扩张, 扩域和基域同构, 称为\textbf{平凡扩张}. 次数有限的域扩张称为\textbf{有限扩张}, 否则称为\textbf{无限扩张}.

例子

复数域 $\mathbb{C}$是实数域$\mathbb{R}$的扩域,而$\mathbb{R}$则是有理数域$\mathbb{Q}$的扩域。这样,显然$\mathbb{C} \big/ \mathbb{R}$也是一个域扩张。实数到复数的域扩张次数: $[\mathbb{C} : \mathbb{R}] = 2$。因为 $\mathbb{C}$可以看作是以$\{1, i\}$为基的实向量空间。故扩张$\mathbb{C} \big/ \mathbb{R}$是有限扩张。$\mathbb{C} = \mathbb{R}(i)$,所以这个扩张是单扩张。

集合 $\mathbb{Q}(\sqrt{2}) = \{ a+ b\sqrt{2} ; \; a, b, \in \mathbb{Q} \}$是在$\mathbb{Q}$ 中添加 $\sqrt{2}$生成的扩域,显然也是一个单扩张。它的次数是$2$,因为 $\{ 1, \sqrt{2}\}$可作为一个基。 $\mathbb{Q}$ 的有限扩张也称为代数数域,在代数数论有重要地位。

有理数的另一个扩张域是关于一个素数$p$的$p$进数域 $\mathbb{Q}_p$。它与 $\mathbb{R}$类似,是有理数域完备化得到的数域。但由于使用的拓扑不同,所以与$\mathbb{R}$有着截然不同的性质。

对任何的素数$p$和正整数$n$,都存在一个元素个数为$p^n$的有限域,记作$GF(p^n)$。它是有限域$GF(p)$(即$\mathbb{Z} \big/ p\mathbb{Z}$)的扩域。

给定域$K$和以$K$中元素为系数的$K$-不可约多项式$P$,$P$为$K$上的多项式环$K[X]$的元素。$P$生成的理想是极大理想,因此$K[X]/P$是域,而且是$K$的扩域。其中不定元$X$是多项式$P$的根\textcolor{red}{($X$所在的等价类代入$P$后,就是$P$所在的等价类,在$K[X]/P$中就是零元)}。

给定域$K$,考虑所有以$K$中元素为系数的有理函数,即可以表示为两个以$K$中元素为系数的多项式$P$、$Q$之比:$\frac{P}{Q}$的函数。它们构成一个域,记作$K(X)$,是多项式环$K[X]$的分式域。它是域$K$的扩域,次数为无限大。

基本性质

设有域扩张$L/K$, 则扩域$L$与$K$有相同的加法和乘法单位元. 加法群$(K,+)$是$(L,+)$的一个子群, 乘法群$(K^{×},\cdot)$是$(L^{\times},\cdot)$的一个子群. 因此, $L$与$K$有相同的特征.

设有域扩张$L/K$及某个中间域$F$, 则域扩张$F/K$和$L/F$的次数乘积等于$L/K$的次数:
\[
[L:K] = [L:F]\cdot[F:K].
\]

代数元与超越元

给定域扩张$L/K$, 如果$L$中一个元素$a$是某个以$K$中元素为系数的(非零)多项式(以下简称为$K$-多项式)的根, 则称$a$是$K$上的一个\textbf{代数元}, 否则称其为\textbf{超越元}. 如果$L$中每个元素都是$K$上的代数元, 就称域扩张$L/K$为\textbf{代数扩张}, 否则称其为\textbf{超越扩张}. 例如 $\sqrt{2}$和 $i$都是$\mathbb{Q}$上的代数元, 而$e$与$\pi$都是$\mathbb{Q}$上的超越元. $\mathbb{Q}$上的代数元和超越元分别叫做代数数与超越数.

每个有限扩张都是代数扩张, 反之则不然. 超越扩张必然是无限扩张. 给定域扩张 $L/K$, 如果$L$中元素要么属于$K$, 要么是$K$上的超越元, 则称$L$是$K$的纯超越扩张. 一个单扩张如果由添加代数元生成则是有限扩张, 如果由添加超越元生成则是纯超越扩张.

\textbf{极小多项式}

给定域扩张 $L/K$, 如果$L$中一个元素 $a$ 是 $K$ 上的代数元, 那么在所有使得$f(a)=0$的首一$K$-多项式$f$中, 存在一个次数最小的, 称为 $a$ 在 $K$ 上的\textbf{极小多项式}, 记为$\pi_{a}$. 设$\pi_a$为$n$次多项式, 则中间域 $K(a)$等于所有以$a$为不定元的$K$-多项式的集合. 更具体地说, 等于所有以$a$为不定元的、次数严格小于$n$的$K$-多项式的集合: $K(a) = K[a] = K_{n-1}[a]$. 这说明$K(a)$中任何元素$b$都可以写成$b = \lambda_1 + \lambda_2a +\cdots+ \lambda_na^{n-1}$的形式. 其中$(\lambda_1,\lambda_2,\ldots,\lambda_n)$是$n$个$K$中元素. 由于$\pi_a$是极小多项式, 所以可推出:$\{1,a,\ldots,a^{n-1}\}$是中间域$K(a)$作为$K$-向量空间的基. 扩张$K(a)/K$的次数是$[K(a):K] = n$.

\textbf{分裂域与代数闭包}

分裂域是将某个多项式的根全部添加到其系数域中生成的域扩张, 将多项式转化为域扩张进行研究. 给定域扩张$L/K$, 称一个$K$-多项式$f$在$L$中\textbf{分裂}, 如果$f$可以写成:
\[
f = \kappa(X-\alpha_1)(X-\alpha_2)\cdots(X-\alpha_k),\kappa\in K,\alpha_1,\alpha_2,\cdots,\alpha_k \in L
\]
的形式, 即$f$的每个根都是$L$中的元素. 如果$f$在$L$中分裂, 但不在$L$的任何一个包含$K$的真子域中分裂(也就是说$L$是令$f$在其中分裂的"最小"的域扩张), 就称$L$是$f$在$K$上的\textbf{分裂域}.

给定域$K$, 如果所有$K$-多项式在$K$分裂, 则称$K$为\textbf{代数闭域}. 给定\textcolor{red}{代数扩张}$L/K$, 如果$L$是代数闭域, 则称其为$K$的\textbf{代数闭包}, 一般记作 $K^{\mathrm{alg}}$. 给定$K$, 则它所有的代数闭包都是$K$-同构的.

域扩张的自同构群

除了将域扩张看作基域上的向量空间外, 另一个研究域扩张的角度是考察域扩张的自同构群. 给定域扩张$L/K$, $L$ 上的一个自同构$\sigma$被称为$K$-自同构, 当且仅当$\sigma$限制在$K$上的部分是\textbf{平凡}的(即为恒等映射):
\[
\forall x\in K, \sigma(x) = x.
\]
所有的$K$-自同构组成一个群, 称为域扩张的自同构群, 记作$\Aut(L/K)$. 这些自同构描绘了$K$"以外"的元素可以怎样相互变换而保持域$L$的域结构不变.

正规、可分与伽罗瓦扩张

伽罗瓦扩张是伽罗瓦理论中的基础概念。有限的伽罗瓦扩张满足伽罗瓦理论基本定理,在此扩张的伽罗瓦群的子群与其中间域之间建立了一一对应的关系,从而给出了中间域的清晰描述。

一般定义伽罗瓦扩张是正规且可分的域扩张。 一个域扩张$L/K$称为正规扩张, 如果对任何一个以$K$中元素为系数的不可约多项式$P$, 只要它有一个根在$L$中, 则它的所有根都在$L$中, 也就是说可以分解为$L$上一次因式的乘积. 正规扩张也就做准伽罗瓦扩张, 它与伽罗瓦扩张的差别是伽罗瓦扩张还是可分扩张. 一个\textcolor{red}{代数扩张}$L/K$称为可分扩张, 如果$L$中每个元素在$K$上的极小多项式是可分的, 即(在$K$的一个代数闭包中)没有重根. 从以上正规扩张和可分扩张的定义中可以推出: 一个域扩张$L/K$是伽罗瓦扩张, 当且仅当它是某个以$K$中元素为系数的可分多项式的分裂域.

伽罗瓦扩张的自同构群称为其伽罗瓦群, 记作$\Gal(L/K)$. 它的阶数(群中元素个数)等于伽罗瓦扩张的次数: $[L:K] = \left|\Gal(L/K)\right|$. 伽罗瓦理论基本定理说明, 当伽罗瓦扩张是有限扩张的时候, 给定$\Gal(L/K)$ 的任一个子群 $H$, 唯一存在一个中间域 $K\subset L^{H} \subset L$ 与之对应, 这个域 $L^{H}$ 恰好是$L$中对所有的$H$中的自同构固定的元素的集合:
\[
L^{H} = \{x;\forall\sigma\in H, \sigma(x) = x\}
\]
这种对应关系被称作\textbf{伽罗瓦对应}. 给定$\Gal(L/K)$的子群$H$, $L^{H}$被称为$H$的\textbf{对应域}. 伽罗瓦对应建立了特定条件下域扩张与群论之间转化的纽带, 通过研究特定群的结构, 可以给出域扩张的仔细刻画.
\end{enumerate}

\begin{filecontents*}{abstract_algebra_notes.bib}
@online{finitely_generated_abelian_group,
author = "Wikipedia",
title = "Finitely generated abelian group",
url = "https://en.wikipedia.org/wiki/Finitely_generated_abelian_group"
}

@online{wiki域扩张,
author = "Wikipedia",
title = "域扩张",
url = "https://zh.wikipedia.org/wiki/%E5%9F%9F%E6%89%A9%E5%BC%A0"
}

@book{姚慕生1998抽象代数学,
  title={抽象代数学},
  author={姚慕生},
  year={1998},
  publisher={复旦大学出版社}
}

@article{孟道骥2010抽象代数,
  title={抽象代数 I. 代数学基础},
  author={孟道骥 陈良云 史毅茜  白瑞蒲},
  publisher={科学出版杜},
  year={2010}
}

\end{filecontents*}

\newpage
\printbibliography
\end{document}