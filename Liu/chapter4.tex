\documentclass{ctexart}
\usepackage{amsmath}
\usepackage{mathtools}
\usepackage{tensor}
\usepackage{mhchem}
\usepackage{xfrac} %提供\sfrac
\usepackage{mathdots}
\usepackage{amssymb}
\usepackage{bbm,mathrsfs}
\usepackage{eucal}
\usepackage{upgreek}
\usepackage{amsxtra}
\usepackage{bm}
\usepackage{extarrows}
\usepackage{cases}

%\usepackage[mathstyleoff]{breqn}

%\usepackage{unicode-math}
%\setmainfont{Cambria}
%\setmathfont{Cambria Math}

\setlength{\parindent}{0pt}
\newcommand\stiring[2]{\genfrac{[}{]}{0pt}{}{#1}{#2}}
\newcommand\dstiring[2]{\genfrac{[}{]}{0pt}{0}{#1}{#2}}
\newcommand\tstiring[2]{\genfrac{[}{]}{0pt}{1}{#1}{#2}}
\DeclareMathOperator{\card}{card}         % 集合基数
\DeclareMathOperator*{\esssup}{ess\,sup}  % 本性上确界
\DeclareMathOperator\dif{d\!}

\newcommand\defeq{\stackrel{\text{d}}{=}}
\newcommand\varnotin{\mathrel{\overline{\in}}}
\newcommand*\abs[1]{\lvert#1\rvert}

\begin{document}
交换律是 $a+b=b+a$, 如 $1+2=2+1=3$.

不能用 a+b=b+a, 1+2=2+1=3. \begin{math}a+b\end{math}.

\begin{equation}
a+b=b+a \label{eq:commutative}
\end{equation}

$3^{3^{3^{\cdot^{\cdot^{\cdot^3}}}}}$

$\prescript{n}{m}{H}_i^j < L$

\[
\sideset{_a^b}{_c^d} \sum_{i=0}^n A_i
= \sideset{}{'} \prod_k f_i
\]

$\overset{\ast}{\underset{\dag}{X}}$

$A_m^{\phantom{m}n}$

$M\indices{^a_b^{cd}_e}$ \qquad
$\tensor[^a_b^c_d]{M}{^a_b^c_d}$

醋中主要是 \ce{H2O}, 含有 \ce{CH3COO-}.

\ce{^{227}_{90}Th} 元素具有强放射性.

\begin{equation}
\ce{2H2 + O2 ->[\text{燃烧}] 2H2O}
\end{equation}

$\overline{a+b} =
\overline a + \overline b$ \\
$\underline a = (a_0, a_1, a_2, \dots)$

$ \overline{\underline{\underline a}
+ \overline{b}^2} - c^{\underline n} $

$\overleftarrow{a+b}$\\
$\overrightarrow{a+b}$\\
$\overleftrightarrow{a+b}$\\
$\underleftarrow{a-b}$\\
$\underrightarrow{a-b}$\\
$\underleftrightarrow{a-b}$

$\vec x = \overrightarrow{AB}$

$\overbrace{a+b+c} = \underbrace{1+2+3}$

\[
( \overbrace{a_0,a_1,\dots,a_n}^{\text{共 $n+1$ 项}} ) =
( \underbrace{0,0,\dots,0}_{n} , 1 )
\]
\[ \underbracket{\overbracket{1+2}+3}_3 \]

% \llap 和 \rlap : 这两个命令分别把参数中的内容向当前位置的左侧和右侧重叠
语言文字 word\\
语言文字\llap{word}\\
\rlap{word}语言文字

\[
\rlap{$\overbrace{\phantom{a \to b}}$} a \to \underbrace{b \to c}
\]

\[ a + \overbrace{b+c+d}^{m} + e + f \]

\[ a + \rlap{$\overbrace{\phantom{b + c + d}}^{m}$} b + \underbrace{c + d + e}_{n} + f \]

\[ \frac 12 + \frac 1a = \frac{2+a}{2a} \]

通分计算 $\frac 12 + \frac 1a$ 得 $\frac{2+a}{2a}$

\[ \frac{1}{\frac 12 (a+b)} = \frac{2}{a+b} \]

\[ \tfrac 12 f(x) =
\frac{1}{\dfrac 1a + \dfrac 1b + c}
\]

\[ \cfrac{1}{1+\cfrac{2}{1 + \cfrac{3}{1+x}}} = \cfrac[r]{1}{1+\cfrac{2}{1+\cfrac[l]{3}{1+x}}} \]

% \usepackage{xfrac}
区别 $\sfrac 1a + b$ 和 $1/(a+b)$

\[
(a+b)^2 = \binom 20 a^2 + \binom 21 ab + \binom 22 b^2
\]

\[
\genfrac{[}{]}{0pt}{}{n}{1} = (n-1)!,
\qquad n > 0.
\]

\[ \stiring{n}{1} = (n-1)!, \qquad n > 0. \]
$\dstiring{n}{1} = (n-1)!, \qquad n > 0.$

$\tstiring{n}{1} = (n-1)!, \qquad n > 0.$

$\sqrt 4 = \sqrt[3]{8} = 2$

\[
\sqrt[n]{\frac{x^2 + \sqrt 2}{x+y}}
\]

\[
(x^p + y^q)^{\frac{1}{1/p+1/q}}
\]

\[
\sqrt[\uproot{16}\leftroot{-2}n]{\frac{x^2 + \sqrt 2}{x+y}}
\]

$\sqrt{\frac 12} < \sqrt{\vphantom{\frac12}2}$

% 数学支架 \mathstrut 表示有一个圆括号高度和深度的支架, 它常用来平衡不同高度和深度的字母
$\sqrt b \sqrt y$ \qquad
$\sqrt{\mathstrut b} \sqrt{\mathstrut y}$

\[
A = \begin{pmatrix}
  a_{11} & a_{12} & a_{13} \\
  0 & a_{22} & a_{23} \\
  0 & 0 & a_{33}
\end{pmatrix}
\]

\[
A = \begin{bmatrix}
  a_{11} & \dots & a_{1n} \\
  & \ddots & \vdots \\
  0 & & a_{nn}
\end{bmatrix}_{n\times n}
\]

% \usepackage{mathdots}
\[
A = \begin{bmatrix}
  a_{11} & \dots & a_{1n} \\
  & \iddots & \vdots \\
  0 & & a_{nn}
\end{bmatrix}_{n\times n}
\]

\[
\begin{pmatrix}
\begin{matrix}1&0\\0&1\end{matrix} & \text{\Large 0} \\
\text{\Large 0} & \begin{matrix}1&0\\0&-1\end{matrix}
\end{pmatrix}
\]

% \hdotsfor{列数}
\[
\begin{pmatrix}
  1 & \frac 12 & \dots & \frac 1n \\
  \hdotsfor{4} \\
  m & \frac m2 & \dots & \frac mn
\end{pmatrix}
\]

复数 $z = (x,y)$ 也可用矩阵 \begin{math}
  \left( \begin{smallmatrix}
    x & -y \\ y & x
  \end{smallmatrix} \right)
\end{math} 来表示.

\[
\sum_{\substack{0<i<n \\ 0<j<i}} A_{ij}
\]

\[
\sum_{\begin{subarray}{l}i<10 \\ j<100 \\ k<1000\end{subarray}} X(i,j,k)
\]

% matrix 等矩阵环境默认至多只有 10 列, 直接使用多于 10 列的矩阵会产生错误.
% 这个最大列数的限制是由计数器 MaxMatrixCols 控制的, 可以通过 \setcounter 等计数器命令临时或全局调整.
\[ \setcounter{MaxMatrixCols}{15}
\begin{Bmatrix}
0 & 0 & 0 & 0 & 0 & 1 & 0 & 1 & 0 & 0 & 1 & 1 & 1 & 0 & 1\\
1 & 1 & 1 & 1 & 1 & 0 & 1 & 0 & 1 & 1 & 0 & 0 & 0 & 1 & 0
\end{Bmatrix}
\]

% \usepackage{mathtools}
\[
\begin{pmatrix*}[r]
  10 & -10 \\ -20 & 3
\end{pmatrix*}
\]

\[ \bordermatrix{
  & 1 & 2 & 3 \cr
1 & A & B & C \cr
2 & D & E & F \cr} \]


$\mathnormal{ABCHIJXYZabchijxyz12345}$

$\mathit{ABCHIJXYZabchijxyz12345}$

$\mathrm{ABCHIJXYZabchijxyz12345}$

$\mathbf{ABCHIJXYZabchijxyz12345}$

$\mathsf{ABCHIJXYZabchijxyz12345}$

$\mathtt{ABCHIJXYZabchijxyz12345}$

$\mathcal{ABCHIJXYZ}$

% \usepackage{amssymb}
$\mathbb{ABCXYZ}$

$\Bbbk$

% bbold
$\mathbb{ABCXYZabcxyz123890}$

% bbm
$\mathbbm{ABCXYZabcxyz12}$

% mathrsfs
$\mathscr{ABCXYZ}$

% eucal
$\mathcal{ABCXYZ}$

% \usepackage{amssymb 或 eufrak}
$\mathfrak{ABCXYZabcxyz123890}$

$\aleph\beth\daleth\gimel$

$\pi\uppi\upalpha\alpha$

$\breve{a}\check{a}\hat{a}\grave{a}\acute{a}\tilde{a}\bar{a}\vec{a}\dot{a}\mathring{a}\ddot{a}$

$(abc)\spbreve(abc)\sptilde$

$\hbar\imath\jmath\ell\wp\Re\Im\partial\infty\prime\emptyset\nabla\surd\top\bot\angle\triangle\forall\exists\neg\flat\natural\sharp\clubsuit\diamondsuit\heartsuit\spadesuit
\backslash\backprime\hslash\varnothing\vartriangle\blacktriangle\triangledown\blacktriangledown\square\blacksquare\lozenge\blacklozenge\circledS\bigstar
\sphericalangle\measuredangle\nexists\complement\mho\eth\Finv\diagup\Game\diagdown\Bbbk$

$\# \& \% \$ \P \S \dag \ddag \copyright \pounds \checkmark \circledR \maltese \yen$

% \usepackage{bm}
% \hm 的效果需要实际字体支持
\textbf{勾股定理 $\bm{a^2+b^2=c^2}$}

勾股定理 $a^2+b^2=c^2$

\[ \bm u + \bm v = (1,0) + (0,1) \]
\[ \hm\int > \bm\int > \int \]

\begin{gather*}
a+b \quad a{+}b \quad a\mathord{+}b \\
\max n \quad {\max} n \quad \mathord{\max} n
\end{gather*}

\[ \mathrm{e}^{\uppi\mathrm{i}} + 1 = 0 \]

空集 $\varnothing$的基数是$0$, 自然数集 $\mathbb{N}= \{1,2,3,\ldots\}$的基数是 $\bm{\aleph_0}$, 则实数集 $\mathbb{R}$的基数 $\#\mathbb{R} = \bm{\aleph_1} = 2^{\bm{\aleph_0}}$.

$\varliminf\varlimsup\injlim\projlim\varinjlim\varprojlim$

\[  \operatorname*{Prob}_{\{1,\ldots,n\}}
      (\bar X) =
    \operatorname{card}(\varnothing)/n = 0. \]

\[ \mathop{\mathord{\sum}'}\limits_{i=1}^n A_n \]

\[ \int_0^1 \int_0^1 f(x,y) \int_0^1 \frac{\dif z}{g(x,y,z)} \dif x\dif y \]

$r = m \bmod n$

$x\equiv y \pmod b$

$x\equiv y \mod c$

$x\equiv y \pod d$

\[ \oint_{C} \kappa_{\mathrm g} \dif s + {\iint_{D} K \dif \sigma} = 2\uppi - \sum_{i=1}^n \alpha_i. \]

\[ \operatorname{Var}(X) = \operatorname{E}(X-\mu)^2 = \sum_{j=1}^\infty (x_j - \mu)^2 \operatorname{Pr}(X = x_j), \qquad \text{其中} \mu = \operatorname{E}\!X. \]

\[
\lim_{N\to +\infty} \frac{1}{2\uppi} \int_{-N}^{N}\hat{f}(\lambda)\mathrm{e}^{\mathrm{i}\lambda x}\dif \lambda = f(x).
\]

群 $G$ 的 $(H,K)$-双陪集为 $H\backslash G/K$.

$S\cup T = (S\cap T)\cup (S\setminus T)$

$\not\in$

$\notin$

$f(x) \defeq ax^2+bx+c$

\[
A \xleftarrow{0<x<1} B
\xrightarrow[x\leq 0]{x\geq 1} C
\]

% \usepackage{extarrows}
\[ \xlongleftarrow[xyz]{a+b+c} \]

$x =y \implies x+a=y+a$\\
$x=y \impliedby x+a=y+a$\\
$x=y \iff x\le y \And x\ge y$

运算 $\heartsuit$ 的交换律:
\[ a \mathbin{\heartsuit} b = b \mathbin{\heartsuit} a \]

$\forall x$, $\forall S$, $x\varnotin S$.

$\abs{x+y} \le \abs{x} + \abs{y}$

\[
\partial_x \partial_y \left[
\frac12 \left(x^2+y^2\right)^2 + xy
\right]
\]

\[
\left.
\int_0^x f(t,\lambda) \,\mathrm{d}t
\right|_{x=1}, \qquad
\lambda \in
\left[\frac12,\infty\right).
\]

\[
\Pr \left( X>\frac12
\middle\vert Y=0 \right)
= \left.
\int_0^1 p(t)\,\mathrm{d}t
\middle/ ( N^2+1 ) \right.
\]

\[
\biggl( \sum_{i=1}^n A_i \biggr) \cdot
\biggl( \sum_{i=1}^n B_i \biggr) > 0
\]

$ 1 + \Bigl(2 - \bigl(3 \times (4 \div 5) \bigr) \Bigr) $

\[ P = \biggl< \frac12 \biggr>, \qquad
M = \left< \begin{matrix}
  a & b \\ c & d \\
  \end{matrix}\right> \]

$a:b=ac:bc$

\[
\Pr(x\colon g(x)>5) = 0.25,
\qquad g\colon x \mapsto x^2
\]

\[ (1,\dots,n) \qquad 1+\dots+n \qquad a=\dots=z \]

\[ \prod_{i=1}^n a_i = a_1 \dotsm a_n \qquad \int_0^1\dotsi\int_0^1 \]

\begin{gather}
  a+b=b+a\\
  ab=ba
\end{gather}

\begin{gather*}
  3+5=5+3=8\\
  3\times 5 = 5\times3
\end{gather*}

\begin{gather}
  3^2 + 4^2 = t^2\notag\\
  5^2 + 12^2 = 13^2\notag\\
  a^2 + b^2 = c^2
\end{gather}

\begin{align}
  x &= t + \cos t + 1\\
  y &= 2\sin t
\end{align}

\begin{align*}
  x &= t & x &= \cos t & x &= t\\
  y &= 2t & y &= \sin(t+1) & y &= \sin t
\end{align*}

\begin{align*}
    & (a+b)(a^2-ab+b^2) \notag \\
={} & a^3 - a^2b + ab^2 + a^2b
      - ab^2 + b^2 \notag \\
={} & a^3 + b^3
\end{align*}

\begin{align*}
    & (a+b)(a^2-ab+b^2) \notag \\
= & a^3 - a^2b + ab^2 + a^2b
      - ab^2 + b^2 \notag \\
= & a^3 + b^3
\end{align*}

\begin{align*}
&\mathrel{\phantom{=}}
   (a+b)(a^2-ab+b^2) \notag \\
&= a^3 - a^2b + ab^2 + a^2b
   - ab^2 + b^2 \notag \\
&= a^3 + b^3 \label{eq:cubesum}
\end{align*}

\begin{flalign}
x &= t  & x &= 2 \\
y &= 2t & y &= 4
\end{flalign}

    \begin{alignat}{2}
      x &= \sin t &\quad&\text{水平方向} \\
      y &= \cos t &&\text{垂直方向}
    \end{alignat}
    
        \begin{alignat*}{6}
     &1 & &+2 & &+3 & &+4 & &+5 & &=15 \\
     &1 & &   & &+3 & &   & &+5 & &=9 \\
     &  & &+2 & &   & &+4 & &   & &=6
    \end{alignat*}

    \begin{align*}
    x^2 + 2x &= -1
    \intertext{移项得}
    x^2 + 2x + 1 &= 0
    \end{align*}
    
        \begin{align*}
    x^2 + 2x &= -1
    \shortintertext{移项得}
    x^2 + 2x + 1 &= 0
    \end{align*}
    设 $G$ 是一个带有运算 $*$ 的集合,则 $G$ 是\emph{群},当且仅当:
\begin{subequations}\label{eq:group}
    \begin{alignat}{2}
    \forall a,b,c &\in G, &\qquad (a*b)*c &= a*(b*c);\label{subeq:assoc}\\
    \exists e, \forall a &\in G, &  e*a &= a; \\
    \forall a, \exists b &\in G, &  b*a &= e.
    \end{alignat}
\end{subequations}
式~\eqref{eq:group} 的三个条件中,\eqref{subeq:assoc}~又称为结合律。

\begin{subequations}
\begin{alignat}{4}
&a_{11}x & &+a_{12}y & &+ a_{13}z & &= A\\
&a_{21}x & &+a_{22}y & &+ a_{23}z & &= B\\
&a_{31}x & &+a_{32}y & &+ a_{33}z & &= C
\end{alignat}
\end{subequations}

\begin{alignat}{2}
a& =b& c& =d\\
a'& =b'& c'& =d'
\end{alignat}

\begin{multline}
  a+b+c+d+e \\
  +f+g+h+i+j\\
  +k+l+m+n+o\\
  +p+q+r+s+t
\end{multline}

    \setlength{\multlinegap}{3em}
    \setlength{\multlinetaggap}{3em}
    \begin{multline*}
    1+2+3 \\ \shoveleft{+4+5+6} \\
    +7+8+9 \\
    \shoveright{+10+11+12} \\ +13+14+15
    \end{multline*}
    
        \begin{equation}  \begin{split}
    \cos 2x &= \cos^2 x - \sin^2 x \\
            &= 2\cos^2 x - 1
    \end{split}  \end{equation}

    \begin{equation}\label{eq:trigonometric}
    \begin{split}
    \frac12 (\sin(x+y) + \sin(x-y))
      &= \frac12(\sin x\cos y + \cos x\sin y) \\
      &\quad + \frac12(\sin x\cos y - \cos x\sin y) \\
 \overline{}     &= \sin x\cos y
    \end{split}
    \end{equation}

% \usepackage{breqn}
\iffalse
\begin{dmath}\label{eq:trigonometric}
\frac12 (\sin(x+y) + \sin(x-y)) = \frac12(\sin x\cos y + \cos x\sin y)
+ \frac12(\sin x\cos y - \cos x\sin y) = \sin x\cos y
\end{dmath}
\fi

    \begin{equation}\label{eq:dirichlet}
    D(x) = \begin{cases}
    1, & \text{if } x \in \mathbb{Q}; \\
    0, & \text{if } x \in
         \mathbb{R}\setminus\mathbb{Q}.
    \end{cases}
    \end{equation}

\[ \left\lvert x - \frac12 \right\rvert
= \begin{dcases}
x-\frac12,  & x \geq \frac12;\\
\frac12-x,  & x < \frac12.
\end{dcases}  \]

    % \usepackage{cases}
    \begin{numcases}{f(x)=}
      1/q, & if $x = p/q \in \mathbb{Q}$; \\
      0,   & else.
    \end{numcases}
    
    \[  \left. \begin{gathered}
      S \subseteq T \\
      S \supseteq T
    \end{gathered} \right\}
    \implies S = T  \]
    
    % \usepackage{mathtools}
    \[  \text{比较曲线}
    \left\{ \begin{lgathered}
      x = \sin t, y = \cos t \\
      x = t + \sin t, y = \cos t
    \end{lgathered} \right.  \]

    \begin{equation}\label{eq:trinary}
    \begin{aligned} x+y &= -1 \\ x+y+z &= 2 \\ xyz &= -6 \end{aligned}
    \implies
    \begin{aligned} x+y &= -1 \\ xy &= -2 \\ z &= 3 \end{aligned}
    \implies
    \begin{alignedat}{3}
                 x &= 1,  &\quad y &= -2, &\quad z &= 3 \\
    \text{或\ }  x &= -2, &      y &= 1,  &      z &= 3
    \end{alignedat}
    \end{equation}

    % \usepackage{mathtools}
    \newcommand\Set[2]{%
      \left\{#1\ \middle\vert\ #2 \right\}}
    \[  \Omega = \Set{x}{\begin{multlined}
    x^7+x^6+x^5 \\ +x^4+x^3+x^2 \\ +x+1=0
    \end{multlined}}  \]

    \begin{align*}
    2^5 &= (1+1)^5 \\
    &= \begin{multlined}[t]
    \binom50\cdot 1^5 + \binom51\cdot 1^4 \cdot 1
      + \binom52\cdot 1^3 \cdot 1^2 \\
    + \binom53\cdot 1^2 \cdot 1^3 + \binom54\cdot 1 \cdot 1^4
      + \binom55\cdot 1^5
    \end{multlined} \\
    &= \binom50 + \binom51 + \binom52 + \binom53 + \binom54 + \binom55
    \end{align*}
\iffalse
\[
f(x) \triangleq \begin{dcases}
  \tfrac32 - x, & \tfrac12 < x \leq 1,\\
  \tfrac34 - x, & \tfrac14 < x \leq \tfrac12,\\
  \tfrac38 - x, & \tfrac18 < x \leq \tfrac14,\\
  \phantom{\tfrac38}\vdots & \phantom{\tfrac18 < x}\vdots
\end{dcases}
\]
\fi

\[
f(x) \triangleq \begin{cases}
  \frac32 - x, & \frac12 < x \leq 1,\\
  \frac34 - x, & \frac14 < x \leq \frac12,\\
  \frac38 - x, & \frac18 < x \leq \frac14,\\
  \phantom{\frac32 -}\!\!\!\vdots & \phantom{\frac18 < x}\!\!\!\vdots
\end{cases}
\]

\[
\underline{\smash{\int f(x)\,\mathrm{d}x}} \underline{\int f(x)\,\mathrm{d}x} \underline{\smash[b]{\int f(x)\,\mathrm{d}x}} \underline{\smash[t]{\int f(x)\,\mathrm{d}x}} \overline{\smash{\int f(x)\,\mathrm{d}x}} \overline{\int f(x)\,\mathrm{d}x} \overline{\smash[b]{\int f(x)\,\mathrm{d}x}} \overline{\smash[t]{\int f(x)\,\mathrm{d}x}}
\]

\[
\sqrt{x} + \sqrt{y} + \sqrt{z}  \sqrt{x} + \sqrt{\smash[b]{y}} + \sqrt{z}
\]
\end{document}