\documentclass{ctexart}
\usepackage{amsmath}
\usepackage{mathtools}
\usepackage{dcolumn}
\usepackage{arydshln}
\usepackage{siunitx}
\usepackage{booktabs}
\usepackage{multirow}
\usepackage{rotating,makecell}
\usepackage{diagbox}
\usepackage{tabularx}
\usepackage[delarray]{tabu}
\usepackage{blkarray}
\usepackage{graphicx}
\usepackage{hhline}

\setlength{\parindent}{0pt}

\begin{document}
\begin{tabular}{lcr}
left & center & right \\
本列左对齐 & 本列居中 & 本列右对齐
\end{tabular}

\begin{tabular}{ll}
\bfseries 功能 & \bfseries 环境 \\
表格 & \ttfamily tabular \\
对齐 & \ttfamily tabbing \\
\end{tabular}

\[
\begin{array}{r|r}
\frac12 & 0 \\
\hline
0 & -\frac12 \\
\end{array}
\]

\begin{tabular}[b]{c}
上 \\ 中间 \\ 下
\end{tabular}
与底部对齐.

\begin{tabular}{|rr|}
\hline
输入 & 输出 \\
\hline
$-2$ & 4 \\
0 & 0 \\
2 & 4 \\
\hline
\end{tabular}
\qquad
输入与输出有关系 $y = x^2$

\begin{tabular}{|c|rrr|p{4em}|}
\hline
姓名 & 语文 & 数学 & 外语 & 备注 \\
\hline
张三 & 87 & 100 & 93 & 优秀 \\
李四 & 75 & 63  & \emph{52} & 补考另行通知 \\
王小二 & 80 & 82 & 78 & \\
\hline
\end{tabular}

\begin{tabular}{|c|r@{.}l|}
\hline
收入 & 12345&6 \\
\hline
支出 &   765&43 \\
\hline
结余 & 11580&17 \\
\hline
\end{tabular}

    \[
    \begin{array}{|c|*{3}{r@{.}l|}}  % 相当于 |c|r@{.}l|r@{.}l|r@{.}l|
    \hline
    \text{收入} & 12345&6  & 5000&0 &   1020&55 \\ \hline
    \text{支出} &   765&43 & 5120&5 &  98760&0  \\ \hline
    \text{节余} & 11580&17 & -120&5 & -97739&45 \\ \hline
    \end{array}
    \]

    % 导言区 \usepackage{dcolumn}
    \newcolumntype{d}{D{.}{.}{2}}
    \begin{tabular}{|c|*{3}{d|}}  % 相当于 |c|d|d|d|
    \hline
    姓名 & \multicolumn{1}{c|}{张三} & \multicolumn{1}{c|}{李四}
         & \multicolumn{1}{c|}{王五} \\ \hline
    收入 & 12345.6  & 5000   &   1020.55 \\ \hline
    支出 &   765.43 & 5120.5 &  98760    \\ \hline
    节余 & 11580.17 & -120.5 & -97739.45 \\ \hline
    \end{tabular}

    \verb=tabular= 环境可以在
    $\left(\begin{tabular}{@{}c@{}}
      文本 \\ 数学
    \end{tabular}\right)$
    模式下通用。

    \verb=tabular= 环境可以在
    $\left(\begin{tabular}{c}
      文本 \\ 数学
    \end{tabular}\right)$
    模式下通用。

    % 第 1 列前是原始间距,第 2 列前只有 1em 间距
    % 第 3、4 列前则是原始间距加 1em
    \begin{tabular}{|c|@{\extracolsep{1em}}c|c|c|}
    \hline
      1 & 2 & 3 & 4 \\
      1 & 2 & 3 & 4 \\
    \hline
    \end{tabular}

{
\renewcommand\arraystretch{2}
\begin{tabular}{|l|r|}
\hline
这是一个 & 宽松的表格 \\ \hline
loose & table \\ \hline
\end{tabular}
}

    % 导言区 \usepackage{array}
    \begin{tabular}[b]{|c|}
    \firsthline
    上 \\ 中间 \\ 下 \\
    \lasthline
    \end{tabular}
    与底部对齐。

    \begin{tabular}[b]{|c|}
    \hline
    上 \\ 中间 \\ 下 \\
    \hline
    \end{tabular}
    与底部对齐。
\[
\left(
\begin{array}{ccc|c}
a_{11} & a_{12} & a_{13} & b_1 \\
a_{21} & a_{22} & a_{23} & b_2 \\
a_{31} & a_{32} & a_{33} & b_3 \\
\end{array}
\right)
\]

\[
\left(
\begin{array}{@{}ccc|c@{}}
a_{11} & a_{12} & a_{13} & b_1 \\
a_{21} & a_{22} & a_{23} & b_2 \\
a_{31} & a_{32} & a_{33} & b_3 \\
\end{array}
\right)
\]

    % \usepackage{arydshln}
    \[
    \left(
    \begin{array}{@{}ccc:c@{}}
      a_{11} & a_{12} & a_{13} & b_1 \\
      a_{21} & a_{22} & a_{23} & b_2 \\
      a_{31} & a_{32} & a_{33} & b_3 \\
      \cdashline{1-3}
      0 & 0 & 0 & b_4 \\
    \end{array}
    \right)
    \]

\begin{tabular}{|c|cc|c|}
\hline
人数 & 患慢性支气管炎 & 未患慢性支气管炎 & 合计 \\
\hline
吸烟 & 43 & 162 & 205 \\
不吸烟 & 13 & 121 & 134 \\
\hline
合计 & 56 & 283 & 339 \\
\hline
\end{tabular}

% \usepackage{booktabs}
\begin{tabular}{S[separate-uncertainty = true]}
\toprule
数量 \\
\midrule
-2147483648 \\
3.14159265 \\
2.99792458e8 \\
3.55+-0.02 \\
\bottomrule
\end{tabular}

\begin{tabular}{|r|r|}
\hline
\multicolumn{2}{|c|}{成绩} \\
\hline
语文 & 数学 \\
\hline
87 & 100 \\
\hline
\end{tabular}

\begin{tabular}{|r|r|}
\hline
\multicolumn{2}{c}{成绩} \\
\hline
语文 & 数学 \\
\hline
87 & 100 \\
\hline
\end{tabular}

    \begin{tabular}{|r|r|}
      \hline
      \multicolumn{1}{|c|}{输入} &
      \multicolumn{1}{c|}{输出} \\ \hline
      1 & 1 \\ 5 & 25 \\ 15 & 225 \\ \hline
    \end{tabular}

    \begin{tabular}{|c|r|r|}
    \hline
    & \multicolumn{2}{c|}{成绩} \\ \cline{2-3}
    姓名 & 语文 & 数学 \\ \hline
    张三 & 87 & 100 \\ \hline
    \end{tabular}

\begin{tabular}{|c|}
\hline
1 \\ \hline
1 \vline\ 2 \\ \hline % 加一个空格的间距
1 \vline\ 2 \vline\ 3 \\ \hline
\end{tabular}

\begin{tabular}{|c|}
\hline
1 \\ \hline
\begin{tabular}{@{}c|c@{}}1 & 2\end{tabular} \\ \hline
\begin{tabular}{@{}c|c|c@{}} 1 & 2 & 3\end{tabular} \\ \hline
\end{tabular}

% 导言区  \usepackage{multirow}
\begin{tabular}{|c|r|r|}
\hline
\multirow{2}*{姓名} & \multicolumn{2}{c|}{成绩} \\ \cline{2-3} & 语文 & 数学 \\ \hline
张三 & 87 & 100 \\ \hline
\end{tabular}

    % 导言区 \usepackage{makecell}
    \begin{tabular}{|r|r|}
    \hline
    \makecell{处理前\\数据} &
    \makecell{处理后\\数据} \\ \hline
    4934 & 8945 \\
    \hline
    \end{tabular}

    % 导言区 \usepackage{makecell}
    \begin{tabular}{|r|r|}
    \hline
    \thead{处理前\\数据} &
    \thead{处理后\\数据} \\ \hline
    4934 & 8945 \\
    \hline
    \end{tabular}

    % 导言区 \usepackage{rotating,makecell}
    \settowidth\rotheadsize{\theadfont 数学课}
    \begin{tabular}{|c|c|}
    \hline
    \thead{姓名} & \rothead{数学课\\成绩} \\\hline
    张三 & 100 \\\hline
    \end{tabular}

    % 导言区 \usepackage{multirow,makecell}
    \begin{tabular}{|c|r|}
    \hline
    \multirowcell{3}{各科\\成绩} & 78 \\
    \cline{2-2} & 82 \\ \cline{2-2}
     & 86 \\ \hline
    \end{tabular}

    % 导言区 \usepackage{diagbox}
    \begin{tabular}{|c|*{4}{c}|}
    \hline
    \diagbox{天干}{地支} & 子 & 丑 & 寅 & 卯 \\
    \hline
    甲 &  1 && 51 & \\
    乙 &&  2 && 52 \\
    丙 & 13 && 3 & \\
    丁 && 14 && 4\\
    \hline
    \end{tabular}

    % 导言区 \usepackage{diagbox}
    \begin{tabular}{|c|*{4}{c}|}
    \hline
    \diagbox{天干}{序号}{地支} & 子 & 丑 & 寅 & 卯 \\
    \hline
    甲 &  1 && 51 & \\
    乙 &&  2 && 52 \\
    丙 & 13 && 3 & \\
    丁 && 14 && 4\\
    \hline
    \end{tabular}

\begin{tabular}{cccc}
\cline{2-2}
张家村 & \multicolumn{1}{|c|}{三里} & &  \\
\cline{2-3}
李家屯 & \multicolumn{1}{|c|}{六里} & \multicolumn{1}{c|}{三里} & \\
\cline{2-4}
王家庄 & \multicolumn{1}{|c|}{五里} & \multicolumn{1}{c|}{八里} & \multicolumn{1}{c|}{七里} \\
\cline{2-4}
& 赵镇 & 张家村 & 李家屯 \\
\end{tabular}

    \begin{tabular*}{\textwidth}{|c@{\extracolsep{\fill}}ccccc|}
    \hline
      数字 & 1 & 2 & 3 & 4 & 5 \\
      字母 & A & B & C & D & E \\
      天干 & 甲 & 乙 & 丙 & 丁 & 戊 \\
    \hline
    \end{tabular*}

    % 导言区 \usepackage{tabularx}
    \begin{tabularx}{\textwidth}{|c|X|X|X|X|X|}
    \hline
      数字 & 1 & 2 & 3 & 4 & 5 \\ \hline
      字母 & A & B & C & D & E \\ \hline
      天干 & 甲 & 乙 & 丙 & 丁 & 戊 \\
    \hline
    \end{tabularx}

    \begin{tabularx}{\textwidth}{|X|X|X|X|X|X|}
    \hline
      数字 & 1 & 2 & 3 & 4 & 5 \\ \hline
      字母 & A & B & C & D & E \\ \hline
      天干 & 甲 & 乙 & 丙 & 丁 & 戊 \\
    \hline
    \end{tabularx}

    % 导言区 \usepackage{tabularx}
    \newcolumntype{Y}{>{\centering\arraybackslash}X}
    \begin{tabularx}{\textwidth}{|c|Y|Y|Y|Y|Y|}
    \hline
      数字 & 1 & 2 & 3 & 4 & 5 \\ \hline
      字母 & A & B & C & D & E \\ \hline
      天干 & 甲 & 乙 & 丙 & 丁 & 戊 \\
    \hline
    \end{tabularx}

    \begin{tabular}{ccccc}
      \toprule
      序号 & 性别 & 年龄 & 身高/cm & 体重/kg \\
      \midrule
      1 & F & 14 & 156 & 42 \\
      2 & F & 16 & 158 & 45 \\
      3 & M & 14 & 162 & 48 \\
      4 & M & 15 & 163 & 50 \\
      \bottomrule
    \end{tabular}

    % 导言区 \usepackage{multirow,booktabs}
    \begin{tabular}{*{6}{c}}
    \bottomrule
    \multirow{2}*{姓名} & \multicolumn{2}{c}{文科} &
      \multicolumn{2}{c}{理科} & \\
    \cmidrule(lr){2-3}\cmidrule(lr){4-5}\cmidrule(lr){6-6}
      \morecmidrules\cmidrule(lr){6-6}
    & 历史 & 文学 & 物理 & 化学 & 总评 \\
    \midrule
    张三 & A & A & B & A & A \\
    \bottomrule
    \end{tabular}

    % 导言区 \usepackage{makecell}
    \begin{tabular}{c|cc}
    \Xhline{2pt}
    自变量 & \multicolumn{2}{c}{因变量} \\
    \Xcline{2-3}{0.4pt}
    半径 & 周长 & 面积 \\
    \Xhline{1pt}
    1.00 & 6.28 & 6.28 \\
    2.00 & 12.57 & 12.57 \\
    3.00 & 18.85 & 28.27 \\
    \Xhline{2pt}
    \end{tabular}

    % \usepackage{makecell}
    \newcolumntype{V}{!{\vrule width 2pt}}
    \begin{tabular}{Vc|ccV}
    \Xhline{2pt}
    自变量 & \multicolumn{2}{cV}{因变量} \\
    \Xcline{2-3}{0.4pt}
    半径 & 周长 & 面积 \\
    \Xhline{1pt}
    1.00 & 6.28 & 6.28 \\
    2.00 & 12.57 & 12.57 \\
    3.00 & 18.85 & 28.27 \\
    \Xhline{2pt}
    \end{tabular}

    \begin{tabular}{|c||cc|}
    \hline\hline
    自变量 & \multicolumn{2}{c|}{因变量} \\
    \cline{2-3}
    半径 & 周长 & 面积 \\
    \hline\hline
    1.00 & 6.28 & 6.28 \\
    2.00 & 12.57 & 12.57 \\
    3.00 & 18.85 & 28.27 \\
    \hline\hline
    \end{tabular}

    % \usepackage{hhline}
    \begin{tabular}{|c||cc|}
    \hhline{|=:t:==|}
    半径 & 周长 & 面积 \\
    \hhline{|=::==|}
    1.00 & 6.28 & 6.28 \\
    \hhline{|=:b:==|}
    \end{tabular}

    % \usepackage{arydshln}
    \[
    \left(
    \begin{array}{@{}ccc:c@{}}
      a_{11} & a_{12} & a_{13} & b_1 \\
      a_{21} & a_{22} & a_{23} & b_2 \\
      a_{31} & a_{32} & a_{33} & b_3 \\
      \cdashline{1-3}
      0 & 0 & 0 & b_4 \\
    \end{array}
    \right)
    \]

    % \usepackage{arydshln}
    \setlength\dashlinedash{1pt}
    \setlength\dashlinegap{2pt}
    \begin{tabular}{:cc:cc:}
    \hdashline
    上 & 上 & 上 & 上 \\
    \cdashline{1-2}
    下 & 下 & 下 & 下 \\
    \hdashline
    \end{tabular}

    % \usepackage{arydshln}
    \begin{tabular}{;{8pt/2pt}cc;{2pt/2pt}cc;{8pt/2pt}}
    \hdashline[8pt/2pt]
    上 & 上 & 上 & 上 \\
    \cdashline{1-2}[2pt/2pt]
    下 & 下 & 下 & 下 \\
    \hdashline[8pt/2pt]
    \end{tabular}

% \usepackage{array} 或调用其他依赖 array 的宏包
\begin{tabular}{>{\bfseries}c|>{\itshape}c>{$}c<{$}}
\hline
姓名 & \textnormal{得分} & \multicolumn{1}{c}{额外加分} \\
\hline
张三 & 85 & +7 \\
李四 & 82 & 0 \\
王五 & 70 & -2 \\
\hline
\end{tabular}

    % \usepackage{array}
    \begin{tabular}{|>{$}r<{$}|>{\setlength\parindent{2em}}m{15em}|%
      >{\centering\arraybackslash}m{4em}|}
    \hline
    \pi & 希腊字母,多用于表示圆周率,也常用作变量。表示圆周率时多使用
      直立体。 & 常用 \\
    \hline
    \aleph & 希伯来字母的第一个,在数学中通常用于表示特殊集合的基数。
      & 不常用 \\
    \hline
    \end{tabular}

    % \usepackage{array}
    \begin{tabular}{c!{$\Rightarrow$}c}
    张三 & 85 \\
    李四 & 82 \\
    王五 & 70 \\
    \end{tabular}

    % \usepackage{tabu}
    \begin{tabu}{ccc}
    \hline
    \rowfont{\bfseries}
    姓名 & 得分 & 额外加分 \\
    \hline
    张三 & 85 & $+7$ \\
    \rowfont{\itshape}
    李四 & 82 & 0 \\
    王五 & 70 & $-2$ \\
    \hline
    \end{tabu}

    \[ \left\{ \begin{matrix}
      1 & 2 \\ 3 & 4
    \end{matrix} \right. \]

    % \usepackage[delarray]{tabu}
    \[  \begin{tabu}({cc})
        1 & 2 \\
        3 & 4
    \end{tabu}  \]

    % \usepackage{blkarray}
    % 如果不用 &| 说明,则竖线 | 将会被看成是中间一列的内容
    \begin{blockarray}{|l|c&|r|}
      张三 & Zhang & 80 \\
           % 不用在 r 后面用 |,也不影响表格线
      李四 & \BAmulticolumn{1}{r}{Li} & 78 \\
      王五 & Wang & 100 \\
    \end{blockarray}

   \begin{blockarray}{|l|c|r|}
      张三 & Zhang & 80 \\
      李四 & \BAmulticolumn{1}{r}{Li} & 78 \\
      王五 & Wang & 100 \\
    \end{blockarray}

    % \usepackage{blkarray}
    \[  \begin{blockarray}{(cc]}
      1 & 2 \\
      3 & 4
    \end{blockarray}  \]

    % \usepackage{blkarray}
    \[  \left[
    \begin{blockarray}{*4r}
    \begin{block}{(rr)rr}
     a & -b & 0 & 0 \\
    -c &  d & 0 & 0 \\
    \end{block}
    \begin{block}{rr(rr)}
    0 & 0 & -a &  b \\
    0 & 0 &  c & -d \\
    \end{block}
    \end{blockarray}
    \right]  \]

    \[  \left[
    \begin{blockarray}{*4r}
    \begin{block*}{(rr)rr}
     a & -b & 0 & 0 \\
    -c &  d & 0 & 0 \\
    \end{block*}
    \begin{block}{rr(rr)}
    0 & 0 & -a &  b \\
    0 & 0 &  c & -d \\
    \end{block}
    \end{blockarray}
    \right]  \]

    \[
    \begin{blockarray}{ccc}
    \begin{block}{cc\}\BAmultirow{4em}}
    1 & 2 & 自然数 \\
    3 & 4 & {} \\ % 空白 {} 占位
    \end{block}
    \begin{block}{cc\}l}
    -1.5 & \frac12 & \BAmultirow{4em}{实数} \\
    3.5 & 40 & \\
    \end{block}
    \end{blockarray}
    \]

\[
  \begin{blockarray}{cc|cccc|cccc}
    & 1\dots 18 & 19 & 20 & 21 & 22 & 23 & 24 & 25 & 26 \\
    \begin{block}{c(c|cccc|cccc@{\hspace*{5pt}})}
    A'_1 & A_1 & \BAmulticolumn{4}{c|}{\multirow{4}{*}{$I$}}&\BAmulticolumn{4}{c}{\multirow{4}{*}{$I$}}\\
    A'_2 & A_2 & &&&&&&&\\
    A'_3 & A_3 & &&&&&&&\\
    A'_4 & A_4 & &&&&&&&\\
    \cline{1-10}% don't use \hline
    B'_1 & B_1 & \BAmulticolumn{4}{c|}{\multirow{4}{*}{$J$}}&\BAmulticolumn{4}{c}{\multirow{4}{*}{$I$}}\\
    B'_2 & B_2 & &&&&&&&\\
    B'_3 & B_3 & &&&&&&&\\
    B'_4 & B_4 & &&&&&&&\\
    \end{block}
  \end{blockarray}
\]

\[M=
\begin{blockarray}{ccc}
A & B & C \\
\begin{block}{[c|cc]}
1 & 2 & 3 \\
4 & 5 & 6 \\
\end{block}
\end{blockarray},\quad
N=
\begin{blockarray}{ccc}
D & E & F \\
\begin{block}{[c|cc]}
7 & 8 & 9 \\
10 & 11 & 12\\
\end{block}
\end{blockarray}
\]
\[
\begin{blockarray}{rccrl}
 & 1 & 2 &  &  \\
\begin{block}{r[cc]rl}
1 & \alpha & \beta  & \leftarrow & i \\
2 & \gamma & \delta &            &   \\
\end{block}
  &        & \uparrow &          &   \\
  &        & j                   &   \\
\end{blockarray}
\]


\end{document} 