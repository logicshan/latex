\documentclass[UTF8]{ctexart}

\usepackage{amsmath,amssymb,amsthm}
\usepackage{mathrsfs}
\usepackage{color}
\usepackage{tikz-cd}
\usepackage{biblatex}
\addbibresource{modular.bib}

\DeclareMathOperator{\charac}{char}
\DeclareMathOperator{\tr}{tr}
\newcommand{\abs}[1]{\left\lvert#1\right\rvert}

\newtheorem*{prop}{Proposition}
\newtheorem*{defi}{Definition}
\newtheorem*{thm}{Theorem}

\begin{document}
\begin{enumerate}
\item \textbf{Proposition}. If $\charac(K)$ divides $\abs{G}$, the $K[G]$ is not semisimple.

\item For the case where $\charac(K)$ divides $\abs{G}$, we have \textit{modular representation theory}.

\item some knowledge of complete discrete valuation rings

1. $R$ is a complete discrete valuation ring with maximal ideal generated by $\pi\in R$. 则 $R$ 是一个有唯一非零极大理想的主理想整环(则 $R$ 为无零因子的交换环, 且每个理想都为主理想).

2. $K$ 为 $R$ 的分式域(field of fractions)(即包含$R$的最小的域).

3. $k := R/(\pi)$ 是剩余域(residue field).

4. $p := \charac(k)$ divides $\abs{G}$ and $\charac(K) = 0$.

\item Representations of $G$ over $k$ are studied by first lifting them to $R$, then projecting to $K$. (这句话需要进一步解释)

Let $R_k(G), R_K(G)$ be the Grothendieck group of finitely-generated modules over $k[G], K[G]$ respectively. Since $\charac(K) = 0$, $K[G]$ is semisimple. \textcolor{red}{Since $k[G]$ is of finite dimension over $k$, it is artinian;} we let $J$ be its Jacobian radical. Finally, let $P_k(G)$ be the Grothendieck group of finitely-generated projective modules over $k[G]$. From earlier results:

1. $R_k(G), R_K(G), P_k(G)$ are free abelian groups, with the following bases, which we will fix from now on.
\begin{itemize}
  \item Basis of $R_k(G)$ or $R_K(G)$: $[M]$ for simple modules $M$.
  \item Basis of $P_k(G)$: $[P]$ for finitely-generated indecomposable projective modules $P$.
\end{itemize}

2. We have a map $c : P_k(G) \to R_k(G)$ taking $[P]$ to $[P]$. (这时候右边的$[P]$还没有用 $R_k(G)$中的基来表示)

3. We have an isomorphism $P_k(G) \cong R_k(G)$, taking $[P]$ to $[P/JP]$; this corresponds to identity matrix (under the above bases).($P_k(G)$ 和 $R_k(G)$ 之间是存在同构的, 但在 cde 三角中定义 $c : P_k(G)\to R_k(G)$ 时, $c$ 不是同构.

注意: 以下说明 $R_k(G)$ 中的元素怎样用 $R_k(G)$ 的基来表示. $[M]\in R_k(G)$, $M$ 的合成因子为 $N_0,N_1,\ldots,N_k$, 则$[M]$ 可表示为 $[N_0] + [N_1] +\ldots+[N_k]\in R_k(G)$.
若 $[P]\in P_k(G), P\cong Q_1\oplus \ldots\oplus Q_r$, 则 $[P]$ 可用$P_k(G)$的基表示为$[Q_1] + \dots + [Q_r]$.

上面的表述不够严谨. 下面进一步说明. 若记$k[G]$上的有限生成模的 Grothendieck group 为 $G_0(k[G])$,
$R_k(G)$为单$k[G]$-模的等价类所生成的自由 Abel 群, 那么 $G_0(k[G])\cong R_k(G)$, 对$[M]\in G_0(k[G])$, $M$ 的合成因子为 $N_0,N_1,\ldots,N_k$, 则$[M]$ 可表示为 $[N_0] + [N_1] +\ldots+[N_k]\in R_k(G)$.\cite{exact-sequences-and-the-grothendieck-group}
$P_k(G)$的情况同理.\cite{projective-modules-and-the-grothendieck-group}

\textbf{Note}: if $[M] = [N] \in R_k(G)$, then $M$ and $N$ have identical composition factors but $M$ and $N$ may not be isomorphic. On the other hand, if $[M] = [N]\in P_k(G)$, then the decomposition factors of $M$ and $N$ are the same and thus $M\cong N$.


\item Definition of $d$

To define $d : R_K(G) \to R_k(G)$, let $M$ be a finitely-generated $K[G]$-module.

\begin{prop}
There is an $R[G]$-module $N$ such that $K\otimes_R N \cong M$. The class of this module $[N/\pi N]\in R_k(G)$ depends only on $M$.

We thus define $d : R_K(G) \to R_k(G)$ via: given $[M]\in R_K(G)$ pick an $R[G]$-module $N$ such that $K\otimes_R N\cong M$. We then define $d([M]) := [N/\pi N]$.
\end{prop}

\item Definition of $e$

This is given in two steps: first we consider the map $\psi : P_R(G)\to P_k(G), [P]\to [P/\pi P]$ where $P_R(G)$ is the Grothendieck group of finitely-generated projective $R[G]$-modules. [Warning: $R[G]$ is not artinian.] We can show that $\psi$ is an isomorphism. Next, we define $e$ as a composition
\[
e : P_k(G) \overset{\psi^{-1}}{\longrightarrow} P_R(G) \longrightarrow R_K(G)
\]
where the second map takes $[M]\in P_R(G)$ to $[K\otimes_R M]\in R_K(G)$.

\item For each projective finitely-generated $k[G]$-module $P$, we have a unique projective finitely-generated $R[G]$-module denoted $\tilde{P}$ for which $\tilde{P}/\pi\tilde{P}\cong P$.

\item \textbf{Proposition}. The matrices for $e$ and $d$ are transpose of each other. Hence, $c = de$ is positive-semidefinite and symmetric.

\item \textbf{Modular Characters}

Recall that a finitely-generated $K[G]$-module $M$ can be represented by its character $\chi_M : G\to K$ where $\chi_M(g) := \tr(g : M\to M)$, its trace as a $K$-linear map. Since $K[G]$ is semisimple, standard character theory says $\chi_M$ uniquely determines $M$. Now, $\chi_M$ is a class function (i.e. it is constant on each conjugacy class) and in fact, forms an orthonormal basis of the space of class functions, where orthonormality follows from Schur's lemma.

We would like to produce a similar theory for $k[G]$-modules. The naive approach of taking $G\to k$ where $g\mapsto \tr(g:M\to M)$ does not lead to a satisfactory theory. \textcolor{red}{The better approach is to lift the character to a function $G\to R$.}

\begin{defi}
An element $g$ of $G$ is said to be \textcolor{blue}{$p$-regular} if its order is coprime to $p$; if it is not $p$-regular, we say it is \textcolor{blue}{$p$-singular}. The collection of $p$-regular elements of $G$ is denoted $G_{\mathrm{reg}}$. Note that this is a union of conjugacy classes.
\end{defi}

Now assume $K$ contains all $n$-th roots of $1$, where $n = \abs{G}$. It follows that $k$ contains all $m$-th roots of $1$, where $m$ is the least common multiple of the orders of the elements of $G_{\mathrm{reg}}$. The $m$-th roots of $1$ in $k$ are all distinct and the canonical map $R\to R/(\pi)$ induces(根据分式域的万有性质) a bijection between the $m$-th roots of $1$ in $K$ and in $k$. Denote this bijection by $\lambda$.

Now we are ready to define modular characters.

\begin{defi}
Let $M$ be a finitely-generated $k[G]$-module and $g\in G_{\mathrm{reg}}$. Consider the $k$-linear map $g:M\to M$. Let $r$ be the order of $g$; since $r|m$, $k$ contains all the eigenvalues of $g$, denoted $\zeta_1(g), \ldots, \zeta_r(g)$.

Now the \textcolor{blue}{modular character} of $M$ is defined as follows:
\[
\varphi_M : G_{\mathrm{reg}} \to K, \quad g \mapsto \sum_i \lambda^{-1}(\zeta_i(g))
\]
\end{defi}

\item Idempotents and Decomposition\cite{idempotents-and-decomposition}

Let $R$ be a general ring, not necessarily commutative. An element $x\in R$ is said to be \textbf{idempotent} if $x^2 = x$.

Throughout this article, we shall focus on idempotents which commute, i.e. $ef = fe$. A set of idempotents $\{e_i\}$ is said to be \textbf{orthogonal} if $e_{i}e_{j} = 0$ for all $i \neq j$. The following are easy to prove.

\begin{enumerate}
  \item The sum of two orthogonal idempotents is also an idempotent.
  \item If $e$ is any idempotent, then $e$ and $1-e$ are orthogonal idempotents.
\end{enumerate}

The key result is the following.

\begin{thm}
Let $R$ be any ring. There is a $1-1$ correspondence between the following:

\begin{itemize}
  \item a decomposition $R = I_1 \oplus \ldots \oplus I_n$ as a direct sum of left ideals, and
  \item orthogonal idempotents $e_1,\ldots,e_n$ such that $e_1 + \ldots + e_n = 1$.
\end{itemize}

\end{thm}

Indecomposable Left Ideals

Now suppose $R$ is an artinian ring (and hence noetherian by the Hopkins-Levitzki theorem). The Krull-Schmidt theorem says that $R$ is a direct sum of indecomposable projective modules $I_i$. Such modules correspond to primitive idempotents.
\begin{defi}
A non-zero idempotent $e$ is said to be \textit{primitive} if it cannot be written as a sum of non-zero orthogonal idempotents $e = f_1 + f_2$.
\end{defi}
\begin{prop}
If $e$ is an idempotent, then $Re$ is indecomposable if and only if $e$ is primitive.
\end{prop}

\begin{defi}
An idempotent $e$ of ring $R$ is said to be \textbf{central} if it lies in $Z(R)$.
\end{defi}

\begin{thm}
There is a bijection between:
\begin{itemize}
  \item an isomorphism $R\cong R_1 \times \ldots \times R_n$ as a product of rings;
  \item a decomposition $R = I_1 \oplus \ldots \oplus I_n$ as a direct sum of (two-sided) ideals;
  \item an expression $1 = e_1 + \ldots + e_n$ as a sum of orthogonal central idempotents.
\end{itemize}
\end{thm}

\begin{defi}
Let $e$ be a non-zero central idempotent (in $Z(R)$). We say $e$ is \textbf{centrally primitive} if we cannot write $e$ as a sum of two non-zero orthogonal central idempotents.
\end{defi}

\end{enumerate}
\newpage
\printbibliography
\end{document} 