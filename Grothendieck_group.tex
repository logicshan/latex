\documentclass{ctexart}

\usepackage{amsmath,amssymb,mathrsfs,color}

\setlength{\parindent}{0pt}

\begin{document}
Grothendieck group 定义在 commutative monoid 上, 某种意义上是包含原来交换幺半群的最小的 Abel 群.
The Grothendieck group construction is a functor from the category of abelian semigroups to the category of abelian groups.

The zeroth algebraic $K$ group $K_0(R)$ of a (not necessarily commutative) ring $R$ is the Grothendieck group of the monoid consisting of isomorphism classes of finitely generated projective modules over $R$, with the monoid operation given by the direct sum. Then $K_0$ is a covariant functor from rings to abelian groups.

Another construction that \textcolor{red}{carries the name Grothendieck group} is the following: Let $R$ be a finite-dimensional algebra over some field $k$ or more generally an artinian ring. Then define the Grothendieck group $G_0(R)$ as the abelian group generated by the set $\{[X]|X\in R\mathrm {-Mod} \}$ of isomorphism classes of finitely generated $R$-modules and the following relations: For every short exact sequence

    \[0\to A\to B\to C\to 0\]

of $R$-modules add the relation

    \[[A]-[B]+[C]=0\]

The abelian group defined by these generators and these relations is the Grothendieck group $G_0(R)$.

In the case where $R$ is a finite-dimensional algebra over some field, the Grothendieck groups $G_0(R)$ (defined via short exact sequences of finitely generated modules) and $K_0(R)$ (defined via direct sum of finitely generated projective modules) coincide. In fact, both groups are isomorphic to the free abelian group generated by the isomorphism classes of simple $R$-modules.(因为 $R$ 为 artinian, 所以不可分解投射模和单模一一对应.)

\end{document}