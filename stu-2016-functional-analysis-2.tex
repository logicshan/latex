\documentclass{ctexart}

\usepackage{amsmath,amssymb}

\newcommand{\norm}[1]{\left\lVert#1\right\rVert}
\newcommand{\abs}[1]{\left\lvert#1\right\rvert}

\DeclareMathOperator{\ind}{ind}

\begin{document}
\begin{enumerate}
\item 设 $X$ 是无限维的线性赋范空间, $X_0$ 是 $X$ 的有限维子空间. 证明:
存在 $x_0 \in X$, 满足:
\begin{enumerate}
\item $\norm{x_0} = 1;$
\item $d(x_0, X_0) \geq 1.$
\end{enumerate}

\item 设 $X$ 是无限维的线性赋范空间, 证明: $X$ 的单位球
\[
S = \{x \in X \mid \norm{x} \leq 1\}
\]
非紧.

\item 证明: $l^\infty$ 不可分. 这里
\[
l^\infty = \{\xi := (a_1, a_2, \cdots, a_n, \cdots) \mid \norm{\xi} :=
\sup_n\abs{a_n} < \infty\}.
\]

\item 设 $k(x,y) \in C([0,1] \times [0,1])$, 令 $T : C([0,1]) \to
C([0,1])$ 为
\[
Tu(x) = \int_0^1k(x,y)u(y)dy.
\]
证明: $T$ 是紧算子. (提示: 用 Arzela-Ascoli 定理.)

\item 设 $k(x,y) \in L^2([0,1] \times [0,1])$, 令
\[
Tu(x) = \int_0^1k(x,y)u(y)dy.
\]
证明:
\begin{enumerate}
\item $T$ 是从 $L^2([0,1])$ 到 $L^2([0,1])$的有界线性算子;
\item 求 $T^* : L^2([0,1]) \to L^2([0,1])$;
\item 证明: $T : L^2([0,1]) \to L^2([0,1])$ 是紧算子. (提示:
  $L^2([0,1])$ 的闭单位球是弱紧的.)
\end{enumerate}

\item 设 $X$ 是线性赋范空间, 证明: 恒同算子 $I : X \to X$ 是紧算子当且
  仅当 $X$ 是有限维的.

\item 设 $X$ 是无限维的线性赋范空间, $T : X \to X$ 是紧算子. 证明: $0
  \in \sigma(T)$.

\item 设 $T$ 为 $l^2$ 上的左移算子, 即
\[
T(a_1, a_2, \cdots) = (a_2, a_3, \cdots).
\]
\begin{enumerate}
\item 求 $\sigma_p(T)$, $\sigma_r(T)$ 和 $\sigma_c(T)$;
\item 求 $T^*$.
\end{enumerate}

\item 设 $X$ 和 $Y$ 均为 Banach 空间, $T : X \to Y$ 是 Fredholm 算子,
  证明:
\begin{enumerate}
\item $T^*$ 也是 Fredholm 算子;
\item $\ind{T^*} = -\ind{T}$.
\end{enumerate}

\item 设 $f \in \mathcal{S}'(\mathbb{R}^n)$, 满足
\[
(\tilde{\Delta} - 1)f = 0.
\]
证明: $f = 0$. 这里
\[
\tilde{\Delta}f = \sum_{i=1}^n{\tilde{\partial}_{x_i}^2f}.
\]

\item 设 $A$ 是 Banach 代数, $A^\times$ 表示 $A$ 中所有可逆元的全体.
  证明: $A^\times$ 是开集.

\item 令 $\Omega = \{z \in \mathbb{C} \mid 1 < \norm{z} < 2\}$,
\[
A = \{f \in C(\overline{\Omega}) \mid f \text{在}\Omega\text{上解析}\}.
\]
令 $\norm{f} = \max_{z \in \Omega}{\abs{f(z)}}$. 证明: $A$ 的 Gelfand谱 $\Delta$ 与 $\overline{\Omega}$ 同胚.
\end{enumerate}
\end{document}